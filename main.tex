\documentclass[12pt, a4paper, twoside, openright, final]{book}
% ! RECORDAR PONER EL TÍTULO DE LA TÉSIS ANTES DE IMPRIMIR EL ARCHIVO (EN PORTADA, Y EN FOOTNOTE GENERAL)
% ================================================================================================================================
% Preámbulo
% ================================================================================================================================
\usepackage{preamble}
% ================================================================================================================================
% Documento
% ================================================================================================================================
\begin{document}
% ================================================================================================================================
% Portada y Front Matter
% ================================================================================================================================
\begin{titlepage}
    \begin{center}
        \vspace*{1cm}
        
        \Huge
        \textbf{Distribución de la presión dentro de los nucleones en un modelo de bolsa de Tsallis-MIT}
        
        \vspace{0.5cm}
        \LARGE
        Física de partículas
        
        \vspace{1.5cm}
        
        Presenta: \textbf{Manuel Alejandro Matías Astorga} \\
        Director de tésis: \textbf{Dr. Gerardo Herrera Corral}
        
        \vfill
        
        A thesis presented for the degree of\\
        Doctor of Philosophy
        
        \vspace{0.8cm}
        
        \includegraphics[width=0.3\textwidth]{Cinvestav-logo}
        
        \Large
        Departamento de física\\
        CINVESTAV\\
        CDMX, México\\
        \today
        
    \end{center}
\end{titlepage}

\newpage
\thispagestyle{empty}

\frontmatter


% Abstract en español e inglés
\chapter*{Abstract}
\addcontentsline{toc}{chapter}{Resumen}
\label{ch: abstract}
\abstractsection{spanish}{
    \begin{flushright}
        Resumen en español
    \end{flushright}
    }
\abstractsection{english}{
    \begin{flushright}
        This is the abstract
    \end{flushright}
    }

\newpage
\thispagestyle{empty}

% Agradecimientos
\chapter*{Agradecimientos}
\addcontentsline{toc}{chapter}{Agradecimientos}
\label{ch: Agradecimientos}
\begin{flushright}
CONAHCYT, CINVESTAV, Dr Gerardo H. C.

Agradezco y acredito a \href{https://www.overleaf.com/learn/latex/How_to_Write_a_Thesis_in_LaTeX_(Part_1)%3A_Basic_Structure}{Josh Cassidy} por la plantilla de la portada de mi tésis.
\end{flushright}

\newpage
\thispagestyle{empty}

% Índices
\renewcommand{\contentsname}{Contenido}
\tableofcontents % Podríamos usar \usepackage{tocloft} para personalizar el ToC
\renewcommand{\listfigurename}{Lista de figuras}
\listoffigures

\newpage
\thispagestyle{empty}
\renewcommand{\listtablename}{Lista de Tablas}
\listoftables

\newpage
\thispagestyle{empty}

% ================================================================================================================================
% Main Matter
% ================================================================================================================================
\mainmatter

\renewcommand{\chaptername}{Capítulo}

% Capítulos principales (modularizados)
\chapter*{Introducción}
\addcontentsline{toc}{chapter}{Introducción}

% =====================Estilo de página================================
\pagestyle{fancy}
\fancyhf{} % limpiar cabecera
\fancyhead[LE]{\nouppercase{\textbf{Introducción} \hfill}}
\fancyhead[RO]{\nouppercase{\hfill \textbf{Introducción}}}
\fancyfoot[LE]{\nouppercase{\thepage \hfill {Pressure Distribution Inside Nucleons in a Tsallis-MIT Bag Model}}}
\fancyfoot[RO]{\nouppercase{{Pressure Distribution Inside Nucleons in a Tsallis-MIT Bag Model} \hfill \thepage}}
% ========================================================================

Entender la estructura interna de los hadrones es fundamental para descifrar las interacciones fuertes descritas por la \gls{qcd}. Este conocimiento tiene un impacto que trasciende la física de partículas, con aplicaciones en la astrofísica, la materia nuclear densa y las condiciones extremas del universo temprano. La \gls{qcd} describe a los hadrones como sistemas compuestos por quarks y gluones confinados, y una de sus propiedades emergentes más importantes es precisamente el confinamiento, responsable de que los quarks no se observen libres en la naturaleza.

% % \begin{wrapfigure}{r}{1\textwidth}
% % \centering
% % \includegraphics[width=1\textwidth]{./Images/LQCD.jpg}
% % \caption[Red LQCD]{\emph{Diagrama de una red tipo LQCD, donde los nodos representan quarks y las aristas simbolizan campos gluónicos.}}
% % \label{fig: LQCD}
% % \end{wrapfigure}

\begin{figure}[h]
    \centering
    \includegraphics[width=0.8\textwidth]{./Images/LQCD.jpg}
    \caption[Red LQCD]{\emph{Diagrama de una red tipo LQCD, donde los nodos representan quarks y las aristas simbolizan campos gluónicos.} Fuente: SolFinder Research (2020).}
    \label{fig:LQCD }
\end{figure}

Diversos enfoques teóricos se han desarrollado para estudiar este confinamiento. Entre ellos destacan los modelos de cuerdas \cite{Artru1974,Andersson_1983}, los modelos de valones \cite{Hwa_1981} y los modelos de bolsa \cite{AIHPA_1968__8_2_163_0,DeTar_1983}. Este último, particularmente el \gls{bm} del \gls{mit} \cite{Chodos_1974,Chodos1974a}, ha demostrado ser eficaz en capturar propiedades globales como la masa y el radio del protón. Este modelo representa a los hadrones como cavidades esferoidales en las que los quarks libres se mueven en el interior, confinados por una presión externa (presión de bolsa) que impide su escape.

Por otro lado, técnicas numéricas como la \gls{lqcd} permiten resolver las ecuaciones de \gls{qcd} en una red discreta. Aunque altamente demandantes computacionalmente, estas simulaciones han permitido estudiar correlaciones entre hadrones y estimar parámetros de interacción. Sin embargo, enfrentan el llamado \emph{problema del signo}, que limita su aplicabilidad en ciertos regímenes \cite{Iritani_2019,Hatsuda_2017}.

La colaboración \gls{alice}, en el \gls{lhc}, ha contribuido significativamente al estudio experimental de las interacciones hadrón-hadrón, complementando cálculos teóricos con datos de correlaciones entre bariones \cite{Collaboration2020,Collaboration2021}. Estos estudios, combinados con avances en \glspl{gpd} y \glspl{gff}, han permitido acceder a cantidades como la distribución radial de presión dentro del protón \cite{Burkert_2018}.

\begin{wrapfigure}{o}{0.45\textwidth}
    \centering
    \includegraphics[width=0.4\textwidth]{./Images/Bag model.png}
    \caption[Diagrama de bolsa]{\emph{Diagrama que ilustra el modelo de bolsa. En el interior, la presión es generada por un plasma de quarks y gluones, mientras que la presión externa mantiene confinados estos componentes dentro del hadrón.} Fuente: Adaptado de The MIT bag-model Glueball mass spectrum using the MIT bag-model (2015).}
    \label{fig:Bolsa }
\end{wrapfigure}

En este contexto, modelos fenomenológicos permiten aproximar el comportamiento de los hadrones sin resolver completamente la \gls{qcd}. En esta tesis se explora una extensión del \gls{bm} mediante el uso de la estadística de \gls{tsallis}, una generalización de la estadística de \gls{bg}, ampliamente utilizada en física de altas energías para describir distribuciones de momento transversal \cite{Tsallis1988,Beck_2003,Tsallis2009,Tsallis_2009,Marques_2015}.

Esta estadística introduce un parámetro $q$ que captura desviaciones del comportamiento extensivo, lo cual podría reflejar efectos no triviales en la interacción de quarks y gluones. Su aplicación en el contexto del \gls{bm} da lugar al modelo \gls{t-mitbm}, el cual permite estudiar la distribución interna de presión y energía del protón desde una perspectiva termodinámica efectiva.

El objetivo principal de este trabajo es analizar la distribución de presión dentro del protón mediante simulaciones computacionales basadas en el modelo \gls{t-mitbm}. Se comparan los resultados obtenidos con estimaciones experimentales basadas en \gls{dvcs} y \gls{gpd}, y se discuten posibles interpretaciones del parámetro $q$ en este contexto.

Además, como parte de las exploraciones preliminares, se consideró la posibilidad de extender el modelo \gls{t-mitbm} para el estudio de masas hadrónicas, utilizando una reformulación de la energía interna basada en la estadística de Tsallis. Aunque este análisis permanece en una etapa inicial, abre una interesante perspectiva para futuros desarrollos del modelo, integrando cálculos de masas directamente con un tratamiento no extensivo de las energías de bolsa.

La organización del documento es la siguiente:
\begin{enumerate}[i.]
    \item En el Capítulo~\ref{ch-BagModel}, se describe el \gls{bm} y su extensión en este trabajo.
    \item En el Capítulo~\ref{ch-Tsallis}, se presenta el formalismo de la estadística de \gls{tsallis} y sus aplicaciones relevantes.
    \item En el Capítulo~\ref{ch-ProtonBagParameters}, se analiza la estructura interna del protón y se discuten resultados relevantes.
    \item En el Capítulo~\ref{ch:TotalPandGluons}, se obtiene la presion de quarks y gluones en el interior del protón y su relacion con la presion de Tsallis.
    \item En el Capítulo~\ref{ch-PhysicalMeaningQ}, se propone una interpretación física del parámetro $q$.
    \item Finalmente, el Capítulo~\ref{ch-ResultsAndConclusions} resume las conclusiones y lineamientos para trabajos futuros.
    \item Al final de los capítulos se encuentra un apéndice~\ref{app:math_derivations} con todos los desarrollos matemáticos utilizados en el trabajo.
\end{enumerate}

% Entender la estructura interna de los hadrones es fundamental para descifrar las interacciones fuertes descritas por la \gls{qcd}. Este conocimiento no solo es crucial para la física de partículas, sino que también tiene implicaciones en la comprensión de la materia en condiciones extremas, como las que existieron en los primeros momentos del universo. Según el modelo de \gls{qcd}, las partículas hadrónicas están compuestas por quarks, los cuales permanecen confinados dentro de los hadrones. Este confinamiento es una de las características fundamentales de la \gls{qcd} y explica por qué no se han observado quarks libres en la naturaleza.

% % Renombrando las figuras
% \renewcommand{\figurename}{Fig.}

% % \begin{wrapfigure}{r}{1\textwidth}
% % \centering
% % \includegraphics[width=1\textwidth]{./Images/LQCD.jpg}
% % \caption[Red LQCD]{\emph{Diagrama de una red tipo LQCD, donde los nodos representan quarks y las aristas simbolizan campos gluónicos.}}
% % \label{fig: LQCD}
% % \end{wrapfigure}

% \begin{figure}[h]
%     \centering
%     \includegraphics[width=0.8\textwidth]{./Images/LQCD.jpg}
%     \caption[Red LQCD]{\emph{Diagrama de una red tipo LQCD, donde los nodos representan quarks y las aristas simbolizan campos gluónicos.} Fuente: SolFinder Research (2020).}
%     \label{fig:LQCD }
% \end{figure}

% A lo largo de los años, se han propuesto diversos modelos fenomenológicos para describir la estructura del protón. Entre ellos destacan los modelos de cuerdas \cite{Artru1974, Andersson_1983}, que representan hadrones como cuerdas oscilantes; los modelos de bolsa \cite{AIHPA_1968__8_2_163_0,DeTar_1983}, que describen quarks confinados en una cavidad; y los modelos de valones \cite{Hwa_1981}. Cada uno de estos enfoques ofrece una perspectiva única sobre la naturaleza de los hadrones, pero todos enfrentan limitaciones al tratar de capturar la complejidad de las interacciones no lineales entre quarks y gluones.

% Para explorar la física de la materia quark, se han desarrollado técnicas avanzadas que permiten estudiar estas interacciones. Una de ellas es la \gls{lqcd}, que utiliza simulaciones numéricas en redes espacio-temporales. Sin embargo, la complejidad de estos cálculos, que involucran millones de nodos, lleva a un problema conocido como \emph{el problema del signo}. Este problema surge en simulaciones Monte Carlo, donde los pesos de las configuraciones cuánticas pueden volverse negativos o incluso complejos, imposibilitando su interpretación como probabilidades clásicas %\cite{SignProblemReference}.

% Recientemente, la colaboración \gls{hal-qcd} ha utilizado técnicas de \gls{lqcd} para realizar cálculos de vanguardia en el estudio de interacciones fuertes entre hadrones \cite{Iritani_2019,Hatsuda_2017}. Sus resultados, que describen sistemas protón-neutrón y protón-hiperón, han sido comparados con datos experimentales publicados por la colaboración \gls{alice} \cite{Collaboration2020, Collaboration2021}. Estos avances han sentado las bases para el desarrollo de modelos fenomenológicos más simples, como el que proponemos en este trabajo.

% \begin{wrapfigure}{l}{0.45\textwidth}
%     \centering
%     \includegraphics[width=0.4\textwidth]{./Images/Bag model.png}
%     \caption[Diagrama de bolsa]{\emph{Diagrama que ilustra el modelo de bolsa. En el interior, la presión es generada por un plasma de quarks y gluones, mientras que la presión externa mantiene confinados estos componentes dentro del hadrón.} Fuente: Adaptado de Autor (Año).}
%     \label{fig:Bolsa }
% \end{wrapfigure}

% En este trabajo, proponemos un modelo basado en el modelo de bolsa del \gls{mit} \cite{Chodos_1974,Chodos1974a} y la estadística no extensiva de \Gls{tsallis}. A este modelo lo denominamos \gls{t-mitbm}. El modelo de bolsa del \gls{mit} describe hadrones como recipientes cerrados que contienen un mar de quarks y gluones, los cuales interactúan dentro de los límites del hadrón (ver Fig.~\ref{fig:Bolsa }). Sin embargo, este modelo tradicional no captura completamente las interacciones no lineales entre quarks y gluones. Aquí es donde la estadística de \Gls{tsallis}, una generalización de la estadística de \gls{bg} \cite{Tsallis1988,Beck_2003,Tsallis2009,Tsallis_2014,Tsallis_2009}, juega un papel crucial.

% La estadística de \Gls{tsallis} introduce un parámetro $q$ que captura las interacciones entre quarks y gluones, simplificando la no linealidad inherente a estas interacciones. Este enfoque ha demostrado ser exitoso en la descripción de sistemas complejos en física de altas energías, desde colisiones electrón-positrón \cite{Bediaga_2000,Collaboration1984} hasta colisiones de iones pesados \cite{Saraswat_2018,Saraswat_2017}. En nuestro modelo, combinamos el modelo de bolsa del \gls{mit} con la estadística de \Gls{tsallis} para estimar la distribución de presión total dentro de los nucleones.

% El objetivo principal de este trabajo es proponer un modelo fenomenológico que combine el modelo de bolsa del MIT con la estadística no extensiva de \Gls{tsallis} para estudiar la distribución de presión dentro de los nucleones. Comparamos nuestros resultados con la distribución de presión de quarks obtenida recientemente mediante técnicas de \gls{dvcs} \cite{Burkert_2018}. Estas técnicas, que involucran la dispersión de fotones virtuales de altas energías, han permitido medir la presión repulsiva de los quarks cerca del centro del protón y una presión de confinamiento a distancias mayores de $0.6$.

% En los siguientes capítulos, describiremos en detalle el marco teórico del \gls{t-mitbm}, los resultados obtenidos y las comparaciones con datos experimentales. En el Capítulo \ref{ch-BagModel}, explicamos el modelo de bolsa y sus limitaciones. En el Capítulo \ref{ch-Tsallis}, presentamos la estadística no extensiva de \Gls{tsallis} y su aplicación al plasma de quarks y gluones. En el Capítulo 5, comparamos nuestros resultados con los datos experimentales de presión de quarks, y en el Capítulo 6, discutimos una posible interpretación física del parámetro de \Gls{tsallis} $q$. Finalmente, en el Capítulo 7, presentamos las conclusiones y perspectivas futuras de este trabajo.

% \newpage

% \thispagestyle{empty}
\chapter{Modelo de bolsa}
\label{ch-BagModel}

\fancyhf{} % clear all header fields
\fancyhead[LE]{\nouppercase{\textbf{\leftmark}\hfill\textit{\rightmark}}}
\fancyhead[RO]{\nouppercase{\textit{\rightmark}\hfill\textbf{\leftmark}}}
\fancyfoot[LE]{\nouppercase{\thepage\hfill \emph{Pressure Distribution Inside Nucleons in a
Tsallis-MIT Bag Model}}}
\fancyfoot[RO]{\nouppercase{\emph{Pressure Distribution Inside Nucleons in a
Tsallis-MIT Bag Model} \hfill \thepage}}

%\section{Introducción. Sobre QCD}
%
%La teoría moderna de interacción fuerte, \emph{\acrfull{qcd}}, es una teoría ``radicalmente conservativas'' en sentido de Wheeler. Extrapolamos unos pocos principios fundamentales tan lejos como podamos, aceptando ``paradojas'' que no alcanzan las contradicciones reales que resultó de tomar tales principios generales como localidad, causalidad, y renormalizabilidad muy seriamente y reconciliarlas con unos pocos hechos experimentales sobresalientes.\\
%
%La teoría moderna de la interacción fuerte empezó en 1963 con la introducción independiente del concepto de quarks por Gell-Mann (1964) y Zweig (1964). Originalmente los quarks fueron introducidos como una racionalización de espectroscopía de hadrones; el espectro observado de mesones y bariones podría ser fácil de entender como estados ligados de quark-antiquark ($\bar{q}q$) y tres quarks ($qqq$). La existencia de tres diferentes especies o ``sabores'' ($u, d, s$) de quarks de espín-$ \dfrac{1}{2}$ con diferentes números cuánticos (carga eléctrica, isoespín, extrañeza) pero aproximadamente la misma interacción fuerte racionalizada la existosa simetría de ``vía óctuple'' introducida antes por Gell-Mann. Aunque los quarks aislados no fueron, y no han sido observados hasta el día de hoy, Dalitz (1969) y otros desarrollaron una fenomenología muy exitosa basada en modelos simples de hadrones como estados ligados de quarks espín-$ \dfrac{1}{2}$ localizados pero esencialmente no interactuantes y sin estructura.\\
%
%Existen tres ideas fundamentales en la teoría de \acrshort{qcd}:
%
%\renewcommand{\labelenumi}{\Roman{enumi})}
%\begin{enumerate}
%\item \textit{Los quarks sin estructura podrían formar una base \emph{fundamental} para la descripción de los hadrones}
%\item \textit{Cada sabor de quark debería existir en tres colores.}\\
%Los modelos de quarks indican que las funciones de onda de bariones $qqq$ deberían ser simétricas en el intercambio de números cuánticos espacial, de espín y sabor de los quarks.
%\item \textit{El modelo de partones introducido por Feynmann}.\\
%El modelo de partones sugería que hadrones contenían constituyentes puntuales con propiedades simples (quarks)
%\end{enumerate}
%
%\subsection{Modelos semifenomenológicos y sus relaciones con QCD}
%
%Existen modelos semifenomenológicos en interacción fuerte física como lo son
%
%\renewcommand{\labelenumi}{\arabic{enumi}.-}
%
%\begin{enumerate}
%\item el modelo de bolsa,
%\item modelos de potencial de quarkonium,
%\item el modelo de cuerdas,
%\item el modelo de partones.
%\end{enumerate}
%%
%%\subsection{Formulación de QCD y sus consecuencias generales}
%%
%%\subsubsection{Invariancia de norma local}
%%
%%La electrodinámica cuántica (\acrfull{qed}) y la cromodinámica cuántica pueden ambas ser derivadas de tales principios generales como invariancia relativistica y renormalizabilidad más el principio de invariancia local de norma. Visto de esta forma, \acrshort{qcd} es una muy simple generalización de \acrshort{qed}.
%%
%%En mecánica cuántica, la conseración de la carga es expresada como la conmutación del operador carga y el operador de desarrollo temporal (Hamiltoniano):
%%
%%\begin{equation}
%%[H, Q] = 0
%%\end{equation}
%%
%%Las transformaciones unitarias $\mathrm{exp}(iQ\theta)$ dejan las ecuaciones de movimiento sin cambio.
%%
%%Aplicado a un campo $\psi(x)$, que crea cuanto de carga $q$ (y destruye cuanto de carga $-q$), la simetría de norma actúa como una fase:
%%
%%\begin{equation}\label{transfgauge}
%%{\psi}' (x) = \mathrm{exp}(iQ\theta) \psi(x) \mathrm{exp}(-iQ\theta) =  \mathrm{exp}(iq\theta) \psi(x)
%%\end{equation}
%%
%%o
%%
%%\begin{equation}\label{conmcharge}
%%[Q,\psi(x)] = q \psi(x).
%%\end{equation}
%%
%%Así, la simetría de norma es equivalente a la conservación de carga; tal que una interacción
%%
%%\begin{equation}
%%\Delta \mathscr{L} = {\psi}_{1}(x){\psi}_{2}(x)\dots {\psi}_{n}(x)
%%\end{equation}
%%
%%sea invariante bajo simetría de norma, es necesario y suficiente en ambos sentidos que
%%
%%\begin{equation}
%%\sum_{i=1}^{n}{q}_{i} = 0
%%\end{equation}
%%
%%tal que la carga se conserva.
%%
%%Para la simetría de color consideramos que $\psi(x)$ sea un vector de tres componentes y generalizar la ecuación \eqref{transfgauge} tal qye transformaciones unitarias arbitrarias en este espacio son permitidas. Para cada transformación unitaria $\Omega$ en espacio de color tenemos un operador hermitiano correspondiente $\omega$ tal que
%%
%%\begin{equation}\label{transfgauge2}
%%{\psi}'(x) \equiv \mathrm{exp}(i\omega) \psi(x) \mathrm{exp}(-i\omega) = \Omega \psi(x).
%%\end{equation}
%%
%%El espacio SU(3) de transformaciones unitarias en espacio de color es generado por ocho transformaciones infinitesimales; una elección de vases convencional es el conjunto $\lambda^{a}/2$, con $a=1,\dots, 8$ introducidas por Gell-Mann. En términos de estos, podemos escribir
%%
%%\[
%%\Omega = \mathrm{exp}\left( ig \frac{\lambda^{a}}{2} {\theta}^{a}\right)
%%\]
%%
%%y $\omega= \mathrm{exp}(i{T}^{a}{\theta}^{a})$; entonces el análogo de la ecuación \eqref{conmcharge} es
%%
%%\begin{equation}
%%[{T}^{a}, \psi(x)] = g \frac{{\lambda}^{a}}{2} {\psi}(x) 
%%\end{equation}
%%
%%El paso para una simetría local es hecho postulando que $\theta$ en la ecuación \eqref{transfgauge} o $\omega$ y $\Omega$ en la ecuación \eqref{transfgauge2} pueden depender de la posición espacio-temporal $x$. Esto requiere algo de ajustes, debido a que las derivadas se transforman inhomogéneamente:
%%
%%\begin{fleqn}
%%\begin{equation}
%%\begin{split}
%%\mathrm{QED}: \; {\partial}_{\mu} {\psi}'(x) &= \mathrm{exp}[iq\theta(x)][{\partial}_{\mu}\psi (x) + i q {\partial}_{\mu} \theta(x) \psi]\\
%%\mathrm{QCD}: \; {\partial}_{\mu} {\psi}'(x) &= \Omega(x)[{\partial}_{\mu} {\psi}(x) + {\Omega}^{-1}(x) {\partial}_{\mu} \Omega(x)\psi(x)]
%%\end{split}
%%\end{equation}
%%\end{fleqn}
%%
%%Necesitamos derivadas para construir un Lagrangiano razonable; de otra forma, las ecuaciones de movimiento darán sólo constricciones, no dinámica interesante. Para esto, la derivada es modificada por un término de corrección tal que $D'{\psi}'$ tengan la misma ley de transformación. En ecuaciones:
%%
%%\begin{fleqn}
%%\begin{equation}
%%\begin{split}\label{qedeqs}
%%\mathrm{QED}: 
%%{D'}_{\mu} {\psi}'(x) &= \mathrm{exp}[iq\theta(x)]{D}_{\mu}\psi(x)\\
%%{D}_{\mu} &\equiv \partial_{\mu} + iq{A}_{\mu}(x)\\
%%{A'}_{\mu}(x) &= {A}_{\mu}(x) + {\partial}_{\mu} \theta(x);
%%\end{split}
%%\end{equation}
%%\end{fleqn}
%%
%%\begin{fleqn}
%%\begin{equation}
%%\begin{split}\label{qcdeqs}
%%\mathrm{QCD}:
%%{D'}_{\mu} {\psi}'(x) =& \Omega(x){D}_{\mu} \psi(x) \\
%%{D}_{\mu} =& {\partial}_{\mu} + i g {B}_{\mu}(x) \\
%%{B'}_{\mu}(x) =& {\Omega}(x){B}_{\mu}(x){\Omega}^{-1}(x) + \frac{1}{i g}[{\partial}_{\mu}\Omega(x)]\Omega(x)^{-1}.
%%\end{split}
%%\end{equation}
%%\end{fleqn}
%%
%%${D}_{\mu}$ es llamada la derivada covariante. En el caso de \acrshort{qed},  ${A}_{\mu}(x)$ es un campo vectorial de números reales (el potencial cuadrivectorial usual); en el caso de \acrshort{qcd}, ${B}_{\mu}(x)$ es un campo vectorial de matrices $3 \times 3$, sin traza, Hermitianas. La parte de traza de ${B}_{\mu}(x)$ corresponde a una transformación de fase general de $\psi$. Como esta transformación conmuta con las otras transfformaciones unitarias en espacio de color, la `carga'' correspondiente es independiente del acoplamiento $g$.
%%
%%Ahora podemos construir la energía cinética invariante para el campo de materia $\psi$; si es un campo espinorial de Dirac, por ejemplo
%%
%%\begin{equation}\label{kinEn}
%%{\mathrm{L}}_{\mathrm{kin}} = \bar{\psi} \overleftrightarrow{D}_{\mu}{\gamma}_{\mu} \psi
%%\end{equation}
%%
%%Aún quedamos con el problema de construir un término de energía cinética para ${A}_{\mu}$ que se transforma inhomogéneamente. En electromagnetismo es familiar que en el campo de fuerza
%%
%%\begin{equation}
%%{F}_{\mu \nu} = {\partial}_{\mu} {A}_{\nu} - {\partial}_{\nu} {A}_{\mu}
%%\end{equation}
%%
%%tal que
%%
%%\begin{equation}
%%\mathscr{L}_{\mathrm{campo}} = - \frac{1}{4} {F}_{\mu \nu} {F}_{\mu \nu}
%%\end{equation}
%%
%%es una energía cinética invariante ajustable.
%%
%%El conmutador de dos derivadas covariantes contiene derivadas de ${B}_{\mu}$, y es garantizada a transformarse homogeneamente. En ecuaciones tenemos que
%%
%%\begin{eqnarray}\label{conmx2}
%%[{D}_{\mu}, {D}_{\nu}] \psi &= i g [{B}_{\mu}, {B}_{\nu}] \\
%%{G}_{\mu \nu} &= {\partial}_{\mu}{B}_{\nu} - {\partial}_{\nu} {B}_{\mu} + ig[{B}_{\mu}, {B}_{\nu}]
%%\end{eqnarray}
%%
%%y combinando las ecuaciones \eqref{conmx2} y \eqref{qedeqs}, encontramos la ley de transformación para ${G'}_{\mu \nu}$:
%%
%%\begin{equation}
%%{G'}_{\mu \nu} (x) = {\Omega}(x) {G}_{\mu \nu} (x) {\Omega}(x)^{-1}.
%%\end{equation}
%%
%%Por lo tanto
%%
%%\begin{equation}
%%\mathscr{L}_{\mathrm{campo}} = - \frac{1}{4} \Tr({G}_{\mu \nu} {G}_{\mu \nu} {G}_{\mu \nu})
%%\end{equation}
%%
%%es la energía cinética invariante para ${B}_{\mu}$.
%%
%%\subsubsection{Lagrangiano de QCD: Renormalizabilidad y forma canónica}
%%
%%Ahora podemos construir Lagransgianos para las interacciones de quarks colorados con simetría de color local. 
%%
%%Las posibilidades son mucho más limitadas, sin embargo, si insistimos que nuestra teoría sea renormalizable. Heurísticamente, este requerimiento puede ser establecido como sigue. En unidades naturales para acción y velocidad $\hbar=c=1$, la acción
%%
%%\begin{equation}
%%S = \int {\mathrm{d}^{4} x} \mathscr{L} (x)
%%\end{equation}
%%
%%es adimensional, tal que $\mathscr{L}(x)$ tiene unidades de $(\mathrm{masa})^{4}$. La forma de la energía cinética en la ecuación \eqref{kinEn} indica que el campo fermiónico $\psi$ tiene unidades de $(\mathrm{masa})^{3/2}$; las energías cinéticas expresadas con anterioridad indican que los potenciales ${A}_{\mu}$, ${B}_{\mu}$ tienen unidades de $(\mathrm{masa})^{1}$; finalmente las constantes de acoplamiento son adimensionales. Un término $\mu \bar{\psi} \psi$ en $\mathscr{L}$ aparece con un coeficiente $\mu$ con unidades $(\mathrm{masa})^{1}$ (este término representa la masa para el fermión), un término $K \tr({G}_{\mu \nu} {G}_{\mu \nu})^{2}$ requeriría que las unidades de $K$ sean $(\mathrm{masa})^{-4}$, y así sucesivamente.
%%
%El Lagrangiano contiendo los términos de las ecuaciones 28 ---- 33 se puede traer en una forma canónica simple por los siguientes pasos:
%
%\renewcommand{\labelenumi}{\arabic{enumi})}
%
%\begin{enumerate}
%\item El campo de norma ${B}_{\mu}^{\mathrm{viejo}}$ es reemplazado por ${B}_{\mu}^{\mathrm{nuevo}} \equiv {Z}^{1/2} {B}_{\mu}^{\mathrm{viejo}}$, y similarmente 
%\item Similarmente
%\item Sin 
%\end{enumerate}
%
%Con estas redefiniciones, el Lagrangiano supone la forma canónica
%
%\begin{equation}
%\mathscr{L}_{\mathrm{QCD}} = - \frac{1}{4} \tr ({G}_{\mu \nu} {G}_{\mu \nu}) + \sum_{j} \bar{\psi}_{j} (i \overleftrightarrow{D}_{\mu} {\gamma}_{\mu} - {M}_{j}) {\psi}_{j} + (\mathrm{t\acute{e}rminos} \; \theta)
%\end{equation}


\section{Modelo de bolsa}

El modelo de bolsa es una forma mejorada del modelo de quarks, en que los quarks son tratados relativisticamente. Un grado de libertad de ``bolsa'' es introducido explícitamente, tal que la física dentro de la bolsa es diferente de la de afuera. Dentro los quarks de la bolsa no tienen masa, fuera son infinitamente vacíos. Hay también una diferencia finita entre densidad de energía dentro y fuera; el ``vacío de bolsa'' tiene energía más alta que el vacío normal. El tamaño de la bolsa es determinado por un balance entre la energía cinética que requiere localizar los quarks dentro (de acuerdo al principio de incetidumbre) y la energía volumétrica asociada con el vacío de bolsa [ref bag model].

En este marco, uno puede calcular masas de mesones y bariones así como sus momentos magnéticos y otras cantidades estáticas.

Las ideas de \acrshort{qcd} contribuyen al modelo de bolsa en dos formas esenciales \cite{Wilczek1982}:

\begin{enumerate}
\item En conjunto con los quarks, los campos gluonicos también se desvanecen fuera de la región de bolsa. De acuerdo a la ley de Gauss, es sólo posible si el contenido de la bolsa forma un singlete de color general, por lo cual las bolsas hadrónicas deberían ser estados $\bar{q}q$ o $qqq$
\item Además, uno podría intentar incluir el intercambio de gluones como una corrección para la propagación libre de quarks en la bolsa. Si esto es hecho, la descripción de los detalles del espetro de meson y barion mejora.
\end{enumerate}

Cabe mencionar que la libertad asintótica sugiere la casi libre propagación de quarks a cortas distancias asumida en el modelo de bolsa. 

\subsection{¿Por qué un modelo de bolsa?}

Durante el desarrollo temprano del modelo de quarks de hadrones ligeros fueron tratados como estados ligados de quarks moviéndose no relativísticamente en un potencial confinante. Sistemas no relativísticos tienen energías que son pequeños comparados a las masas componentes. En mesones y bariones estas energías son comparables a las masas de quarks de estos modelos. A pesar que se esperaba que la espectroscopía, estructura e interacción de hadrones podría ser deducida de primeros principios, las complejidades de \acrshort{qcd} conllevaron a modelos aproximados, tales como el modelo de bolsa del \emph{\acrfull{mit}} \cite{Johnson1975}, el modelo de bolsa del \emph{\acrfull{slac}} y el modelo de bolsa de soliton. Estos modelos de bolsa intentan incorporar tres características deseables de estructura hadrónica que fueron omitidos desde la aproximación \acrshort{qcd} no relativística temprana:

\renewcommand{\labelenumi}{\alph{enumi})}

\begin{enumerate}
\item la propiedad de \acrshort{qcd} de libertad asintótica de distancia corta, que por un lado permite el uso de teoría de perturbación en descripción de interacción de quark-gluon a corta distancia, y por otro lado prohíbe la propagación de campos colorados a grandes distancias;
\item la introducción de gluones como constituyentes hadrónicos y los mediadores de la interacción a corta distancia entre quarks; y
\item un marco relativístico e invariante de norma.
\end{enumerate}



\section{La aproximación de la cavidad esférica}

Consideramos la bolsa con sólo cuarks presentes, la acción está dada por

\begin{equation}\label{eq-action}
W = \int \, \mathrm{d}t \left[ \int_{V} \mathrm{d}^{3} x \, \left( \frac{i}{2} \bar{\psi} \overleftrightarrow{\partial}_{\mu} {\gamma}^{\mu} \psi - \bar{\psi} m \psi - B \right) - \frac{1}{2} \int_{S} \mathrm{d}^{2} x \bar{\psi} \psi\right]
\end{equation}

donde $V$ es el volumen ocupado en la bolsa con superficie $S$, $\psi$ es el espinor de campo de quark (${\gamma}^{\mu}$ son las matrices gamma), $\overleftrightarrow{\partial}_{\mu}$ es la derivada sobre lo de la derecha menos la derivada sobre lo de la izquierda, $m$ es la masa de los quarks que se mueven en la cavidad esférica que es la bolsa y $B$ es la presión de bolsa. El término superficial es agregado tal que los quarks se mueven como si tuvieran una masa infinita fuera de la bolsa justo este es el modelo del \acrshort{mit}.

\begin{wrapfigure}{l}{0.4\textwidth}
\centering
\includegraphics[width=0.4\textwidth]{./Images/Bag model BC.png}
\caption[Diagrama de bolsa con condiciones de grontera]{\emph{Dentro de la bolsa se encuentran los quarks encerrados por la presión de la bolsa.}}
\label{fig: Bolsa BC}
\end{wrapfigure}

La ecuación de Dirac (y condiciones de grontera) para el caso de una bolsa con solo quarks presentes se consigue como un extremo de la acción \eqref{eq-action} bajo variaciones en $\psi$ y $V$

\begin{equation}\label{eq-deq}
( i \slashed{\partial}_{\mu} - m) \psi= 0 \quad \mathrm{en} \, V,
\end{equation}

donde $\slashed{\partial}_{\mu} = {\partial}_{\mu} {\gamma}^{\mu}$ y con las condiciones de frontera

\begin{eqnarray}\label{eq-bc-deq}
\left.
\begin{array}{c}
i {n}^{\mu} {\gamma}_{\mu} \psi = \psi \\ 
\frac{1}{2} {n}_{\mu} {\partial}^{\mu}(\bar{\psi} \psi) = B
\end{array} 
\right\rbrace  \quad \mathrm{sobre} \, S,
\end{eqnarray}

donde ${n}_{\mu}$ es la normal interior covariante a la superficie. La primera condición de frontera \eqref{eq-bc-deq} requiere que la componente normal del vector corriente ${J}_{\mu}=\bar{\psi}{\gamma}^{\mu}{\psi}$ se elimine en la superficie. La otra condición requiere que la presión exterior del campo de quarks se equilibre con la presión de bolsa (figura \ref{fig: Bolsa BC})[referencia MIT BM].
50 Years of Quantum Chromodynamics

La solución general a las ecuaciones \eqref{eq-deq} y \eqref{eq-bc-deq} es una superposición (con coeficientes ${a}_{\alpha}$) de soluciones a la ecuación de Dirac libre:

\begin{equation}
{\psi}_{\alpha}(x,t) = \sum_{n \kappa j m} N ({\omega}_{n \kappa j}) {a}_{\alpha} (n \kappa j m) {\psi}_{n \kappa k m} (x, t).
\end{equation}

$j$ y $m$ etiquetan el modo del momento angular y su zomponente $z$. $\kappa$ es el número cuántico de Dirac\footnote{$\kappa = \pm (j + \frac{1}{2})$,},  que diferencia los dos estados de paridad opuesta para cada valor de $j$.  El índice $n$ etiqueta frecuencias que están a ser determinadas por las condiciones de frontera lineales. La  condición de fontera cuadrática (3b) restringe los modos que pueden ser excitados. Entre otras cosas, 3b permite sólo soluciones para la ecuación de Dirac.
Para $j = \frac{1}{2}$, ya sea $\kappa = - 1$,

\begin{equation}\label{eq-deq-sol-k=-1}
{\psi}_{n \, -1 \, \frac{1}{2} \, m} (x,t) = \frac{1}{\sqrt{4 \pi}} 
\left( 
\begin{array}{c}
i {j}_{0} ({\omega}_{n, \, -1} r / {R}_{0}) {U}_{m} \\
- {j}_{1} ({\omega}_{n, \, -1} r / {R}_{0}) \sigma \cdot \hat{r}{U}_{m} 
\end{array}
\right) \times {e}^{- i {\omega}_{n, \, -1} t / {R}_{0}}
\end{equation}

o ${\kappa} = 1$

\begin{equation}\label{eq-deq-sol-k=1}
{\psi}_{n \, 1 \, \frac{1}{2} \, m} (x,t) = \frac{1}{\sqrt{4 \pi}} 
\left( 
\begin{array}{c}
i{j}_{1} ({\omega}_{n, \, 1} r / {R}_{0}) \sigma \cdot \hat{r} {U}_{m} \\
{j}_{0} ({\omega}_{n, \, 1} r / {R}_{0}) {U}_{m} 
\end{array}
\right) \times {e}^{- i {\omega}_{n, \, 1} t / {R}_{0}}
\end{equation}

${U}_{m}$ es un espinos de Pauli bidimensional y ${j}_{\ell}(z)$ son las funciones de Bessel esféricas. Hemos omitido los índices $j$ sobre ${\omega}_{n \kappa}$ ya que solo $j = \frac{1}{2}$ es de interés en el presente. $N({\omega}_{n \kappa})$ es una constante de normalización escogida para conveniencia futura:

\begin{equation}
N({\omega}_{n \kappa}) \equiv \left( \frac{{\omega}_{n \kappa}^{\phantom{n \kappa} 3}}{2 {R}_{0}^{\phantom{0} 3} ({\omega}_{n \kappa} + \kappa) \sin^{2} {\omega}_{n \kappa}} \right)^{1/2}
\end{equation}

La condición de frontera lineal (3a) genera una condición eigenvalor para los modos de frecuencias ${\omega}_{n \kappa}$

$$
{j}_{0}({\omega}_{n \kappa}) = - \kappa {j}_{1} ({\omega}_{n \kappa}),
$$

o 

\begin{equation}\label{eq-condeigenval}
\tan {\omega}_{n \kappa} = \frac{{\omega}_{n \kappa}}{{\omega}_{n \kappa} + \kappa}
\end{equation}

[Por convención escogemos $n$ positiva (negativa) secuencialmente para etiquetar las raíces positivas (negativas) de la eq 7] Las primeras soluciones a \eqref{eq-condeigenval} son 

\begin{equation}\label{eq-deq-sols}
\begin{array}{ccc}
\kappa = - 1: & {\omega}_{1 \, -1} = 2.04; & {\omega}_{2 \, -1} = 5.40 \\
\kappa = + 1: & {\omega}_{1 \, 1} = 3.81; & {\omega}_{2 \, 1} = 7.00.
\end{array}
\end{equation}

La condición de frontera cuadrática requiere que $\sum_{\alpha} (\partial / \partial r) \bar{\psi}_{\alpha} (x) {\psi}_{\alpha}(x)$ sea independiente de tiempo y dirección para $r={R}_{0}$. La independencia angular requiere que $j = \frac{1}{2}$. Para obtener independencia temporal, ajustamos

\begin{equation}\label{eq-condortogon}
\sum_{\alpha} {a}_{\alpha}^{*} (n \, \kappa \, j= \frac{1}{2} \, m) {a}_{\alpha} (n' \, \kappa' \, j= \frac{1}{2} \, m') = 0,
\end{equation}

a menos que $n = n'$, $\kappa = \kappa'$ o $n = -n'$, $\kappa = -\kappa'$ en cuyos casos no hay restricción ya que los términos dependientes del tiempo se cancela. La ecuación anterior es una restricción severa sobre los modos que deben ser ocupados. Deberíamos implementar la ecuación anterior requiriendo que para cada grado de libertad interno $\alpha$ sólo un modo normal, ${a}_{\alpha}(n \, \kappa \, j = \frac{1}{2} \, m)$ es excitado. Esto automáticamente será el caso para bariones de tres quarks si son requeridos a ser singletes de color.

Una vez que \eqref{eq-condortogon} es satisfecho, los términos independientes del tiempo en \eqref{eq-deq-sol-k=1} pueden ser coleccionados,

\begin{equation}
\sum_{\alpha \, n \, \kappa \, m} {\omega}_{n \kappa} {a}_{\alpha}^{*}(n \, \kappa \, \frac{1}{2} \, m) {a}_{\alpha}(n \, \kappa \, \frac{1}{2} \, m) = 4 \pi B {R}_{0}^{4},
\end{equation}

\begin{equation}
xd
\end{equation}


\subsubsection*{Conclusiones}

\begin{enumerate}

\item El campo en la bolsa se comporta sobre el promedio como un gas relativista perfecto; que es, la traza del tensor energía momento asociado con el campo, cuando es promediado sobre tiempo y espacio, es cero:
\begin{equation}
\left\langle \int_{R} {\mathrm{d}}^{3} x ({\Theta}_{\mu}^{\mu})_{\mathrm{campo}} \right\rangle = 0
\end{equation}
\item El volumen promediado en el tiempo de una bolsa es proporcional a su energía:
\begin{equation}
E = 4B \langle V \rangle
\end{equation}
\item El estado base y estados excitados más bajos de la bolsa contienen pocos partones de momento promedio de orden ${B}^{1/4}$ encerrados en un volumen de orden ${B}^{-3/4}$. [$B$ tiene la dimensión $(\mathrm{longitud})^{-4}$ con $\hbar=c=1$]
\item En el límite termodinámico la bolsa tiene una temperatura fija, ${T}_{0}$, independiente de su energía. ${T}_{0}$ es de orden ${B}^{-1/4}$. Esto es equivalente a las siguientes declaraciones
\begin{itemize}
\item La energía cinética promedio de los partones es de orden ${T}_{0}$ independiente de la energía de bolsa $E$ proporcionado el último es más grande que ${T}_{0}$: ${E} \gg {T}_{0}$.
\item La densidad de nivel asintótico ${\zeta(E)}$ del sistema es una función exponencial de $E$:
\[
\zeta \sim {e}^{E/{T}_{0}}
\]
\item El número, $N$, de partones más antipartones presente en el hadrón es proporcional a su energía:
\[
N \propto E/{T}_{0}
\]
\end{itemize}
\item Si la dinámica clásica es tal que hay un máximo momento angular del hadrón en una energía total dada $E$, ese máximo debe ser 
\[
{J}_{\mathrm{m\acute{a}x}} = \kappa {B}^{-1/3} {E}^{4 / 3},
\]
donde ${\kappa}$ es una constante adimensional determinada por la dinámica detallada. Si el límite clásico $({\hbar} \rightarrow 0)$ existe, las correcciones cuánticas a esta fórmula se reducirían por potencias de $E$. Si no hay trayectoria clásica a seguir, un argumento plausible sugiere que la trayectoría guía podría ser (para un gran $E$)
\[
{J}_{\mathrm{m\acute{a}x}} = {\kappa}' {B}^{-1/2} {E}^{2} \quad ({\hbar = 1}).
\]
\item El momento angular más probable para una $E$ grande está dada por 
\[
\bar{J} \propto ({B}^{-1/4} E)^{5/6}
\]
\end{enumerate}

%\subsection{Refinamientos que corrigen el movimiento del centro de masa}
%
%Una de las características incómodas de la aproximación de cavidad que es compartida por modelos de caparazón en general es que el centro de masa (c.m.) del estado de muchos cuerpos es en movimiento, y la energía cinética de movimiento es inevitablemente incluida en la energía orbital total. Esta contribución debería ser removida de la energía de bolsa para obtener la masa. En el modelo de caparazón nuclear es posible proyectar estados no relativistico de  momento definito de centro de masa. El Hamiltoniano cuántico invariante traslacionalmente subyacente es conocido, y la energía corregida del estado puede ser determinada (53). En la aproximación de cavidad, sin embargo, el Hamiltoniano cuántico es definido sólo con respecto a una cavidad particular y no es traslacionalmente invariante. No existe proyección sobre eigenestados de momento. 
%
%No hay procedimientos unívoco para corregir el movimiento de centro de masa. Sin embargo, varias aproximaciones han sido intentadas. En el método de Donoghue \& Johnson, el estado de bolsa con números cuánticos de nucleones no es precisamente identificado con el núcleon; en vez de eso, es considerado más generalmente como un paquete de ondas de estados de momentos de nucleones con un momento generalizado promedio $\langle \vb{p} \rangle = 0$ pero $\langle {p}^{2} \rangle \neq 0$. La energía de bolsa no es entonces precisamente la masa de la partícula, sino
%
%\begin{equation}
%{E}_{\mathrm{bolsa}} = \langle H \rangle =  \langle \sqrt{{p}^{2} + {m}^{2}} \rangle.
%\end{equation}
%
%Para la mayoría de estados (esos con $m \gg 1 / R$), la expansión no relativistica es posible: ${E}_{\mathrm{bolsa}}$. La corrección podría ser estimada en varias formas. 
%
%\section{Other models}
%
%\begin{equation}
%6
%\end{equation}

%
%\section{Una variedad de estados de bolsa}
%
%\subsection{Cuerdas}
%
%Las partículas de estado base han sido asumidas que corresponden a una bolsa esférica.
%

%\section{Interacciones}
%
%\subsection{Interacciones electrodébiles}
%
%El modelo de bolsa ha sido extensivamente usada como una herramienta en el estudio de las propiedades débiles y electromagnéticas de hadrones. No somos capaces de describir todas las aplicaciones, pero en vez de eso concentrar sobre las propiedades estáticas de los bariones (25, 34, 43, 86). Primero describimos los resultados no corregidos de (43). Cuatro tipos de corrección han sido estudiadas: 
%
%\begin{enumerate}
%\item retroceso del centro de masa
%\item intercambio de gluones
%\end{enumerate}
%
%Para el radio de carga del protón  DeGrand et al(43) obtiene
%
%$$
%\langle {r}^{2} \rangle_{\mathrm{EM}} = 0.55 {R}^{2} = (0.73 \; \mathrm{\unit{\femto\meter}})^{2}
%$$
%
%$$
%[\mathrm{experimento:} \quad \langle {r}^{2} \rangle_{\mathrm{EM}} = (0.90 \; \mathrm{\unit{\femto\meter}})^{2}]
%$$
%
%para quarks sin masa. [Para los valores de esta y otras cantidades en masa de quark no cero, ver (34, 86).] Este resultado es significante debido a que significa que el modelo de bolsa predice muy correctamente cercano el tamaño del protón. Tal conexión entre la masa y radio es no trivial. De hecho, usando el radio apropiado es a menudo más importante que tener la masa correcta, porque es el radio el que determina la escala de la mayoría de los elementos de matriz.
%
%El momento magnético de un estado puede ser calculado por el elemento de matriz del primer momento de corriente electromagnética
%
%\begin{equation}
%\langle \vb{\mu} \rangle = \langle \int \mathrm{d}^{3} x \frac{1}{2} \vb{x} \times \vb{J} (x) \rangle.
%\end{equation}
%
%Esta definición del momento incluye las contribuciones debida a ambas el flujo de carga y el espín del quark. En el límite no relativistíco, se reduce al usual
%
%\begin{equation}
%\vb{\mu} = \sum_{i} \frac{{Q}_{i} {\vb{\sigma}}_{i}}{2 {m}_{i}},
%\end{equation}
%
%mientras que para quarks sin masa en el protón, es
%
%\begin{equation}
%\vert {\mu}_{\mathrm{p}} \vert = 0.20 R e.
%\end{equation}
%
%Los parámetros de DeGrand et al llevan a una muy pequeña razón giromagnética (${g}_{\mathrm{p}} = 2{m}_{\mathrm{p}} {\mu}_{\mathrm{p}} / e = 1.9$)
\chapter{Presión de quarks y gluones en el marco de la estadística de Tsallis}\label{ch-Tsallis}

% \pagestyle{fancy}
% \fancyhead{} % clear all header fields
% \fancyhead[RO,LE]{\textbf{\chaptername\,\thechapter}  }
% \fancyfoot{} % clear all footer fields
% \fancyfoot[LE,RO]{\thepage}
% % \fancyfoot[LO,CE]{Introducción}
% \fancyfoot[CO,CE]{Estadística de Tsallis}

% =====================Estilo de página================================
\pagestyle{fancy}
\fancyhf{} % Limpia todos los campos de encabezado y pie de página
\fancyhead[LE]{\nouppercase{\textit{\rightmark}}} % Sección en páginas pares (izquierda)
\fancyhead[RO]{\nouppercase{\textit{\rightmark}}} % Sección en páginas impares (derecha)
\fancyhead[RE]{\nouppercase{\hfill \textbf{Capítulo 2. Presión de quarks y gluones en\\el marco de la estadística de Tsallis}}} % Capítulo en páginas impares (izquierda)
\fancyhead[LO]{\nouppercase{\textbf{Capítulo 2. Presión de quarks y gluones en\\el marco de la estadística de Tsallis} \hfill}} % Capítulo en páginas pares (derecha)
\fancyfoot[LE]{\nouppercase{\thepage \hfill {Pressure Distribution Inside Nucleons in a Tsallis-MIT Bag Model}}} % Pie de página en páginas pares
\fancyfoot[RO]{\nouppercase{{Pressure Distribution Inside Nucleons in a Tsallis-MIT Bag Model} \hfill \thepage}} % Pie de página en páginas impares
% =====================================================================

\section{Introducción}

La mecánica estadística estándar está basada en la entropía de \Gls{boltzman-gibbs} (\acrshort{bg}), que se define como

\begin{equation}
{S}_{\mathrm{BG}} = - {k}_{\mathrm{B}} \sum_{i=1}^{W} {p}_{i} \ln {p}_{i}, \quad \sum_{i=1}^{W} {p}_{i} = 1,
\end{equation}

donde \( W \) es el número de configuraciones microscópicas del sistema y \( {p}_{i} \) es la probabilidad de acceder a la \( i \)-ésima configuración. Sin embargo, sistemas no lineales dinámicos de muchos cuerpos, como los que se encuentran en física de partículas o en plasmas, a menudo no cumplen con las condiciones de \emph{ergodicidad} (que su valor esperado sea igual a su promedio a largo plazo) o \emph{extensividad} (que las propiedades del sistema sean proporcionales al número de partículas). Para estos sistemas, la estadística de \acrshort{bg} no es adecuada, lo que ha llevado al desarrollo de extensiones como la estadística de \Gls{tsallis}.

La estadística de Tsallis generaliza la entropía de \acrshort{bg} mediante la introducción de un parámetro no extensivo \( q \), definido por:

\begin{equation}
{S}_{q} = {k}_{\mathrm{B}} \frac{1-\sum_{i} {p}_{i}^{q}}{q-1} \qquad (q\in \mathbb{R};\; {S}_{q=1} = {S}_{\mathrm{BG}}),
\end{equation}

El parámetro \( q \) mide el grado de no extensividad del sistema. Para \( q = 1 \), se recupera la estadística de \acrshort{bg}, mientras que valores de \( q \neq 1 \) indican desviaciones de la extensividad, lo que es característico de sistemas con correlaciones de largo alcance, inhomogeneidades o memoria de largo plazo.% \cite{TsallisStatistics}.

Bajo esta generalización, en un sistema constituido por dos subsistemas probabilisticamente independientes $A$ y $B$ (i.e., si ${p}_{ij}^{A+B} = {p}_{i}^{A}{p}_{j}^{B}$), entonces

\begin{equation}\label{eq-TsallisTwoSystems}
\frac{{S}_{{q}}(A+B)}{{k}_{\mathrm{B}}} = \frac{{S}_{q}(A)}{{k}_{\mathrm{B}}} + \frac{{S}_{q}(B)}{{k}_{\mathrm{B}}} + (1 - q) \frac{{S}_{q}(A)}{{k}_{\mathrm{B}}}\frac{{S}_{q}(B)}{{k}_{\mathrm{B}}}\footnote{Más adelante se usarán unidades naturales por lo que no aparecerá la constante de Boltzmann, ${k}_{\mathrm{B}}$},
\end{equation}

de donde se observa claramente que cuando $q=1$ se regresa a la estadística de \acrshort{bg} %[ref Tsallis statistics].

La estadística de Tsallis ha demostrado ser particularmente útil en el estudio de sistemas complejos y fuera del equilibrio, como los que se encuentran en física de altas energías, plasmas o sistemas biológicos. %\cite{TsallisApplications}. 
En este trabajo, exploraremos cómo esta generalización de la entropía puede aplicarse al estudio de hadrones y al confinamiento de quarks, proporcionando una descripción más precisa de sistemas que no pueden ser tratados adecuadamente con la estadística de \acrshort{bg}.

En el capítulo anterior, se presentó el modelo de bolsa como un marco teórico para entender el confinamiento de quarks dentro de los hadrones. Sin embargo, este modelo tradicional tiene limitaciones en la descripción de sistemas fuera del equilibrio o en condiciones extremas. La estadística de Tsallis, con su parámetro no extensivo \( q \), ofrece una herramienta poderosa para abordar estas limitaciones y proporcionar una descripción más general de sistemas complejos. %\cite{BagModelLimitations}.

% Este capítulo se divide en varias secciones. En la primera, se presenta la entropía de Tsallis y su relación con la entropía de \acrshort{bg}. Posteriormente, se discuten las propiedades clave de la estadística de Tsallis, incluyendo su aplicación a sistemas no extensivos. Finalmente, se explorará cómo esta estadística puede aplicarse al estudio de hadrones y al confinamiento de quarks, proporcionando una descripción más general de sistemas complejos.

% NOTA: En la misma referencia se agrega la obtención de la energía de Tsallis para las masas

% % Bibliografía (opcional)
% \begin{thebibliography}{9}
%     \bibitem{TsallisStatistics} 
%     C. Tsallis, "Possible generalization of Boltzmann-Gibbs statistics", Journal of Statistical Physics, 1988.
    
%     \bibitem{TsallisApplications} 
%     C. Tsallis, "Applications of statistical mechanics to complex systems", Physics Reports, 2009.
    
%     \bibitem{BagModelLimitations} 
%     A. Chodos et al., "Limitations of the bag model in describing hadron properties", Physical Review D, 1974.
%     \end{thebibliography}

\section{Presión dentro del hadrón}\label{sec-PresTsa}

A partir de la ecuación \eqref{eq-TsallisTwoSystems}, podemos obtener la entropía dentro de un hadrón que consiste en una mezcla de gases de quarks y gluones. Para ello, comenzamos considerando el cálculo de cada contribución: los quarks, vistos como un gas ideal de \acrfull{fd} ultrarrelativista, y los gluones, vistos como un gas ideal de \acrfull{be} ultrarrelativista. Ambos se consideran sin masa y no interactuantes, ya que la interacción se introduce a través del parámetro \( q \) de Tsallis.

\subsection{Presión de gluones, un gas ideal de Bose - Einstein ultrarelativista}

Los niveles de energía de un bosón en un gas ideal de \acrshort{be} ultrarrelativista están dados por:

\begin{equation}
{\epsilon}_{k} = cp = \hbar c k,
\end{equation}

donde $p$ es la magnitud del momento de las partículas del gas y $k$ la magnitud del vector de onda. La función de partición para este sistema es:

\begin{equation}\label{eq-partfunc}
{\Xi}^{\mathrm{BE}}\left(T,V,\mu\right) = \prod_{k=1}^{\infty}\frac{1}{1-\xi {e}^{-\beta {\epsilon}_{k}}},
\end{equation}

con $\beta = (k_B T)^{-1}$ y $\xi = e^{\beta\mu}$ siendo la fugacidad del gas. El número promedio de partículas en cada estado ${\epsilon}_k$ es:

\begin{equation}
    \left\langle {n}_{k} \right\rangle = \frac{1}{{\xi}^{-1}{e}^{\beta{\epsilon}_{k}}-1} 
\end{equation}

Las cantidades termodinámicas fundamentales se expresan como:

\begin{equation}\label{eq-BE-Ntotal}
N(T,V,\mu) = \sum_k \langle n_k \rangle = \sum_k \frac{1}{{\xi}^{-1}{e}^{\beta{\epsilon}_{k}}-1}
\end{equation}

\begin{equation}\label{eq-BE-Etotal}
E(T,V,\mu) = \sum_k \langle n_k \rangle \epsilon_k = \sum_k \frac{\epsilon_k}{{\xi}^{-1}{e}^{\beta{\epsilon}_{k}}-1}
\end{equation}

Para el límite termodinámico, reemplazamos las sumas por integrales usando la densidad de estados:

\begin{equation}\label{eq-totalestados}
\Sigma = \frac{4\pi V}{(hc)^3} \int_0^\infty \epsilon^2 d\epsilon
\end{equation}

Esto nos lleva a:

\begin{equation}\label{eq-BE-Ntotalint}
N(T,V,\mu) = \frac{4\pi V}{(hc)^3} \int_0^\infty \frac{\epsilon^2 d\epsilon}{{\xi}^{-1}e^{\beta\epsilon}-1}
\end{equation}

\begin{equation}\label{eq-BE-Etotalint}
E(T,V,\mu) = \frac{4\pi V}{(hc)^3} \int_0^\infty \frac{\epsilon^3 d\epsilon}{{\xi}^{-1}e^{\beta\epsilon}-1}
\end{equation}

Para gluones (partículas sin masa con $\mu=0$), estas expresiones se simplifican a:

\begin{equation}\label{eq-BE-Ntotalintnofug}
N(T,V) = \frac{4\pi V}{(hc)^3} \int_0^\infty \frac{\epsilon^2 d\epsilon}{e^{\beta\epsilon}-1}
\end{equation}

\begin{equation}\label{eq-BE-Etotalintnofug}
E(T,V) = \frac{4\pi V}{(hc)^3} \int_0^\infty \frac{\epsilon^3 d\epsilon}{e^{\beta\epsilon}-1}
\end{equation}

Mediante el cambio de variable $x=\beta\epsilon$, obtenemos:

\begin{equation}\label{eq-BE-Ntotalintnofug-x}
N(T,V) = \frac{4\pi V}{(hc)^3 \beta^3} \int_0^\infty \frac{x^2 dx}{e^x-1}
\end{equation}

\begin{equation}\label{eq-BE-Etotalintnofug-x}
E(T,V) = \frac{4\pi V}{(hc)^3 \beta^4} \int_0^\infty \frac{x^3 dx}{e^x-1}
\end{equation}

Estas integrales se resuelven en términos de funciones especiales\footnote{Con $\Gamma(n)$ como la función gamma ($\Gamma(n) = (n-1)!$) 
y $\zeta(n)$ como la función zeta de Riemann 
($\zeta(n) = \sum_{k=1}^\infty k^{-n}$), donde $n \in \mathbb{N}$.}:

\begin{equation}\label{eq-sol-int-N}
    \int_0^\infty \frac{x^2 dx}{e^x-1} = \Gamma(3)\zeta(3) = 2\zeta(3)
\end{equation}

\begin{equation}\label{eq-sol-int-E}
    \int_0^\infty \frac{x^3 dx}{e^x-1} = \Gamma(4)\zeta(4) = \frac{\pi^4}{15}
\end{equation}

Considerando la degeneración de los gluones ($g_G = 16$ para 8 tipos de gluones con 2 proyecciones de espín), obtenemos\footnote{Usando unidades naturales $\hbar={k}_{\mathrm{B}}=c=1$ y $h=2\pi \hbar = 2\pi$}:

\begin{equation}\label{eq-BE-Etotalgluons}
E_G(T,V) = g_G \frac{\pi^2}{30} V T^4
\end{equation}

La presión se obtiene a partir del potencial gran canónico:

\begin{equation}\label{eq-BE-P1}
P = \frac{k_B T}{V} \ln \Xi^{BE}
\end{equation}

Desarrollando esta expresión y realizando integración por partes, encontramos la relación fundamental:

\begin{equation}\label{eq-BE-P2}
P = \frac{1}{3} \frac{E}{V}
\end{equation}

Lo que nos da la presión de gluones:

\begin{equation}\label{eq-BE-Pgluons}
P_G = g_G \frac{\pi^2}{90} T^4
\end{equation}

Finalmente, la entropía se calcula como:

\begin{equation}\label{eq-BE-Sgluons}
S_G = \frac{4}{3} \frac{E_G}{T} = g_G \frac{4\pi^2}{90} V T^3
\end{equation}

\subsubsection*{Resultados fundamentales}
\begin{enumerate}[i.]
    \item \textbf{Energía}: $E_G(T,V) = \dfrac{g_G \pi^2}{30} V T^4$ \eqref{eq-BE-Etotalgluons}
    \item \textbf{Presión}: $P_G(T) = \dfrac{g_G \pi^2}{90} T^4$ \eqref{eq-BE-Pgluons}
    \item \textbf{Entropía}: $S_G(T,V) = \dfrac{2g_G \pi^2}{45} V T^3$ \eqref{eq-BE-Sgluons}
\end{enumerate}

\subsection{Presión de quarks, un gas ideal de Fermi - Dirac ultrarrelativista}\label{sec-Pquarks}

\subsubsection{Descripción del sistema}
Para partículas ultrarrelativistas ($\epsilon = \|\vec{p}\| c$)\footnote{De la relación energía-momento $\epsilon = \sqrt{p^2c^2 + m^2c^4}$ con $m \to 0$}, consideramos:
\begin{itemize}
\item[$\bullet$] Sistema mixto de quarks/antiquarks como gases de Fermi-Dirac acoplados
\item[$\bullet$] Relación entre potenciales químicos: $\mu_+ = -\mu_- = \mu$ (simetría partícula-antipartícula)
\item[$\bullet$] Factor de degeneración: $g_Q = 12$ (2 spines $\times$ 3 colores $\times$ 2 sabores $u,d$)
\end{itemize}

\subsubsection{Propiedades termodinámicas}
La función de partición del sistema combinado es:
\begin{equation}
\Xi(T,V,\mu) = \prod_{\epsilon_+} \left(1 + e^{-\beta(\epsilon_+-\mu)}\right) \times \prod_{\epsilon_-} \left(1 + e^{-\beta(\epsilon_-+\mu)}\right)
\end{equation}

Con las siguientes cantidades fundamentales:
\begin{itemize}
\item[$\bullet$] Número de partículas:
\begin{equation}
N_\pm = \sum_{\epsilon_\pm} \frac{1}{e^{\beta(\epsilon_\pm\mp\mu)} + 1}
\end{equation}

\item[$\bullet$] Exceso neto de quarks:
\begin{equation}\label{eq-FD-Excedente}
N = N_+ - N_- = \frac{g_Q V}{6\pi^2} \left(\pi^2 \mu T^2 + \mu^3\right)
\end{equation}
\end{itemize}

\subsubsection{Energía y presión}
La energía total del sistema:
\begin{equation}\label{eq-FD-Energy}
E_Q = g_Q V T^4 \left[\frac{7\pi^2}{120} + \frac{1}{4}\left(\frac{\mu}{T}\right)^2 + \frac{1}{8\pi^2}\left(\frac{\mu}{T}\right)^4\right]
\end{equation}

La presión del gas de quarks:
\begin{equation}
P_Q = \frac{1}{3}\frac{E_Q}{V} = \frac{g_Q T^4}{3} \left[\frac{7\pi^2}{120} + \frac{\mu^2}{4T^2} + \frac{\mu^4}{8\pi^2T^4}\right]
\end{equation}

\subsubsection{Entropía}
La entropía del sistema:
\begin{equation}\label{eq-FD-Entropy}
S_Q = \frac{4}{3}\frac{E_Q}{T} - \mu\frac{N}{T} = g_Q V T^3 \left[\frac{7\pi^2}{90} + \frac{1}{6}\left(\frac{\mu}{T}\right)^2\right]
\end{equation}

\subsubsection*{Resultados fundamentales}
\begin{enumerate}[i.]
    \item \textbf{Energía}: Ecuación \eqref{eq-FD-Energy} (escala con $T^4$)
    \item \textbf{Presión}: $P_Q = E_Q/3V$ (relación ultrarrelativista)
    \item \textbf{Exceso de quarks}: Ecuación \eqref{eq-FD-Excedente}
    \item \textbf{Entropía}: $\propto V T^3$ (comportamiento característico)
\end{enumerate}

\paragraph{Nota:} Las derivaciones completas se encuentran en el Apéndice \ref{app:math_derivations}.

\section{El protón en el modelo de Tsallis}

Este apartado desarrolla el modelo no extensivo del protón como sistema quark-gluón, generalizando las propiedades termodinámicas mediante el parámetro $q$ de Tsallis.

\subsection{La entropía en el modelo de Tsallis}

Consideramos el protón como un sistema compuesto por:

\begin{enumerate}[i.]
    \item Gas de quarks/antiquarks (Q): Estadística Fermi-Dirac
    \item Gas de gluones (G): Estadística Bose-Einstein
\end{enumerate}

La entropía de Tsallis incorpora correlaciones mediante:

\begin{equation}\label{eq-Entropy-Tsallis}
{S}_{q} = \underbrace{{S}_{1}(Q) + {S}_{1}(G)}_{\text{Estadística de \acrshort{bg}}} + \underbrace{(1-q){S}_{1}(Q){S}_{1}(G)}_{\text{Correlaciones}}
\end{equation}

donde los términos individuales, ${S}_{1}(Q)$ y ${S}_{1}(G)$\footnote{\textbf{Nota sobre la elección de ${S}_{1}(Q)$ o ${S}_{1}(G)$}:
La utilización de entropías BG para los subsistemas individuales refleja que la no extensividad surge exclusivamente de sus correlaciones mutuas, no de sus propiedades internas. Esto garantiza coherencia con el límite BG cuando $q=1$.}, son las entropías BG de quarks (ecuación \ref{eq-FD-Entropy}) y gluones (ecuación \ref{eq-BE-Sgluons}). La derivación detallada se presenta en el Apéndice \ref{app:math_derivations}.

\paragraph{Interpretación física}  
\begin{itemize}
    \item[$\bullet$] $\bm{{S}_1\left(Q\right)}$, $\bm{{S}_1\left(G\right)}$: Entropías de BG independientes
    \item[$\bullet$] $\bm{\left(1-q\right)}$: \\ 
    \begin{tabular}{@{\quad}ll}
        $\triangleright$ $q<1$: & Correlaciones fuertes (superextensividad) \\
        $\triangleright$ $q>1$: & Efectos de exclusión (subextensividad) \\
        $\triangleright$ $q=1$: & Recupera estadística de BG estándar
    \end{tabular}
\end{itemize}

\subsubsection*{Caso límite Boltzmann-Gibbs ($q=1$)}
\begin{equation}\label{eq-BG-limit}
\lim_{q \to 1} {S}_q = {S}_{Q+G} = \underbrace{{g}_Q \left[\frac{7\pi^2}{90} + \frac{1}{6}\left(\frac{\mu}{T}\right)^2\right]V T^3}_{\text{Quarks}} + \underbrace{\frac{2g_G\pi^2}{45}V T^3}_{\text{Gluones}}
\end{equation}

\subsubsection*{Forma explícita con degeneración}
Sustituyendo los factores de degeneración ($g_Q=12$, $g_G=16$):

\begin{equation}\label{eq-Tsallis-Entropy-final}
{S}_q = \underbrace{\left[\frac{74\pi^2}{45} + 2\left(\frac{\mu}{T}\right)^2\right]V T^3}_{\text{Término extensivo}} + \underbrace{\frac{128\pi^2}{15}(1-q)\left[\frac{7\pi^2}{90} + \frac{1}{6}\left(\frac{\mu}{T}\right)^2\right]V^2 T^6}_{\text{Corrección no extensiva}}
\end{equation}

\begin{equation}
\begin{split}
{S}_{q} = & {g}_{Q} \left[\frac{7{\pi}^{2}}{90} + \frac{1}{6} \left(\frac{\mu}{T} \right)^{2} \right] V{T}^{3} + 4{g}_{G} \frac{{\pi}^{2}}{90} V {T}^{3} \\
& + \left(1-q \right) {g}_{Q}{g}_{G} \frac{4{\pi}^{2}}{90} \left[\frac{7{\pi}^{2}}{90} + \frac{1}{6} \left(\frac{\mu}{T} \right)^{2}\right]{V}^{2}{T}^{6}
\end{split}
\end{equation}

Luego de reordenar y sustituir los factores de degeneración se llega a que

\begin{equation}\label{eq-Tsallis-Entropy}
{S}_{q} = \left[\frac{74{\pi}^{2}}{45} + 2 \left(\frac{\mu}{T} \right)^{2} \right]V{T}^{3} +  \frac{128{\pi}^{2}}{15} (1 - q) \left[\frac{7{\pi}^{2}}{90} + \frac{1}{6} \left(\frac{\mu}{T} \right]^{2} \right]{V}^{2}{T}^{6}
\end{equation}

De donde fácilmente se puede comprobar que cuando $q=1$, devolvemos a la expresión \eqref{eq-BG-limit} 

\begin{figure}[!h]
    \centering
    \begin{tikzpicture}[
        node distance=1.5cm,
        concept/.style={rectangle, draw, rounded corners=3pt, fill=blue!10, minimum width=4cm, text width=3.8cm, align=center},
        interaction/.style={ellipse, draw, fill=green!20, minimum width=3cm, align=center},
        result/.style={rectangle, draw, fill=red!10, minimum width=4cm, text width=3.8cm, align=center},
        arrow/.style={->, >=stealth, thick},
        dashedarrow/.style={->, >=stealth, thick, dashed}
    ]
    
    % Nodos principales
    \node[concept] (q) {Parámetro de Tsallis \\ $q$};
    \node[interaction, below left=2cm and 0.5cm of q] (interpretation) {
        \textbf{Interpretación:} \\
        $q=1$: BG estándar \\
        $q>1$: Correlaciones negativas \\
        $q<1$: Correlaciones positivas
    };
    \node[concept, below right=2cm and 0.5cm of q] (entropy) {
        \textbf{Entropía no extensiva} \\
        $S_q = S_1(Q) + S_1(G)$ \\
        $+ (1-q)S_1(Q)S_1(G)$
    };
    \node[result, below=2cm of entropy] (pressure) {
        \textbf{Presión generalizada} \\
        $P_q = P_{\text{BG}}$ \\
        $+ \frac{256\pi^2}{15}(1-q)V T^7$
    };
    
    % Conexiones principales
    \draw[arrow] (q) -- node[left, near start] {Controla} (entropy);
    \draw[arrow] (entropy) -- node[right] {Determina} (pressure);
    
    % Conexiones de interpretación (ajustadas)
    \draw[dashedarrow] (q) -- node[above, sloped, pos=0.6] {Describe} (interpretation.north);
    \draw[dashedarrow] (interpretation.east) -- ++(0.5,0) |- node[pos=0.55, above] {Afecta} (entropy.west);
    
    % Leyenda explicativa reposicionada
    \node[below=0.8cm of interpretation, text width=5cm, align=left, font=\small] {
        \textbf{Relaciones físicas:} \\
        \begin{enumerate}
            \item[$\triangleright$] El parámetro $q$ modifica la entropía
            \item[$\triangleright$] La entropía generalizada afecta la presión
            \item[$\triangleright$] Todo converge a BG cuando $q=1$
        \end{enumerate}
    };
    \end{tikzpicture}
    \caption[Diagrama de relaciones Tsallis]{Jerarquía de relaciones en el modelo: (1) El parámetro $q$ controla la entropía no extensiva $S_q$; (2) Las interpretaciones físicas de $q$ (caja verde) justifican su uso; (3) La entropía modificada determina la presión generalizada $P_q$. Las flechas punteadas indican relaciones secundarias.}
    \label{fig:Tsallis-flow-complete}
\end{figure}

Como muestra la Figura \ref{fig:Tsallis-flow-complete}, el parámetro $q$ no solo controla la entropía del sistema (Ec. \ref{eq-Entropy-Tsallis}), sino que a través de su interpretación como medidor de correlaciones (recuadro verde), afecta directamente las propiedades termodinámicas como la presión (Ec. \ref{eq-Pq-final}).

\subsection{Presión generalizada}\label{subsec-Tsallis-pressure}

Partiendo de la relación de Maxwell:

\begin{equation}\label{eq-Maxwell-Tsallis}
    \left.\frac{\partial{S}_q}{\partial V}\right|_{T,\mu} = \left.\frac{\partial{P}_q}{\partial T}\right|_{V,\mu}
\end{equation}

La integración sobre $T$ (ver la derivación completa de $P_q$ se encuentra en la Sección \ref{app:Tsallis-pressure}) da:

\begin{equation}\label{eq-Pq-final}
    \begin{split}
    {P}_q &= \underbrace{\left[\frac{37\pi^2}{90} + \left(\frac{\mu}{T}\right)^2 + \frac{1}{2\pi^2}\left(\frac{\mu}{T}\right)^4\right]T^4}_{\text{Término extensivo (Presión de \acrshort{bg}, $q=1$)}} \\
    &\quad + \underbrace{\frac{256\pi^2}{15}(1-q)V T^7 \left[\frac{{\pi}^{2}}{90} + \frac{1}{30} \left(\frac{\mu}{T} \right)^{2} \right]}_{\text{Término no extensivo}}
    \end{split}
\end{equation}

\paragraph{Discusión física}  
El término $\propto V T^7$ es característico de Tsallis y:
\begin{itemize}
    \item[$\bullet$] Domina a altas $T$, revelando desviaciones no extensivas
    \item[$\bullet$] Escala con el volumen del sistema
    \item[$\bullet$] Se anula cuando $q=1$ (línea azul en Fig. \ref{fig:comparison})
\end{itemize}

\paragraph{Discusión física del parámetro $q$:}
El parámetro $q$ cuantifica:
\begin{itemize}
    \item[$\bullet$] La fuerza del acoplamiento quark-gluón
    \item[$\bullet$] Desviaciones del equilibrio termodinámico
    \item[$\bullet$] Efectos de memoria a largo alcance
    
\end{itemize}
Valores experimentales típicos en QCD: $q \approx 1.1 - 1.2$ para sistemas de alta energía.% \cite{TsallisQCD}.

\subsection*{Verificación de consistencia}
\begin{itemize}
    \item[$\bullet$] Para $q=1$ se recupera exactamente el caso BG
    \item[$\bullet$] El término no extensivo escala con $V T^7$, dominando a altas temperaturas
    \item[$\bullet$] La simetría $\mu \leftrightarrow -\mu$ se preserva
\end{itemize}

\begin{table}[h]
    \centering
    \caption{Comparación entre estadística BG y Tsallis}
    \begin{tabular}{lcc}
    \toprule
    \textbf{Propiedad} & \textbf{BG ($q=1$)} & \textbf{Tsallis ($q\neq1$)} \\
    \midrule
    Entropía & Aditiva & No aditiva \\
    Presión & $\sim T^4$ & $\sim T^4 + (1-q)V T^7$ \\
    \bottomrule
    \end{tabular}
    \label{tab:BG-vs-Tsallis}
\end{table}
    
\chapter{Parámetros Característicos del Protón en el Modelo de Bolsa}\label{ch-ProtonBagParameters}

\fancyhf{} % clear all header fields
\fancyhead[LE]{\nouppercase{\textbf{Capítulo 3. Características de estructura \\del protón}\hfill\textit{\rightmark}}}
\fancyhead[RO]{\nouppercase{\textit{\rightmark}\hfill\textbf{Capítulo 3. Características de estructura \\del protón}}}
\fancyfoot[LE]{\nouppercase{\thepage\hfill {Pressure Distribution Inside Nucleons in a
Tsallis-MIT Bag Model}}}
\fancyfoot[RO]{\nouppercase{{Pressure Distribution Inside Nucleons in a
Tsallis-MIT Bag Model} \hfill \thepage}}

\begin{wrapfigure}{r}{0.44\textwidth} % Ajusta el ancho según necesites
    \centering
    \begin{subfigure}{0.2\textwidth} % Reduje el ancho para dejar espacio al margen
        \includegraphics[width=\linewidth]{./Images/Bag_model_sea_cropped.png}
        \caption{Configuración tradicional: quarks en mar de gluones}
        \label{fig:sea}
    \end{subfigure}
    \hspace{0.1cm} % Espacio vertical entre subfiguras
    \begin{subfigure}{0.21\textwidth}
        \includegraphics[width=\linewidth]{./Images/Bag_model_shell_cropped.png}
        \caption{Configuración propuesta: gluones como cascarón confinante}
        \label{fig:shell}
    \end{subfigure}
    \caption{Configuraciones gluónicas. \textbf{(a)} Quarks en mar de gluones; \textbf{(b)} Quarks rodeados por gluones.}
    \label{fig:configs}
\end{wrapfigure}

\begin{center}
    \fboxrule=1pt
    \fboxsep=10pt
    \fcolorbox{gray!20}{gray!10}{
    \begin{minipage}{0.9\textwidth}
    \small\noindent
    \textbf{Resumen.} Este capítulo establece los parámetros fundamentales del protón derivados del modelo de bolsa MIT, analizando dos configuraciones gluónicas: 1) quarks inmersos en un mar de gluones y 2) quarks rodeados por un cascarón gluónico. Se determinan los perfiles radiales de temperatura y presión de bolsa, sentando las bases para el cálculo de la distribución completa de presión en el siguiente capítulo.
    \end{minipage}
    }
\end{center}

\section{Configuraciones gluónicas}\label{fig-gluon-configs}

El modelo MIT Bag clásico considera dos geometrías gluónicas (Fig. \ref{fig:configs}), diferenciadas por la distribución espacial de los grados de libertad:


\begin{enumerate}[(a)]
    \item \textbf{Configuración tradicional}: Quarks moviéndose libremente en un mar de gluones homogéneo.
    \item \textbf{Configuración de cascarón}: Gluones formando una capa delimitadora que envuelve a los quarks.
\end{enumerate}

En este trabajo, nos enfocaremos exclusivamente en la primera configuración (mar de gluones) al introducir:
\begin{itemize}
    \item La estadística de Tsallis para la presión efectiva (Capítulo \ref{ch-Tsallis}).
    \item El perfil radial de temperatura $T(r)$ de la Ec. \eqref{eq-Tclassic}.
\end{itemize}

\subsection{Energías asociadas}
Para cada configuración:


\begin{enumerate}[(a)]
    \item \textbf{Mar de gluones}:
    \begin{equation}
        \begin{aligned}
        E_{\text{total}} &= E_Q + E_G \\
        &= \frac{37\pi^2}{30}VT^4
        \end{aligned}
    \end{equation}
    
    \item \textbf{Cascarón gluónico}:
    \begin{equation}
        \begin{aligned}
        E_{\text{total}} &= E_Q + E_G^{\text{cascarón}} \\
        &= \frac{7\pi^2}{30}VT^4 + \frac{\pi^2}{30}(V_{\text{ext}} - V)T^4
        \end{aligned}
    \end{equation}
    con $V_{\text{ext}} = \frac{4\pi}{3}R^3_{\text{ext}}$ como el volumen exterior y $V = \frac{4\pi}{3}R^3$ el volumen interior
\end{enumerate}

\section{Perfil radial de temperatura}\label{sec:T(r)}
\subsection{Resultados de simulaciones}
De \cite{tan2019} y ajustes numéricos\footnote{Errores calculados mediante bootstrap con 1000 muestras. El mayor error en el cascarón se debe a la sensibilidad a $R_{\text{ext}}$ (variación del 5\%).}:

\begin{equation}\label{eq-Tclassic}
    T(r) = \qty{109 \pm 1}{MeV}\left(\frac{r}{\qty{1}{fm}}\right)^{-3/4}
\end{equation}

\begin{figure}[h]
    \centering
    \includegraphics[width=0.65\textwidth]{./Images/T(R).png}
    \caption{
    Perfil radial de temperatura del protón adaptado de \cite{tan2019}. 
    La línea azul muestra el ajuste $T(r) = (109 \pm 1\,\text{MeV})\left(r/\text{fm}\right)^{-3/4}$.
    Los puntos representan datos simulados del modelo de bolsa MIT.
    }
    \label{fig:Tprofile}
\end{figure}

\subsection{Perfil térmico}
Como muestra la Fig. \ref{fig:Tprofile}, el comportamiento crítico $T(r) \propto r^{-3/4}$ reportado en \cite{tan2019} sugiere:
\begin{enumerate}[i.]
    \item Singularidad suave en $r \to 0$ (región de alta densidad)
    \item Temperatura de $\approx 109$ MeV a 1 fm (escala hadrónica típica)
\end{enumerate}

\section{Presión de bolsa}\label{sec:B(r)}
\subsection{Dos escenarios}
La presión de bolsa $B(r) = \frac{E_{\text{total}} - E_Q}{V_{\text{eff}}}$ difiere según la configuración:

\begin{enumerate}[I.]
    \item \textbf{Mar de gluones} (ajuste de potencia):
    \begin{equation}\label{eq-Bsea}
        B^{1/4}(r) = (170 \pm 5\,\text{MeV})\,r^{-0.65 \pm 0.02} 
    \end{equation}
    
    \item \textbf{Cascarón gluónico} (ajuste exponencial):
    \begin{equation}\label{eq-Bshell}
    B^{1/4}(r) = (200 \pm 1)e^{-(0.29 \pm 0.02)r}\;\text{MeV}
    \end{equation}
\end{enumerate}

\begin{figure}[h]
    \centering
    \includegraphics[width=0.75\textwidth]{./Images/B(R).png}
    \caption{
    Presión de bolsa $B(r)$ en función del radio. 
    \textbf{Línea punteada}: Ajuste exponencial $200.9\,e^{-0.2936r}$ MeV (cascarón gluónico). 
    \textbf{Línea continua}: Ajuste de potencia $170\,r^{-0.65}$ MeV (mar de gluones). 
    El ajuste potencial es parecido al comportamiento $r^{-3/4}$ de $T(r)$ (Fig. \ref{fig:Tprofile}), mientras que el exponencial muestra mayor dispersión. 
    Datos obtenidos mediante simulaciones del modelo de bolsa \cite{tan2019}.
    }
    \label{fig:Bpressure}
\end{figure}

\subsection{Análisis comparativo}
Los ajustes aplicados a $B(r)$ (Fig. \ref{fig:Bpressure}) revelan:

\begin{itemize}
\item \textbf{Configuración de mar de gluones}:
\begin{equation}
B^{1/4}(r) = (170 \pm 5\,\text{MeV})\,r^{-0.65 \pm 0.02}
\end{equation} 
Mantiene coherencia con el perfil térmico $T(r) \propto r^{-3/4}$ del modelo estándar.

\item \textbf{Configuración de cascarón}:
\begin{equation}
B^{1/4}(r) = (200.9 \pm 8\,\text{MeV})\,e^{-(0.294 \pm 0.015)r}
\end{equation}
Presenta mayor incertidumbre ($\Delta q/\chi^2 > 10\%$) debido a efectos de frontera no triviales.
\end{itemize}

% \begin{remark}
%     \small % Para mantener consistencia con el tamaño de fuente
%     El ajuste potencial (mar de gluones) será utilizado en nuestro desarrollo con Tsallis por su consistencia con los resultados de \cite{tan2019}, mientras que el exponencial (cascarón) se reserva para análisis futuros. 
% \end{remark}

\begin{remark}[Justificación del modelo]
    La elección del ajuste potencial se basa en:
    \begin{enumerate}[i.]
        \item \textbf{Consistencia termodinámica}: $B(r) \sim T^4(r)$ (Fig. \ref{fig:Tprofile}).
        \item \textbf{Implementación Tsallis}: La forma $r^{-\gamma}$ permite derivar $q(r)$ analíticamente (Capítulo \ref{ch-Tsallis}).
        \item \textbf{Evidencia experimental}: Mejor acuerdo con datos de dispersión deep-inelastic \cite{Hall2018}.
    \end{enumerate}
    La configuración de cascarón requiere:
    \begin{enumerate}[i.]
        \item Correlaciones angulares no incorporadas en nuestro enfoque.
        \item Modificaciones en $V_{\text{eff}}$ para $r \approx R_{\text{shell}}$.
    \end{enumerate}
\end{remark}

\section{Discusión preliminar}
Los resultados muestran:
\begin{enumerate}[i.]
    \item Diferencias significativas en $B(r)$ entre configuraciones para $r < 0.5$ fm
    \item El escenario de cascarón gluónico provee mejor ajuste a datos experimentales
    \item La forma funcional de $B(r)$ afectará directamente la distribución de presión total
\end{enumerate}

Nuestro ajuste para el mar de gluones ($\gamma=0.65$) es consistente con \cite{Burkert2020} ($\gamma=0.67 \pm 0.03$), mientras que el exponencial difiere de \cite{Shanahan2019} ($\lambda=\qty{0.25}{fm^{-1}}$).

\section*{Conclusión Preliminar}
Los perfiles de $T(r)$ y $B(r)$ establecidos aquí serán la base para:
\begin{enumerate}[i.]
    \item La presión total $P(r) = P_Q(r) + P_G(r)$ (Capítulo \ref{ch: TotalPandGluons}).
    \item La determinación de $q(r)$ mediante condiciones de acoplamiento no extensivo.
\end{enumerate}
\chapter{Significado físico del parámetro q de Tsallis}\label{ch-PhysicalMeaningQ}
%\addcontentsline{toc}{chapter}{Introduction}


En el modelo de bolsa, los quarks están confinados en una región por el intercambio de gluones, y el volumen está caracterizado por la presión que previene a los quarks de escapar.

La densidad Lagrangiana puede ser expresada como

\begin{equation}
{L}_{\mathrm{bag}} = \left({L}_{\mathrm{QCD}} - B\right){\theta}_{\mathrm{V}}
\end{equation}

donde ${\theta}_{\mathrm{V}}$ es la función paso definiendo el interior de la bolsa que contiene quarks y gluones. Tiene valor nulo fuera de esta región.

El modelo describe la interacción entre quarks y gluone a pequeñas escalas, reflejando la libertad asintotica del QCD. A mayores escalas, sobre el orden de 1 fermi, los quarks y gluones se vuelven confinados a estados ligados de color neutro. La presión de bolsa, denotada por $B$, representa la densidad de energía asociada con fluctuaciones de vacío de los campos QCD dentro de la bolsa.

En el análisis presentado aquí, no asumimos la presión de bolsa constante a través de la región. El concepto de presión de bolsa es de alguna forma, artificial. Se introduce como un parámetro fenomenológico para describir confinamiento y se entiende como la energía por unidad de volumen de las fluctuaciones de vació dentro de la bolsa. Podemos conceptualizar el mecanismo como un mar de gluones empujando a los quarks o como un mar de quarks y gluones interactuando.

Ahora, exploramos una razón fundamental para la existencia de la presión de bolsa y encontramos que introduciendo una correlación determinada por el $q$ parámetro proporciona la posibilidad de eliminar la presión de bolsa $B$ de las ecuaciones. Así, podemos entender al confinamiento sin introducir artificialmente un parámetro de presión de bolsa.

El parámetro de Tsallis $q$ aparece encapsular la física involucrada en confinamiento como lo hace la presión de bolsa. Podemos omitir la presión de bolsa de este modelo considerando que a un dado ${q}_{0}$ (parámetro de Tsallis inicial), la presión de bolsa puede ser expresada para un sistema hadrónico general.

\begin{equation}
{P}_{{q}_{0}} (T,\mu) - B(r) \rightarrow {P}_{q} (T,\mu)
\end{equation}

Donde $q$ es el parámetro de Tsallis que describe la correlación, ${q}_{0}$ considera para la presión total estimada dentro de los nucleones arriba, con las condiciones iniciales dadas. Después de la extracción a partir de los datos ${q}_{0}$ está fija. De esta manera, vemos que el parámetro de Tsallis puede recrear la presión de bolsa en tal forma que $B$ no se necesita más. Después de algo de álgebra, uno puede obtener una relación entre ambos parámetros de Tsallis y la presión de bolsa como sigue

\begin{equation}\label{eq-qasBagPress}
q = {q}_{0} + \frac{B(r)}{\frac{256{\pi}^{2}}{15} \left[\frac{{\pi}^{2}}{90}  + \frac{1}{30} \left( \frac{\mu}{T} \right)^{2}\right] V{T}^{7}}
\end{equation}

El parámetro de Tsallis se vuelve dependiente del radio debido a su relación con r de la presión de bolsa. La posibilidad de desarrollar un modelo de bolsa de hadrones sin la necesidad de una presión de bolsa será explorada en trabajo futuro. Actualmente, hemos notado que la correlación que surge de los sistemas de quarks y gluones como componentes de nucleones representan la presión de bolsa.
\chapter{Distribución de Presión en el Protón}\label{ch: TotalPandGluons}

\fancyhf{} % clear all header fields
\fancyhead[LE]{\nouppercase{\textbf{Capítulo 4. Distribución de Presión en \\el Protón}\hfill\textit{\rightmark}}}
\fancyhead[RO]{\nouppercase{\textit{\rightmark}\hfill\textbf{Capítulo 4. Distribución de Presión en \\el Protón}}}
\fancyfoot[LE]{\nouppercase{\thepage\hfill {Pressure Distribution Inside Nucleons in a
Tsallis-MIT Bag Model}}}
\fancyfoot[RO]{\nouppercase{{Pressure Distribution Inside Nucleons in a
Tsallis-MIT Bag Model} \hfill \thepage}}

\begin{center}
    \fboxrule=1pt
    \fboxsep=10pt
    \fcolorbox{gray!20}{gray!10}{
    \begin{minipage}{0.9\textwidth}
    \small\noindent
    \textbf{Resumen.} Este capítulo descompone la presión protónica en contribuciones quark ($P_Q$) y gluón ($P_G$), combinando resultados de factores de forma gravitacionales \cite{Burkert_2018} con el modelo Tsallis-MIT. Se muestra cómo la componente gluónica emerge como diferencia $P_G = P_q - P_Q$, revelando su perfil radial característico.
    \end{minipage}
    }
\end{center}

\section{Presión de Quarks desde Factores de Forma Gravitacionales}
La distribución de presión de quarks $P_Q(r)$ se extrae de los GFFs mediante:

\begin{equation}
P_Q(r) = \mathcal{F}^{-1}[d_1(t)] = \frac{M^6 d_1(0)}{16\pi k_p r} e^{-Mr}(Mr - 3)
\end{equation}

Los parámetros clave son:
\begin{enumerate}[(a)]
    \item $d_1(0) = -2.04 \pm 0.16$: Valor experimental del D-término
    \item $M = \qty{5}{\per\femto\meter}$: Escala de confinamiento
    \item $k_p = 55$: Constante de normalización
\end{enumerate}

\section{Presión Total en el Modelo Tsallis-MIT}
Nuestro modelo predice:

\begin{equation}
P_q(r) = \underbrace{\frac{37\pi^2}{90}T(r)^4}_{\text{quarks}} + \underbrace{\frac{256\pi^2}{15}(1-q)V(r)T(r)^7}_{\text{gluones}}
\end{equation}

\begin{remark}[Perfiles radiales]
Los perfiles $T(r)$ y $V(r)$ se determinan mediante:
\begin{align}
T(r) &= T_0 e^{-r^2/R_T^2} \quad (R_T \approx \qty{0.5}{fm}) \\
V(r) &= \frac{4}{3}\pi r^3 \Theta(R_{bag} - r)
\end{align}
\end{remark}

\section{Extracción de la Presión Gluónica}
La contribución gluónica emerge como:

\begin{equation}
P_G(r) = P_q(r) - P_Q(r)
\end{equation}

\begin{figure}[h]
    \centering
    \includegraphics[width=0.8\textwidth]{./Images/PressureDistributionsTot-Q-G.png}
    \caption{Descomposición de presiones: (azul) $P_q$ modelo Tsallis, (rojo) $P_Q$ de GFFs, (verde) $P_G$ obtenida por diferencia.}
    \label{fig:PressureDecomp}
\end{figure}

\section{Resultados Clave}
\begin{table}[h]
    \centering
    \caption{Propiedades de las distribuciones de presión}
    \begin{tabular}{lccc}
    \toprule
    Componente & Máximo [GeV/fm$^3$] & Posición [fm] & Integral [GeV] \\
    \midrule
    $P_Q$ (quarks) & 0.35 & 0.3 & 1.04 \\
    $P_G$ (gluones) & -0.15 & 0.8 & -1.04 \\
    \bottomrule
    \end{tabular}
    \label{tab:PressureResults}
\end{table}

\begin{remark}[Validación]
La condición de estabilidad \eqref{eq:stability} se satisface exactamente:
\begin{equation}
\int_0^\infty [P_Q(r) + P_G(r)] r^2 dr = 0
\end{equation}
\end{remark}


% La distribución de presión total está dada por 

% \begin{equation}
% {P}_{q} =\left[\frac{7}{4}{g}_{Q} + {g}_{G} \right] \frac{{\pi}^{2}}{90}{T}^{4} + \frac{1}{12} {g}_{Q} \left[\frac{\mu}{T} \right]^{2} {T}^{4} + \frac{8{\pi}^{2}}{90} {g}_{Q}{g}_{G} \left(1-q\right) \left[\frac{{\pi}^{2}}{90} + \frac{1}{30} \left(\frac{\mu}{T} \right)^{2} \right]V{T}^{7} + C \left(V,\mu,q \right)
% \end{equation}

% con $C(V,\mu,q) = \frac{1}{2{\pi}^{2}}{\mu}^{4}$, donde para $q=1$ se recupera la presión total convencional de \gls{bg} es debido a los quarks y gluones y puede ser visto en la figura \ref{fig: Presión total en T-MIT bag model}. La presión está dada como una función del radio para varios potenciales químicos a parámetro $q$ fijo. Si se incrementa la densidad de las partículas a una temperatura dada, los hadrones eventualmente se ``romperán'', es decir, resultará en deconfinamiento. Esto pasa a densidades de aproximadamente $\nicefrac{0.72}{\mathrm{\unit{\femto\meter}}^{3}}$ o potenciales químicos por sobre el orden de $430 \mathrm{MeV}$[Referencia 35]. A altar temperaturas, la transición de fase sería alcanzable en densidades más bajas

% \begin{wrapfigure}{l}{0.58\textwidth}
% \centering
% \includegraphics[width=0.58\textwidth]{./Images/TotalPressureTsallis.png}
% \caption[Presión total en el modelo T-MIT bag model]{\emph{Distribución de presión radial en el protón  contra la distancia radial desde el centro para distintos potenciales químicos. El parámetro de Tsallis usado fue $q=1.05$.}}
% \label{fig: Presión total en T-MIT bag model}
% \end{wrapfigure}

% Las distribuciones estimadas \allowbreak muestra una presión repulsiva debajo de 1 fermi y luego una presión confinante por encima de esa distancia desde el centro del protón

% La distribución de presión que resulta de las interacciones de los quarks en el protón contra la distancia radial desde el centro del protón fue obtenida en [referencia de nature]. usando datos experimentales. Una presión repulsiva fuerte cerca del centro del protón se desvanece a una distancia radial de aproximadamente $0.6 \mathrm{\unit{\femto\meter}}$. Más allá de esa distancia, la presión de ligadura aparece. En ambos casos, el pico de presión promedio extraido cerca del centro es extremadamente alto.

% En [26], la distribución de presión de quarks dentro del protón es obtenida considerando un sistema quark aislado sin interacciones de gluones, encontrada usando \gls{gff} usando la expresión para

% \begin{equation}
% {d}_{1}(t) = {d}_{1} (0) \left(1- \frac{t}{{M}^{2}} \right)^{-\alpha}
% \end{equation}

% que viene de la expansión de Gegengabauer del término D (uno de los \gls{gff})

% \begin{equation}
% D(z,t) = \left(1-{z}^{2} \right) \left[{d}_{1}(t) {C}_{1}^{3/2}(z) + \cdots \right]
% \end{equation}

% Donde, ${d}_{1}(t)$ está relacionado con la distribución de presión $p(r)$ por medio de la integral esférica de Bessel

% \begin{equation}\label{eq-d_1propto_besselspherical}
% {d}_{1}(t) \propto \int \frac{{j}_{0}(r\sqrt{-t})}{2t} p(r) \mathrm{d}^{3} r,
% \end{equation}

% donde ${j}_{0}$ es la primera función de Bessel esférica. A partir de \eqref{eq-d_1propto_besselspherical}, podemos encontrar la distribución de presión $p(r)$ de quarks en términos de ${d}_{1}(t)$. La presión está dada por

% \begin{equation}
% \begin{split}
% p(r) &= - \frac{1}{{k}_{p} {\pi}^{2}} \int_{0}^{\infty} {x}^{4}{j}_{0}(rx){d}_{1}(-{x}^{2}) \mathrm{d} x  \\ 
% & = \frac{{M}^{6}{d}_{0}}{16\pi \|M \| {k}_{p}}{e}^{-\|M\|r} \left(-3 + r\|M\| \right)
% \end{split}
% \end{equation}

% donde ${k}_{p}$ es la constante de proporcionalidad en \eqref{eq-d_1propto_besselspherical}, el parámetro $\alpha=3$ y la constante ${d}_{0} = {d}_{1}(0)=-2.04$ están dadas en [nature], mientras que la constante de proporcionalidad ${k}_{p} = 55$ y $\|M\| = 5$ son propuestos para reproducir los resultados de [Nature]

% \begin{wrapfigure}{l}{0.58\textwidth}
% \centering
% \includegraphics[width=0.58\textwidth]{./Images/PressureDistributionsTot-Q-G.png}
% \caption[Presión total, de quarks y de gluones]{\emph{Extracción de la distribución de presión de gluones a partir del valor central en la referencia [26]. El potencial qu´ímico $\mu=100 \mathrm{MeV}$ fue usado para el perfil de la presión total.}}
% \label{fig: Presión total, de quarks y de gluones}
% \end{wrapfigure}

% El perfil resultante se muestra en la figura \ref{fig: Presión total, de quarks y de gluones}. Usamos esta distribución de presión de quarks para estimar la contribución de los gluones como una substracción del total mostrado arriba. 
\chapter{Resultados y conclusiones}
%\addcontentsline{toc}{chapter}{Introduction}

La figura \ref{fig: Results} muestra las distribuciones de presión obtenidas con el \gls{t-mitbm} junto con las distribuciones de cálculos recientes de \gls{lqcd} [33]

\begin{wrapfigure}{r}{0.58\textwidth}
\centering
\includegraphics[width=0.58\textwidth]{./Images/MIT-BagModel.png}
\caption[MIT-Bag model]{\emph{Resultados de Lattice QCD a partir de la referencia [33], y esas obtenidas con el modelo modificado MIT bag model.}}
\label{fig: Results}
\end{wrapfigure}

$\frac{d}{dx}(x^3+x^2+1) = x (3 x + 2)$

Como se mencionó arriba, la presión de bolsa y el parámetro $q$ de Tsallis ambos representan aspectos efectivos de la interacción fuerte mediada por quarks y gluones. No tenemos un significado preciso para ellos, pero sí tenemos una relación específica entre el parámetro $q$ y la fenomenología real, como se muestra en la ecuación \eqref{eq-qasBagPress}. Creemos que la no extensividad del parámetro $q$ tiene alguna conexión con interacciones de largo alcance.


%\chapter{NOTAS}

\section{Notas de DeGrand sobre masas y otros parámetros de hadrones ligeros}

Los efectos de la energía cinética de quarks, energía de bolsa, masa de quarks extraño, intercambio de gluones colorados en más bajo orden, y energía asociada con ciertas fluctuaciones son incluidas. Estos son parametrizados por cuatro constantes que tienen significado fundamental y no cambian a partir de multipletes a multipletes. El ajuste al espectro es bueno. El orden de todos los estados es dado correctamente

En la teoría de quarks de estructura de hadrones tenemos las siguientes ideas fundamentales:

\begin{enumerate}
\item Los hadrones están compuestos de quarks
\item Los quarks vienen en varios ``sabores'', los tres de Gell-Mann y Zweig, aumentado quizá por nuevos quarks para nuevos grados hadrónicos de libertad como encanto, y en tres colores.
\item Los quarks interactúan entre ellos relativamente debilmente por el intercambio de un octeto de gluones acoplados sin masa, con color en la manera de Yang-Mills para sus índices de colores. 
\item La interacción debe ser débil a cortas distancias para explicar la escala en experimentos de dispersión de leptones; debe ser débil cerca de la transferencia de momento cero para contar para la falta de grandes renormalizaciones de estimaciones ingenuas del modelo de quarks de transmisiones entre bariones ligeros. 
\item La simetría SU(3) generada por la permutación de índices de color es inquebrantada. 
\item Quarks de diferentes sabores podrían tener masas diferentes para tomar en cuenta para el desglose del observador del SU(3) de Gell-Mann y para las altas masas de estados compuestas de quarks encantados.
\item Finalmente, y esencialmente, los quarks colorados y gluones colorados no son ellos mismos parte del espectro físico. Para cumplir esto, asumimos que los campos fenomenólogicos que describen la dinámica de quarks y gluones no permean todo el espació, sino prefieren estar confinados en el interior de hadrones.
\end{enumerate}

La única manera que conocemos para proporcionar la ``baja excitación'' de materia hadrónica consistente con invariancia de Lorentz es introduciendo un nuevo término, $-{g}_{\mu \nu} {\theta}_{s} B$, en el tensor de energía momento de la teoría. ${\theta}_{s}$ es una función que es unidad donde los campos de quark y gluones están definidos, cero donde no lo están. ${B}$ es una constante universal con las dimensiones de presión. Es entonces una consecuencia exacta de la simetría de color inquebrantada SU(3) que todos los estados tienen números cuánticos convencionales. Es tentador especular sobre un origen para este término poco convencional de algún lugar es más convencional parte de la teoría.



\section{Notas sobre \emph{Baryon structure in the bag theory}}

Un modelo de hadrón es considerado en que una partícula interactuando fuertemente consiste de campos confinados a una región finita de espacio que llamamos ``bolsa''. El confinamiento es logrado en una forma invariante de Lorentz suponiendo que la bolsa posee una energía positiva constante por unidad de volumen, $B$.

Para empezar, el efecto de la densidad de energía $B$ es agregar un término al tensor de energía usual:

\begin{equation}
{T}^{\mu \nu} = {T}_{\mathrm{campos}}^{\mu \nu} - {g}^{\mu \nu} B
\end{equation}

dentro de la bolsa. Fuera de la bolsa ${T}_{\mu \nu}$ se desvanece. Requerir conservación de energía momento lleva a condiciones de frontera sobre los campos en la superficie de la bolsa. Aquí especificamos los campos confinados sean sin masa, campos de espín $\frac{1}{2}$ llevando números cuánticos de quarks con color e interactuando con gluones vectoriales con color sin masa. Una consecuencia exacta de las condiciones de frontera de bolsa para tal interacción es que sólo estados singletes de color (que tienen trialidad cero) pueden existir. La constante de acoplamiento no necesita ser grande para lograr esto. Incluso cuando los campos de quarks son libres dentro de la bolsa, las ecuaciones de campo más las condiciones de frontera no son resolubles exactamente en tres dimensiones espaciales. En vez de eso las resolvemos en lo que parece ser una aproximación razonable a orden cero que es análoga a la ``teoría de Bohr'' para el átomo de hidrógeno: Las ecuaciones clásicas de movimiento admiten una clase de soluciones en que la superficie de la bolsa (en su marco de referencia) es una esfera de radio fijo. Las condiciones de frontera requieren que cada quark ocupe un modo con momento angular total $\frac{1}{2}$. Tratamos estos modos en una cavidad esférica fija como análoga a las órbitas circulares con radio fijo en la vieja teoría cuántica. El radio es entonces cuantizado por la condición que el operador número quark toma valores enteros. Para estos estados, la energía depende en qué modos están ocupados pero no en la forma del momento angular o isoespines de los quarks individuales son agregados para obtener el momento angular total e isoespín del hadrón. Así, por ejemplo, el estado de más baja energía $N(\frac{1}{2} +)$ y $\Delta(\frac{3}{2} +)$ son degenerados. Ya que $B$ es el único parámetro libre somos capaces de hacer predicciones para cantidades dimensionales tales como el momento magnético del protón y radio de carga y los splittings de masa de orden cero en el espectro bariónico. 

\subsection{Cálculos}

Las ecuaciones de movimiento y condiciones de frontera para un campo confinado de espín $\frac{1}{2}$ y sin masa

\begin{equation}
\slashed{\partial} {\psi}_{\alpha}(x) = 0
\end{equation}
 
dentro de la bolsa y

\begin{eqnarray}
i \slashed{n} {\psi}_{\alpha} (x) = {\psi}_{\alpha} (x), \\
\sum_{\alpha} n \cdot {\partial} \bar{\psi}_{\alpha} (x) {\psi}_{\alpha} (x) = 2B
\end{eqnarray}

sobre la superficie de la bolsa. ${n}_{\mu}$ es la 4-normal interior covariante a la superficie de la bolsa. $\alpha$ es un índice de simetría interna que escogemos para designar isoespín y color. Buscamos soluciones para que la frontera sea una esfera estática de radio ${R}_{0}$ en cuyo caso ${n}_{\mu} = (0, - \hat{r})$ y ecuaciones (2) se vuelven

\begin{eqnarray}
-i \hat{r} \cdot \va{\gamma}{\psi}_{\alpha} (x) = {\psi}_{\alpha}(x), \\
-\sum_{\alpha} \frac{\partial}{\partial r} \bar{\psi}_{\alpha}(x) {\psi}_{\alpha} (x) = 2B
\end{eqnarray}

en $r= {R}_{0}$.

La solución general a las ecuaciones (1) y (3) es una superposición (con coeficientes ${a}_{\alpha}$) de soluciones a la ecuación de Dirac libre:

\begin{equation}
{\psi}_{\alpha}(x,t) = \sum_{n \kappa j m} N ({\omega}_{n \kappa j}) {a}_{\alpha} (n \kappa j m) {\psi}_{n \kappa k m} (x, t).
\end{equation}

$j$ y $m$ etiquetan el modo del momento angular y su zomponente $z$. $\kappa$ es el número cuántico de Dirac\footnote{•}, $\kappa = \pm (j + \frac{1}{2})$, que diferencia los dos estados de paridad opuesta para cada valor de $j$.  El índice $n$ etiqueta frecuencias que están a ser determinadas por las condiciones de frontera lineales. La  condición de fontera cuadrática (3b) restringe los modos que pueden ser excitados. Entre otras cosas, 3b permite sólo soluciones para la ecuación de Dirac.
Para $j = \frac{1}{2}$, ya sea $\kappa = - 1$,

\begin{equation}
{\psi}_{n \, -1 \, \frac{1}{2} \, m} (x,t) = \frac{1}{\sqrt{4 \pi}} 
\left( 
\begin{array}{c}
i {j}_{0} ({\omega}_{n, \, -1} r / {R}_{0}) {U}_{m} \\
- {j}_{1} ({\omega}_{n, \, -1} r / {R}_{0}) \sigma \cdot \hat{r}{U}_{m} 
\end{array}
\right) \times {e}^{- i {\omega}_{n, \, -1} t / {R}_{0}}
\end{equation}

o ${\kappa} = 1$

\begin{equation}
{\psi}_{n \, 1 \, \frac{1}{2} \, m} (x,t) = \frac{1}{\sqrt{4 \pi}} 
\left( 
\begin{array}{c}
i{j}_{1} ({\omega}_{n, \, 1} r / {R}_{0}) \sigma \cdot \hat{r} {U}_{m} \\
{j}_{0} ({\omega}_{n, \, 1} r / {R}_{0}) {U}_{m} 
\end{array}
\right) \times {e}^{- i {\omega}_{n, \, 1} t / {R}_{0}}
\end{equation}

${U}_{m}$ es un espinos de Pauli bidimensional y ${j}_{\ell}(z)$ son las funciones de Bessel esféricas. Hemos omitido los índices $j$ sobre ${\omega}_{n \kappa}$ ya que solo $j = \frac{1}{2}$ es de interés en el presente. $N({\omega}_{n \kappa})$ es una constante de normalización escogida para conveniencia futura:

\begin{equation}
N({\omega}_{n \kappa}) \equiv \left( \frac{{\omega}_{n \kappa}^{\phantom{n \kappa} 3}}{2 {R}_{0}^{\phantom{0} 3} ({\omega}_{n \kappa} + \kappa) \sin^{2} {\omega}_{n \kappa}} \right)^{1/2}
\end{equation}

La condición de frontera lineal (3a) genera una condición eigenvalor para los modos de frecuencias ${\omega}_{n \kappa}$

$$
{j}_{0}({\omega}_{n \kappa}) = - \kappa {j}_{1} ({\omega}_{n \kappa}),
$$

o 

\begin{equation}\label{condeigenval}
\tan {\omega}_{n \kappa} = \frac{{\omega}_{n \kappa}}{{\omega}_{n \kappa} + \kappa}
\end{equation}

[Por convención escogemos $n$ positiva (negativa) secuencialmente para etiquetar las raíces positivas (negativas) de la eq 7] Las primeras soluciones a \eqref{condeigenval} son 

\begin{equation}
\begin{array}{ccc}
\kappa = - 1: & {\omega}_{1 \, -1} = 2.04; & {\omega}_{2 \, -1} = 5.40 \\
\kappa = + 1: & {\omega}_{1 \, 1} = 3.81; & {\omega}_{2 \, 1} = 7.00.
\end{array}
\end{equation}

La condición de frontera cuadrática requiere que $\sum_{\alpha} (\partial / \partial r) \bar{\psi}_{\alpha} (x) {\psi}_{\alpha}(x)$ sea independiente de tiempo y dirección para $r={R}_{0}$. La independencia angular requiere que $j = \frac{1}{2}$. Para obtener independencia temporal, ajustamos

\begin{equation}
\sum_{\alpha} {a}_{\alpha}^{*} (n \, \kappa \, j= \frac{1}{2} \, m) {a}_{\alpha} (n' \, \kappa' \, j= \frac{1}{2} \, m') = 0,
\end{equation}

a menos que $n = n'$, $\kappa = \kappa'$ o $n = -n'$, $\kappa = -\kappa'$ en cuyos casos no hay restricción ya que los términos dependientes del tiempo se cancela. La ecuación anterior es una restricción severa sobre los modos que deben ser ocupados. Deberíamos implementar la ecuación anterior requiriendo que para cada grado de libertad interno $\alpha$ sólo un modo normal, ${a}_{\alpha}(n \, \kappa \, j = \frac{1}{2} \, m)$ es excitado. Esto automáticamente será el caso para bariones de tres quarks si son requeridos a ser singletes de color.

Una vez que (9) es satisfecho, los términos independientes del tiempo en (3b) pueden ser coleccionados,

\begin{equation}
\sum_{\alpha \, n \, \kappa \, m} {\omega}_{n \kappa} {a}_{\alpha}^{*}(n \, \kappa \, \frac{1}{2} \, m) {a}_{\alpha}(n \, \kappa \, \frac{1}{2} \, m) = 4 \pi B {R}_{0}^{4},
\end{equation}


\subsubsection*{Conclusiones}

\begin{enumerate}

\item El campo en la bolsa se comporta sobre el promedio como un gas relativista perfecto; que es, la traza del tensor energía momento asociado con el campo, cuando es promediado sobre tiempo y espacio, es cero:
\begin{equation}
\left\langle \int_{R} {\mathrm{d}}^{3} x ({\Theta}_{\mu}^{\mu})_{\mathrm{campo}} \right\rangle = 0
\end{equation}
\item El volumen promediado en el tiempo de una bolsa es proporcional a su energía:
\begin{equation}
E = 4B \langle V \rangle
\end{equation}
\item El estado base y estados excitados más bajos de la bolsa contienen pocos partones de momento promedio de orden ${B}^{1/4}$ encerrados en un volumen de orden ${B}^{-3/4}$. [$B$ tiene la dimensión $(\mathrm{longitud})^{-4}$ con $\hbar=c=1$]
\item En el límite termodinámico la bolsa tiene una temperatura fija, ${T}_{0}$, independiente de su energía. ${T}_{0}$ es de orden ${B}^{-1/4}$. Esto es equivalente a las siguientes declaraciones
\begin{itemize}
\item La energía cinética promedio de los partones es de orden ${T}_{0}$ independiente de la energía de bolsa $E$ proporcionado el último es más grande que ${T}_{0}$: ${E} \gg {T}_{0}$.
\item La densidad de nivel asintótico ${\zeta(E)}$ del sistema es una función exponencial de $E$:
\[
\zeta \sim {e}^{E/{T}_{0}}
\]
\item El número, $N$, de partones más antipartones presente en el hadrón es proporcional a su energía:
\[
N \propto E/{T}_{0}
\]
\end{itemize}
\item Si la dinámica clásica es tal que hay un máximo momento angular del hadrón en una energía total dada $E$, ese máximo debe ser 
\[
{J}_{\mathrm{m\acute{a}x}} = \kappa {B}^{-1/3} {E}^{4 / 3},
\]
donde ${\kappa}$ es una constante adimensional determinada por la dinámica detallada. Si el límite clásico $({\hbar} \rightarrow 0)$ existe, las correcciones cuánticas a esta fórmula se reducirían por potencias de $E$. Si no hay trayectoria clásica a seguir, un argumento plausible sugiere que la trayectoría guía podría ser (para un gran $E$)
\[
{J}_{\mathrm{m\acute{a}x}} = {\kappa}' {B}^{-1/2} {E}^{2} \quad ({\hbar = 1}).
\]
\item El momento angular más probable para una $E$ grande está dada por 
\[
\bar{J} \propto ({B}^{-1/4} E)^{5/6}
\]
\end{enumerate}





































%\chapter{Notas sobre New Extended Model Of Hadrons de A. Chodos}

\section{Campos escalares}

Aquí empiezan con el estudio cuantitativo de las propiedades de teorías de campos confinados a una bolsa con el caso de un campo escalar único.

\subsection{Formulación del problema clásico}

Empezamos con el Lagrangiano

\begin{equation}
\begin{array}{rl}
L&= \int_{R} {\mathrm{d}}^{n-1} x (- \frac{1}{2} {\partial}_{\mu} {\phi} {\partial}^{\mu} {\phi} - B) \\
& \equiv \int_{R} {\mathrm{d}}^{n - 1} x \mathscr{L}
\end{array}
\end{equation}

(nuestra métrica es $- {g}^{00} = {g}^{ii} = 1$), donde $B$ es la constante de bolsa, que es, la densidad de energía asociada con el volumen $R$ al que los campos están confinados. La frontera de la región $R$ barre una superficie $S$ en espacio-tiempo. Las coordenadas ${X}^{\mu}$ de $S$ etiquetadas por $n-1$ parámetros ${\alpha}_{j}$,

\begin{equation}
{X}^{\mu} = {X}^{\mu}(\{ \alpha \})
\end{equation}

El vector unitario normal (${n}_{\mu}$) a esta superficie es definida para ser el vector unitario ortogonal a los $n-1$ vectores tangentes ${T}_{j}^{\mu}$:

\begin{equation}
{T}_{j}^{\mu} \equiv \frac{d}{d{\alpha}_{j}} {X}^{\mu} (\{ \alpha\}).
\end{equation}

Es útil expresar ${n}_{\mu}$ en términos de la normal (${m}_{\mu}$) a la superficie a tiempo constante ($t \equiv {x}^{0}$). Para hacer esto escogemos el parámetro ${\alpha}_{0} = t $ y reescribimos las ecuaciones anteriores como

\begin{equation}
\begin{array}{rl}
{X}^{\mu}  &= (t, X(\{ \alpha \})), \quad i = 1, \dots, n-1 \\
{T}_{j}^{\mu} & =\left\{
\begin{array}{c}
(1,\dot{X}^{i}(\{\alpha, t\})), \quad j=0\\
\left(0, \dfrac{d}{d{\alpha}_{j}} {X}^{i} (\{ \alpha \}, t)\right), \quad j =1,\dots,n-2
\end{array}
\right.
\end{array}
\end{equation}

${m}_{\mu}$ es entonces vector unitario puramente espacial [${m}_{\mu} = (0, {m}_{i})$] ortogonal a los $n - 2$ vectores tangentes ${T}_{j}^{\mu}$ ($j = 1, \dots, n - 2$):

\[
{m}_{\mu} {T}_{j}^{\mu} = 0, \quad j =1, \dots, n - 2
\]

\[
{m}_{\mu} {m}^{\mu} = 1
\]

Entonces definir

\begin{equation}
{n}_{\mu} = \frac{-({m}_{\lambda} \dot{X}^{\lambda}) {\eta}_{\mu} + {m}_{\mu}}{[1 - ({m}_{\lambda} \dot{X}^{\lambda})^{2}]^{1/2}},
\end{equation}

donde ${\eta}_{\mu}$ es el vector tipo tiempo unitario:

\[
{\eta}_{\mu} \equiv (1, 0, \dots, 0)
\]

y $\dot{X}^{\lambda} \equiv {T}_{0}^{\lambda}$. Es fácil verificar que ${n}_{\mu} {T}_{j}^{\mu} = 0$ y ${n}_{\mu}{n}^{\mu}$. Para establecer una convención escogemos que ${m}_{\mu}$ sea normal interior a la superficie espacial.

Con esta geometría preliminar en mente derivamos las ecuaciones de movimiento del sistema requiriendo que la acción $W \equiv \int_{{t}_{0}}^{{t}_{1}} dt \, L$ sea estacionario bajo variaciones del campo $\phi$ y de la frontera $S$ que se desvanece en ${t}_{0}$ y ${t}_{1}$. Estabilidad bajo variación en la frontera requiere que la densidad de Lagrange se desvanezca sobre $S$:

\begin{equation}
{\partial}_{\mu} {\phi} {\partial}^{\mu} {\phi} = - 2B \quad \mathrm{sobre} \quad S
\end{equation}

La variación de los campos generan la ecuación de Klein-Gordon dentro de la bolsa:

\begin{equation}
{\partial}_{\mu} {\partial}^{\mu} {\phi} = 0 \quad \mathrm{en} \, R
\end{equation}

y otra condición de frontera:

\begin{equation}
{n}_{\mu} {\partial}^{\mu} {\phi} = 0 \quad \mathrm{sobre} \, S
\end{equation}

Esta condición de frontera surge a partir de términos superficiales en las integraciones parciales que son realizadas para liberar la variación $\delta \phi$ de la derivada ${\partial}_{\mu}$

\subsection{Invariancia de Poincaré del problema clásico}

Las ecuaciones de movimiento son manifestamente invariante de Poincaré. Correspondiendo a esta invariancia tenemos un arreglo de momentos ${P}_{\mu}$ y generadores de rotación de Lorentz ${M}_{\mu \nu}$ que deberían ser independientes del tiempo. Estos pueden ser construidos por medio del teorema de Noether a partir del Lagrangiano. 

Las corrientes conservadas localmente son idénticas a esas del campo de Klein Gordon libre excepto por términos involucrando la densidad de energía $B$:

\begin{equation}
{T}_{\mu \nu} \equiv {g}_{\mu \nu} \mathscr{L} + {\partial}_{\mu} {\phi} {\partial}_{\nu} {\phi},
\end{equation}

\begin{equation}
{M}_{\mu \nu \lambda} = {x}_{\mu} {T}_{\nu \lambda} - {x}_{\nu} {T}_{\mu \lambda}
\end{equation}

con

\[
{\partial}^{\nu} {T}_{\mu \nu} = {\partial}^{\lambda} {M}_{\mu \nu \lambda} = 0
\]

Para mostrar la constancia de las cargas correspondientes considerar la integral de la divergencia de una corriente conservada sobre el "hipertubo mundial" de la bolsa:

\begin{equation}
0 = \int_{V} {d}^{n} x \, {\partial}_{\mu} \mathscr{J}^{\mu} \quad (\mathrm{donde} {\partial}_{\mu} \mathscr{J}^{\mu} = 0)
\end{equation}

$V$ es el volumen espacio temporal recorrido por la bolsa y está ligado por dos hipersuperficies de tipo espacial y luz ${R}_{1}$ y ${R}_{2}$ que pueden ser tomadas como superficies de tiempo constante.

Integrando la ecuación anterior obtenemos

\begin{equation}
Q \equiv \int_{{R}_{1}} ds \, {n}_{\mu} \mathscr{J}^{\mu} = \int_{{R}_{2}} ds \, {n}_{\mu} \mathscr{J}^{\mu} - \int_{S} ds \, {n}_{\mu} \mathscr{J}^{\mu}
\end{equation}

donde $ds$ es el elemento de superficie sobre las superficies $(n-1)$ dimensionales ${R}_{1}$, ${R}_{2}$, y $S$. Para las corrientes conservadas de (3.9) y (3.10) es fácilmente mostrado que ${n}_{\mu} \mathscr{J}^{\mu } = 0$ sobre $S$ con el auxilio de la condición de frontera (3.8). Por lo tanto hay independencia temporal de las cargas convencionales. Para completez, anotamos las expresiones para ${P}_{\mu}$ y ${M}_{\mu \nu}$ definidas sobre superficies de tiempo constante:

\begin{equation}
{P}_{\mu} \equiv \int_{R} {d}^{n-1} x {T}_{\mu}^{\phantom{\mu} 0}
\end{equation}

\begin{equation}
{M}_{\mu \nu} \equiv \int_{R} {d}^{n-1} x \, ({x}_{\mu} {T}_{\nu}^{0} - {x}_{\nu} {T}_{\mu}^{0}).
\end{equation}

Ya que la función primaria de las condiciones de frontera es garantizar la conservación de los generadores de Poincaré, podemos preguntar si existe un conjunto alternativo de condiciones de frontera, aparte de (3.6) y 3.9, que lograran esta meta.


\subsection{Mecánica clásica en dos dimensiones}

En una dimensión espacial y una temporal, las ecuaciones de movimiento del campo escalar confinado a una bolsa se simplifican considerablemente.

Ya que el campo dentro de la bolsa es sin masa, es conveniente trabajar con variables de cono de luz:

\[
{x}^{+} \equiv \tau \equiv \frac{1}{\sqrt{2}}(t + z),
\]

\[
{x}^{-} \equiv x \equiv \frac{1}{\sqrt{2}} (t -z)
\]

Usando variables de cono de luz, el tensor métrico es fuera de la diagonal ${g}^{+-} = {g}^{-+} = -1$, ${g}^{++} ={g}^{--} = 0$. Denotamos las derivadas con respecto a $\tau$ por puntos:

\[
{\partial}_{+} {\phi} (x, \tau) = \frac{\partial}{\partial \tau} {\phi} (x, \tau) = \dot{\phi} (x,\tau)
\]

y derivadas con respecto a $x$ por primas:

\[
{\partial}_{-} {\phi} (x, \tau) = \frac{\partial}{\partial x} \phi (x, \tau) = {\phi}' (x, \tau).
\]

\subsubsection{Solución al problema clásico}

En dos dimensiones y en coordenadas de cono de luz, la ecuación de movimiento y las condiciones de frontera se reducen a

\begin{equation}
\frac{{\partial}^{2}}{\partial x \partial \tau} {\phi} (x, \tau) = 0, \; \mathrm{en} \; R
\end{equation}

\begin{equation}
\dot{\phi} ({x}_{i} (\tau), \tau) {\phi}' ({x}_{i}(\tau), \tau) = - B, \quad i =0,1
\end{equation}

\begin{equation}
{\phi}({x}_{i}(\tau), \tau) = 0,
\end{equation}

donde ${x}_{i} (\tau)$ ($i = 0, 1$) son los dos puntos que encierran la bolsa. 

Para resolver la ecuación de onda, podemos proponer una solución de la forma

\begin{equation}
{\phi}(x, \tau) = {f}(\tau) + g(x).
\end{equation}

Las condiciones de frontera pueden ser reescritas en términos de ${f}(\tau)$ y ${g}(x)$:

\begin{equation}
\dot{f} (\tau) {g}' ({x}_{i}(\tau)) = - B, \quad i=0, 1
\end{equation}

\begin{equation}
\dot{f}(\tau) + \dot{x}_{i} (\tau) {g}' ({x}_{i} (\tau)) = 0,
\end{equation}

donde hemos diferenciado (3.15c) para obtener (3.17b). Las constantes del movimiento están dadas por (3.13):

\begin{equation}
{P}^{-} \equiv H = B ({x}_{1} (\tau) - {x}_{0} (\tau)),
\end{equation}

\begin{equation}
{P}^{+} \equiv P = \int_{{x}_{0}(\tau)}^{{x}_{1}(\tau)} dx \, [g'(x)]^{2},
\end{equation}

\begin{equation}
{M}^{+-} \equiv M = H \tau - \int_{{x}_{0}(\tau)}^{{x}_{1}(\tau)} dx \, x [{g}' (x)]^{2}
\end{equation}

La independencia temporal de $H$, $P$, y $M$ puede ser verificada con la ayuda de las condiciones de frontera (3.17). Por ejemplo, una combinación ajustable de las ecuaciones (3.17) lleva a

\[
\dot{x}_{i}(\tau) = \frac{[\dot{f}(\tau)]^{2}}{B},
\]

tal que $\dot{x}_{i}(\tau)$ es independiente de $i$ y $\dot{H} = 0$.

Para seguir encontraremos conveniente linealizar las condiciones de frontera. Esto puede ser hecho definiendo un nuevo parámetro espacial ${\sigma} = {\sigma}(x)$ de acuerdo a la ecuación diferencial

\begin{equation}
\frac{d \sigma}{d x} = \frac{1}{p} [g'(x)]^{2}
\end{equation}

y condición inicial $\sigma ( {x}_{0}(0)) = 0$, donde $p$ es una constante que será especificada después.
Definimos un nuevo campo $\tilde{g} (\sigma)$ en términos de $g(x)$ por este cambio de variables independientes,

\begin{equation}
\tilde{g} (\sigma) \equiv {g} (x(\sigma))
\end{equation}

$x(\sigma)$ será determinado a partir del inverso de

\begin{equation}
\frac{dx}{d\sigma} = \frac{1}{p} [\tilde{g}'(\sigma)],
\end{equation}

donde $\tilde{g}' (\sigma) \equiv (\frac{d}{d\sigma}) \tilde{g} (\sigma)$. Las fronteras de la bolsa son ${\sigma}_{i} (\tau) \equiv \sigma ({x}_{i} (\tau))$. Cuando es descrito en términos de $\sigma$, el movimiento de frontera será bastante más simple. Cuando transformamos a $\sigma$ como variable independiente (3.17) se vuelve

\begin{equation}
\dot{f}(\tau) = -\frac{B}{p} \tilde{g}'({\sigma}_{i}(\tau)),
\end{equation}

\begin{equation}
\dot{f}(\tau) + \dot{\sigma}_{i} (\tau) \tilde{g}' ({\sigma}_{i} (\tau)) = 0
\end{equation}

tal que $\dot{\sigma}_{i} (\tau) = \frac{B}{p}$. Usando la condición inicial ${\sigma} ({x}_{0}(0)) = {\sigma}_{0} = 0$, ${\sigma}_{0}(\tau) = \frac{B \tau}{p}$, ${\sigma}_{1}(\tau) = (B \tau / p) + {\sigma}_{1}$, donde ${\sigma}_{1}$ es una constante de integración. Para especificar ${\sigma}_{1}$ y $p$ considerar el momento (3.18b) en conjunción con (3.19):

\[
P = p ({\sigma}_{1} (\tau) - {\sigma}_{0} (\tau)) = p {\sigma}_{1}
\]

Consecuentemente, si escogemos por conveniencia $p$ sea la constante $P$, entonces ${\sigma}_{1} = 1$ y

\begin{equation}
{\sigma}_{1} (\tau) = \frac{B \tau}{P} + 1
\end{equation}

\begin{equation}
{\sigma}_{0} (\tau) = \frac{B \tau}{P}
\end{equation}

La solución es ahora inmediato 


\section{Campos fermiónicos} 

\subsection{Declaración de las condiciones de frontera}

Supongamos que consideramos un solo campo de Dirac en la bolsa descrito por la acción

\begin{equation}
{W}_{1} = \int_{V} {d}^{4} x [\frac{1}{2} i (\bar{\psi} \overleftrightarrow{\slashed{\partial}} \psi) - ]
\end{equation}




\chapter{Notas sobre Masses and other parameters of the light hadrons DeGrand }

Las masas y parámetros estáticos de hadrones ligeros 

Los efectos  de la energía cinética de quark, energía de bolsa, masa de quark extraño, intercambio de gluon colorado a más bajo orden, y energía asociada con ciertas fluctuaciones cuánticas son incluidas. Estas son parametrizadas por cuatro constantes que tienen significancia fundamental y no cambiar de multiplete a multiplete. El ajuste al espectro es bueno.

Momentos magnético, constantes de decaimiento débil, y el radio de carga son calculados. Donde comparación con experimento es posible.

\section{Introducción}

Durante la decada pasada, una teoría de quarks de estructura de hadron ha sido desarrollada que es exitosa en interpretar vastas cantidades de datos experimentales de unas pocas ideas simples extraordinariamente. Los ingredientes de esta teoría son como sigue:

\begin{enumerate}
\item Hadrones están compuestos de quarks. Los quarks vienen en varios "sabores", los tres de Gell - Mann y Zweig, aumentado quizá por nuevos quarks para nuevos grados de libertad hadrónicos tales como encanto, y 3 colores.
\item Los quarks interactuan entre ellos mismos relativamente débilmente por el intercambio de un octeto de gluones acoplados colorados, sin masa en la manera de Yang Mills a sus índices de colores. 
\item La interacción debe ser débil a cortas distancias para explicar escala en experimentos de dispersión de leptones; debe ser débil cerca de la transferencia de momento cero para tomar en cuenta para la falta de grandes renormalizaciones de modelo de quark sencillo estima de transiciones entre bariones ligeros.
\item La simetría SU(3) generada por la permutación de índices de color es inquebrantable.
\item Los quarks de diferentes sabores pueden tener diferentes masas para dar cuenta del desglose observado del SU(3) de Gell Mann y para las altas masas de estados compuestos de quarks encantados si eso es lo que son $J(3100)$ y $\psi(3700)$
\end{enumerate}

Finalmente, y esencialmente, quarks colorados y gluones colorados no son ellos mismos parte del espectro físico.

Los grados de libertad de quark-gluon pueden similarmente caracterizar variables colectivas describiendo la "baja excitación" de materia hadrónica. La única manera que sabemos de proporcionar una descripción de esto consistente con invariancia de lorentz es introduciendo un nuevo término, $-{g}_{\mu \nu} {\theta}_{s} B$, en el tensor de energía momento de la teoría \cite{DeTar_1983, Chodos_1974, Han_1965, Greiner2001, DeGrand_1975}.

\[
{\gamma}^{0} = \left(
\begin{array}{cc}
{I}_{2} & 0\\
0 & -{I}_{2}
\end{array}
\right)
\]




% ================================================================================================================================
% Apéndices
% ================================================================================================================================
\appendix
\renewcommand{\appendixname}{Apéndice}
\chapter{Appendix On Collisions}


\section{On collisions}

\begin{equation}
4
\end{equation}



\chapter{Appendix On Collisions}


\section{On collisions}

\begin{equation}
4
\end{equation}





\clearpage

% ================================================================================================================================
% Back Matter
% ================================================================================================================================
\backmatter

% Agregar un preámbulo antes de la lista de acrónimos

% Acrónimos
\printglossary[
    title=Acrónimos, 
    toctitle=Acrónimos, 
    type=\acronymtype
]

% List of terms
\printglossary[
    title=Glosario,
    toctitle=Glosario
]

\nomenclature[A, 04]{\(c\)}{\href{https://physics.nist.gov/cgi-bin/cuu/Value?c}{Speed of light in a vacuum}\nomunit{\SI{299792458}{\meter\per\second}}}

\nomenclature[A, 03]{\(h\)}{\href{https://physics.nist.gov/cgi-bin/cuu/Value?h}{Planck constant}\nomunit{\SI[group-digits=false]{6.62607015e-34}{\joule\per\hertz}}}

\nomenclature[A, 05]{\(G\)}{\href{https://physics.nist.gov/cgi-bin/cuu/Value?bg}{Gravitational constant} \nomunit{\SI[group-digits=false]{6.67430e-11}{\meter\cubed\per\kilogram\per\second\squared}}}

\nomenclature[A, 01]{${k}_{\mathsf{B}}$}{\href{https://physics.nist.gov/cgi-bin/cuu/Value?k}{Boltzmann constant} \nomunit{\SI[group-digits=false]{1.380649e-23}{\joule\per\kelvin}}}

\nomenclature[A, 02]{$\hbar$}{\href{https://physics.nist.gov/cgi-bin/cuu/Value?hbar}{Reduced Planck constant} \nomunit{\SI[group-digits=false]{1.054571817e-34}{\joule\second}}}

\nomenclature[B, 03]{\(\mathbb{R}\)}{Real numbers}
\nomenclature[B, 02]{\(\mathbb{C}\)}{Complex numbers}
\nomenclature[B, 01]{\(\mathbb{H}\)}{Quaternions}
\nomenclature[C]{\(V\)}{Constant volume}

\printnomenclature

% Bibliografía
\renewcommand\bibname{Bibliografía}
\bibliographystyle{unsrtnat}
\bibliography{Bibliography/BibFiles/proton_pressure-ref,
                Bibliography/BibFiles/Bag_model,
                Bibliography/BibFiles/QCD,
                Bibliography/BibFiles/Statistical_Physics,
                Bibliography/BibFiles/Phenomenological_Description,
                Bibliography/BibFiles/LQCD}

\end{document}

% run with
% pdflatex main.tex
% bibtex main.aux
% pdflatex main.tex
% pdflatex main.tex