\documentclass{beamer}

\usetheme{metropolis} % Usa Metropolis
\usepackage{graphicx}
\usepackage{booktabs}
\usepackage{amsmath}
\usepackage{siunitx}
\usepackage[sort&compress,numbers]{natbib} % Citas bibliográficas
\usepackage{hyperref}

% Logo en la esquina inferior derecha (ajusta el path si tienes un logo)
\setbeamertemplate{footline}{
  \leavevmode%
  \vspace{0.3cm}
  \hbox{%
  \hspace*{0.5cm}
  \includegraphics[height=0.5cm]{images_link/Cinvestav-logo.jpg}\hspace*{0.5cm}%
  \insertframenumber{} / \inserttotalframenumber\hspace*{0.5cm}
  }%
}

% Acrónimos (opcional)
% \usepackage[acronym]{glossaries}
% \makeglossaries
% \newacronym{qcd}{QCD}{Quantum Chromodynamics}

\graphicspath{{figures/}{images_link/}}

\title{Estudio del Protón mediante el Modelo de Bolsa y Estadística de Tsallis}
\subtitle{Defensa de Tesis Doctoral}
\author{Manuel Alejandro Matías Astorga}
\date{\today}
\institute{CINVESTAV IPN}

\begin{document}

\maketitle

\begin{frame}{Índice}
  \tableofcontents
\end{frame}

% Ejemplo de sección
\section{Introducción}
\begin{frame}{Motivación}
  \begin{itemize}
    \item ¿Qué es un protón?
    \item Motivación del estudio: entender estructura interna del protón.
  \end{itemize}
\end{frame}

\section{Marco teorico}

\begin{frame}{Bag Model}
    En la referencia principal como esta \cite{Barboza_Mendoza_2019} tenemos que desar
\end{frame}


\appendix
\renewcommand\bibname{Bibliografía}
\bibliographystyle{unsrtnat}
\bibliography{../Bibliography/Thesis_unified}


\end{document}