% \acrlong{ }
% Displays the phrase which the acronyms stands for. Put the label of the acronym inside the braces. In the example, \acrlong{gcd} prints Greatest Common Divisor.
% \acrshort{ }
% Prints the acronym whose label is passed as parameter. For instance, \acrshort{gcd} renders as GCD.
% \acrfull{ }
% Prints both, the acronym and its definition. In the example the output of \acrfull{lcm} is Least Common Multiple (LCM).

% Glosarios y acrónimos
\makeglossaries
\newacronym{bg}{BG}{Boltzmann - Gibbs}
\newacronym{be}{BE}{Bose - Einstein}
\newacronym{fd}{FD}{Fermi - Dirac}
\newacronym{qcd}{QCD}{Cromodinámica Cuántica}
\newacronym{qed}{QED}{Electrodinámica Cuántica}
\newacronym{mit}{MIT}{Massachusetts Institute of Technology}
\newacronym{slac}{SLAC}{Stanford Linear Accelerator Center}
\newacronym{gff}{GFF}{Gravitational form factors}
\newacronym{bm}{BM}{Bag Model}
\newacronym{lqcd}{LQCD}{Lattice QCD}
\newacronym{t-mitbm}{T-MIT bag model}{Tsallis-MIT Bag Model}
\newacronym{dvcs}{DVCS}{Deep Virtual Compton Scattering}
\newacronym{gpd}{GPD}{Generalized Parton Distributions}
\newacronym{cff}{CFFs}{Compton Form Factors}