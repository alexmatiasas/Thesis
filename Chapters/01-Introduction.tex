\chapter*{Introducción}
\addcontentsline{toc}{chapter}{Introducción}

% =====================Estilo de página================================
\pagestyle{fancy}
\fancyhf{} % limpiar cabecera
\fancyhead[LE]{\nouppercase{\textbf{Introducción} \hfill}}
\fancyhead[RO]{\nouppercase{\hfill \textbf{Introducción}}}
\fancyfoot[LE]{\nouppercase{\thepage \hfill {Pressure Distribution Inside Nucleons in a Tsallis-MIT Bag Model}}}
\fancyfoot[RO]{\nouppercase{{Pressure Distribution Inside Nucleons in a Tsallis-MIT Bag Model} \hfill \thepage}}
% ========================================================================

Entender la estructura interna de los hadrones es fundamental para descifrar las interacciones fuertes descritas por la \gls{qcd}. Este conocimiento tiene un impacto que trasciende la física de partículas, con aplicaciones en la astrofísica, la materia nuclear densa y las condiciones extremas del universo temprano. La \gls{qcd} describe a los hadrones como sistemas compuestos por quarks y gluones confinados, y una de sus propiedades emergentes más importantes es precisamente el confinamiento, responsable de que los quarks no se observen libres en la naturaleza.

% % \begin{wrapfigure}{r}{1\textwidth}
% % \centering
% % \includegraphics[width=1\textwidth]{./Images/LQCD.jpg}
% % \caption[Red LQCD]{\emph{Diagrama de una red tipo LQCD, donde los nodos representan quarks y las aristas simbolizan campos gluónicos.}}
% % \label{fig: LQCD}
% % \end{wrapfigure}

\begin{figure}[h]
    \centering
    \includegraphics[width=0.8\textwidth]{./Images/LQCD.jpg}
    \caption[Red LQCD]{\emph{Diagrama de una red tipo LQCD, donde los nodos representan quarks y las aristas simbolizan campos gluónicos.} Fuente: SolFinder Research (2020).}
    \label{fig:LQCD }
\end{figure}

Diversos enfoques teóricos se han desarrollado para estudiar este confinamiento. Entre ellos destacan los modelos de cuerdas \cite{Artru1974,Andersson_1983}, los modelos de valones \cite{Hwa_1981} y los modelos de bolsa \cite{AIHPA_1968__8_2_163_0,DeTar_1983}. Este último, particularmente el \gls{bm} del \gls{mit} \cite{Chodos_1974,Chodos1974a}, ha demostrado ser eficaz en capturar propiedades globales como la masa y el radio del protón. Este modelo representa a los hadrones como cavidades esferoidales en las que los quarks libres se mueven en el interior, confinados por una presión externa (presión de bolsa) que impide su escape.

Por otro lado, técnicas numéricas como la \gls{lqcd} permiten resolver las ecuaciones de \gls{qcd} en una red discreta. Aunque altamente demandantes computacionalmente, estas simulaciones han permitido estudiar correlaciones entre hadrones y estimar parámetros de interacción. Sin embargo, enfrentan el llamado \emph{problema del signo}, que limita su aplicabilidad en ciertos regímenes \cite{Iritani_2019,Hatsuda_2017}.

La colaboración \gls{alice}, en el \gls{lhc}, ha contribuido significativamente al estudio experimental de las interacciones hadrón-hadrón, complementando cálculos teóricos con datos de correlaciones entre bariones \cite{Collaboration2020,Collaboration2021}. Estos estudios, combinados con avances en \glspl{gpd} y \glspl{gff}, han permitido acceder a cantidades como la distribución radial de presión dentro del protón \cite{Burkert_2018}.

\begin{wrapfigure}{o}{0.45\textwidth}
    \centering
    \includegraphics[width=0.4\textwidth]{./Images/Bag model.png}
    \caption[Diagrama de bolsa]{\emph{Diagrama que ilustra el modelo de bolsa. En el interior, la presión es generada por un plasma de quarks y gluones, mientras que la presión externa mantiene confinados estos componentes dentro del hadrón.} Fuente: Adaptado de The MIT bag-model Glueball mass spectrum using the MIT bag-model (2015).}
    \label{fig:Bolsa }
\end{wrapfigure}

En este contexto, modelos fenomenológicos permiten aproximar el comportamiento de los hadrones sin resolver completamente la \gls{qcd}. En esta tesis se explora una extensión del \gls{bm} mediante el uso de la estadística de \gls{tsallis}, una generalización de la estadística de \gls{bg}, ampliamente utilizada en física de altas energías para describir distribuciones de momento transversal \cite{Tsallis1988,Beck_2003,Tsallis2009,Tsallis_2009,Marques_2015}.

Esta estadística introduce un parámetro $q$ que captura desviaciones del comportamiento extensivo, lo cual podría reflejar efectos no triviales en la interacción de quarks y gluones. Su aplicación en el contexto del \gls{bm} da lugar al modelo \gls{t-mitbm}, el cual permite estudiar la distribución interna de presión y energía del protón desde una perspectiva termodinámica efectiva.

El objetivo principal de este trabajo es analizar la distribución de presión dentro del protón mediante simulaciones computacionales basadas en el modelo \gls{t-mitbm}. Se comparan los resultados obtenidos con estimaciones experimentales basadas en \gls{dvcs} y \gls{gpd}, y se discuten posibles interpretaciones del parámetro $q$ en este contexto.

Además, como parte de las exploraciones preliminares, se consideró la posibilidad de extender el modelo \gls{t-mitbm} para el estudio de masas hadrónicas, utilizando una reformulación de la energía interna basada en la estadística de Tsallis. Aunque este análisis permanece en una etapa inicial, abre una interesante perspectiva para futuros desarrollos del modelo, integrando cálculos de masas directamente con un tratamiento no extensivo de las energías de bolsa.

La organización del documento es la siguiente:
\begin{enumerate}[i.]
    \item En el Capítulo~\ref{ch-BagModel}, se describe el \gls{bm} y su extensión en este trabajo.
    \item En el Capítulo~\ref{ch-Tsallis}, se presenta el formalismo de la estadística de \gls{tsallis} y sus aplicaciones relevantes.
    \item En el Capítulo~\ref{ch-ProtonBagParameters}, se analiza la estructura interna del protón y se discuten resultados relevantes.
    \item En el Capítulo~\ref{ch:TotalPandGluons}, se obtiene la presion de quarks y gluones en el interior del protón y su relacion con la presion de Tsallis.
    \item En el Capítulo~\ref{ch-PhysicalMeaningQ}, se propone una interpretación física del parámetro $q$.
    \item Finalmente, el Capítulo~\ref{ch-ResultsAndConclusions} resume las conclusiones y lineamientos para trabajos futuros.
    \item Al final de los capítulos se encuentra un apéndice~\ref{app:math_derivations} con todos los desarrollos matemáticos utilizados en el trabajo.
\end{enumerate}

% Entender la estructura interna de los hadrones es fundamental para descifrar las interacciones fuertes descritas por la \gls{qcd}. Este conocimiento no solo es crucial para la física de partículas, sino que también tiene implicaciones en la comprensión de la materia en condiciones extremas, como las que existieron en los primeros momentos del universo. Según el modelo de \gls{qcd}, las partículas hadrónicas están compuestas por quarks, los cuales permanecen confinados dentro de los hadrones. Este confinamiento es una de las características fundamentales de la \gls{qcd} y explica por qué no se han observado quarks libres en la naturaleza.

% % Renombrando las figuras
% \renewcommand{\figurename}{Fig.}

% % \begin{wrapfigure}{r}{1\textwidth}
% % \centering
% % \includegraphics[width=1\textwidth]{./Images/LQCD.jpg}
% % \caption[Red LQCD]{\emph{Diagrama de una red tipo LQCD, donde los nodos representan quarks y las aristas simbolizan campos gluónicos.}}
% % \label{fig: LQCD}
% % \end{wrapfigure}

% \begin{figure}[h]
%     \centering
%     \includegraphics[width=0.8\textwidth]{./Images/LQCD.jpg}
%     \caption[Red LQCD]{\emph{Diagrama de una red tipo LQCD, donde los nodos representan quarks y las aristas simbolizan campos gluónicos.} Fuente: SolFinder Research (2020).}
%     \label{fig:LQCD }
% \end{figure}

% A lo largo de los años, se han propuesto diversos modelos fenomenológicos para describir la estructura del protón. Entre ellos destacan los modelos de cuerdas \cite{Artru1974, Andersson_1983}, que representan hadrones como cuerdas oscilantes; los modelos de bolsa \cite{AIHPA_1968__8_2_163_0,DeTar_1983}, que describen quarks confinados en una cavidad; y los modelos de valones \cite{Hwa_1981}. Cada uno de estos enfoques ofrece una perspectiva única sobre la naturaleza de los hadrones, pero todos enfrentan limitaciones al tratar de capturar la complejidad de las interacciones no lineales entre quarks y gluones.

% Para explorar la física de la materia quark, se han desarrollado técnicas avanzadas que permiten estudiar estas interacciones. Una de ellas es la \gls{lqcd}, que utiliza simulaciones numéricas en redes espacio-temporales. Sin embargo, la complejidad de estos cálculos, que involucran millones de nodos, lleva a un problema conocido como \emph{el problema del signo}. Este problema surge en simulaciones Monte Carlo, donde los pesos de las configuraciones cuánticas pueden volverse negativos o incluso complejos, imposibilitando su interpretación como probabilidades clásicas %\cite{SignProblemReference}.

% Recientemente, la colaboración \gls{hal-qcd} ha utilizado técnicas de \gls{lqcd} para realizar cálculos de vanguardia en el estudio de interacciones fuertes entre hadrones \cite{Iritani_2019,Hatsuda_2017}. Sus resultados, que describen sistemas protón-neutrón y protón-hiperón, han sido comparados con datos experimentales publicados por la colaboración \gls{alice} \cite{Collaboration2020, Collaboration2021}. Estos avances han sentado las bases para el desarrollo de modelos fenomenológicos más simples, como el que proponemos en este trabajo.

% \begin{wrapfigure}{l}{0.45\textwidth}
%     \centering
%     \includegraphics[width=0.4\textwidth]{./Images/Bag model.png}
%     \caption[Diagrama de bolsa]{\emph{Diagrama que ilustra el modelo de bolsa. En el interior, la presión es generada por un plasma de quarks y gluones, mientras que la presión externa mantiene confinados estos componentes dentro del hadrón.} Fuente: Adaptado de Autor (Año).}
%     \label{fig:Bolsa }
% \end{wrapfigure}

% En este trabajo, proponemos un modelo basado en el modelo de bolsa del \gls{mit} \cite{Chodos_1974,Chodos1974a} y la estadística no extensiva de \Gls{tsallis}. A este modelo lo denominamos \gls{t-mitbm}. El modelo de bolsa del \gls{mit} describe hadrones como recipientes cerrados que contienen un mar de quarks y gluones, los cuales interactúan dentro de los límites del hadrón (ver Fig.~\ref{fig:Bolsa }). Sin embargo, este modelo tradicional no captura completamente las interacciones no lineales entre quarks y gluones. Aquí es donde la estadística de \Gls{tsallis}, una generalización de la estadística de \gls{bg} \cite{Tsallis1988,Beck_2003,Tsallis2009,Tsallis_2014,Tsallis_2009}, juega un papel crucial.

% La estadística de \Gls{tsallis} introduce un parámetro $q$ que captura las interacciones entre quarks y gluones, simplificando la no linealidad inherente a estas interacciones. Este enfoque ha demostrado ser exitoso en la descripción de sistemas complejos en física de altas energías, desde colisiones electrón-positrón \cite{Bediaga_2000,Collaboration1984} hasta colisiones de iones pesados \cite{Saraswat_2018,Saraswat_2017}. En nuestro modelo, combinamos el modelo de bolsa del \gls{mit} con la estadística de \Gls{tsallis} para estimar la distribución de presión total dentro de los nucleones.

% El objetivo principal de este trabajo es proponer un modelo fenomenológico que combine el modelo de bolsa del MIT con la estadística no extensiva de \Gls{tsallis} para estudiar la distribución de presión dentro de los nucleones. Comparamos nuestros resultados con la distribución de presión de quarks obtenida recientemente mediante técnicas de \gls{dvcs} \cite{Burkert_2018}. Estas técnicas, que involucran la dispersión de fotones virtuales de altas energías, han permitido medir la presión repulsiva de los quarks cerca del centro del protón y una presión de confinamiento a distancias mayores de $0.6$.

% En los siguientes capítulos, describiremos en detalle el marco teórico del \gls{t-mitbm}, los resultados obtenidos y las comparaciones con datos experimentales. En el Capítulo \ref{ch-BagModel}, explicamos el modelo de bolsa y sus limitaciones. En el Capítulo \ref{ch-Tsallis}, presentamos la estadística no extensiva de \Gls{tsallis} y su aplicación al plasma de quarks y gluones. En el Capítulo 5, comparamos nuestros resultados con los datos experimentales de presión de quarks, y en el Capítulo 6, discutimos una posible interpretación física del parámetro de \Gls{tsallis} $q$. Finalmente, en el Capítulo 7, presentamos las conclusiones y perspectivas futuras de este trabajo.

% \newpage

% \thispagestyle{empty}