\chapter{Derivaciones Matemáticas Detalladas}\label{app:math_derivations}

% =====================Estilo de página================================
\pagestyle{fancy}
\fancyhf{} % Limpia todos los campos de encabezado y pie de página
% \fancyhead[LE]{\nouppercase{\textit{\rightmark}}} % Sección en páginas pares (izquierda)
% \fancyhead[RO]{\nouppercase{\textit{\rightmark}}} % Sección en páginas impares (derecha)
\fancyhead[RE]{\nouppercase{\hfill \textbf{\leftmark}}} % Capítulo en páginas impares (izquierda)
\fancyhead[LO]{\nouppercase{\textbf{\leftmark} \hfill}} % Capítulo en páginas pares (derecha)
\fancyfoot[LE]{\nouppercase{\thepage \hfill {Distribución de la presión dentro de los nucleones en un modelo de bolsa de Tsallis-MIT}}} % Pie de página en páginas pares
\fancyfoot[RO]{\nouppercase{{Distribución de la presión dentro de los nucleones en un modelo de bolsa de Tsallis-MIT} \hfill \thepage}} % Pie de página en páginas impares
% =====================================================================

%%%%%%%%%%%%%%%%%%%%%%%%%%%%%%%%%%%%%%%%%%%%%%%%%%%%%%%%%%%%%%%%%%%%%%%%%%%%%%%%%%%%%%%%%%%%%%%%%%%%%%%%%%%%%

\section{Derivaciones detalladas del gas de gluones}
\label{app:BE_derivation}

En esta sección se derivan de forma detallada las expresiones termodinámicas del gas de gluones modelado como un sistema ideal de Bose-Einstein ultrarrelativista. Estas expresiones se utilizan en la sección \ref{sec-PresTsa} para el desarrollo de la entropía y presión dentro del hadrón bajo estadística de Tsallis.

% ------------------------------------------------------------------------------------------------ %

\subsection{Densidad de estados y función de partición}

Para gluones (bosones sin masa), la relación energía-momento en el régimen ultrarrelativista está dada por:

\begin{equation}
\epsilon = p c = \hbar c |\vec{k}|,
\end{equation}

donde \( |\vec{k}| \) es el módulo del vector de onda y \( p \) la magnitud del momento lineal. 

La densidad de estados en el espacio de fases tridimensional, considerando volumen \( V \), es:

\begin{equation}
g(\epsilon)\,d\epsilon = g_G \frac{V \epsilon^2 d\epsilon}{\pi^2 (hc)^3},
\end{equation}

donde \( g_G = 16 \) es el factor de degeneración para gluones (8 tipos de gluones × 2 polarizaciones transversales). Esta expresión surge de contar los modos disponibles en una caja cúbica de volumen \( V \) con condiciones de frontera periódicas, y se utiliza para convertir sumas discretas en integrales continuas.

La función de partición del sistema se expresa como:

\begin{equation}
\Xi = \prod_j \frac{1}{1 - e^{-\beta \epsilon_j}},
\end{equation}

donde \( j \) es un índice que recorre todos los niveles de energía disponibles y \( \beta = ({k}_{\mathrm{B}} T)^{-1} \).

% ------------------------------------------------------------------------------------------------ %

\subsection{Integrales fundamentales}

Las cantidades termodinámicas se expresan en términos de integrales de la forma:

\begin{equation}
I_n = \int_0^\infty \frac{\epsilon^n d\epsilon}{e^{\beta\epsilon} - 1},
\end{equation}

las cuales se evalúan fácilmente mediante el cambio de variable \( x = \beta \epsilon \), resultando en:

\begin{equation}
I_n = \frac{1}{\beta^{n+1}} \int_0^\infty \frac{x^n dx}{e^x - 1} = \frac{\Gamma(n+1)\zeta(n+1)}{({k}_{\mathrm{B}} T)^{n+1}},
\end{equation}

donde \( \Gamma \) es la función gamma y \( \zeta \) es la función zeta de Riemann.

Los valores específicos de estas integrales más relevantes son:

\begin{align}
\int_0^\infty \frac{x^2 dx}{e^x - 1} &= 2\zeta(3) \approx 2.404, \\
\int_0^\infty \frac{x^3 dx}{e^x - 1} &= \frac{\pi^4}{15} \approx 6.494.
\end{align}

% ------------------------------------------------------------------------------------------------ %

\subsection{Energía del sistema}
\label{app:BE-Energy}

La energía total del sistema se obtiene integrando sobre todos los modos disponibles, ponderados por su ocupación térmica:

\begin{equation}
E_G = \int_0^\infty g(\epsilon) \, \epsilon \, f_{\mathrm{BE}}(\epsilon)\, d\epsilon,
\end{equation}

donde \( f_{\mathrm{BE}}(\epsilon) = \frac{1}{e^{\beta\epsilon} - 1} \) es la distribución de ocupación de Bose-Einstein. Sustituyendo y usando el resultado anterior para la integral:

\begin{equation}
E_G = \frac{g_G V}{\pi^2 (hc)^3} \int_0^\infty \frac{\epsilon^3 d\epsilon}{e^{\beta\epsilon} - 1} = \frac{g_G V}{\pi^2 (hc)^3} \frac{\pi^4 ({k}_{\mathrm{B}} T)^4}{15}.
\end{equation}

En unidades naturales \( \hbar = c = {k}_{\mathrm{B}} = 1 \), se simplifica a:

\begin{equation}
E_G = g_G \frac{\pi^2}{30} V T^4.
\end{equation}

% ------------------------------------------------------------------------------------------------ %

\subsection{Presión}
\label{app:BE-Pressure}

La presión se obtiene a partir del potencial gran canónico, el cual se relaciona con la función de partición mediante:

\begin{equation}
\ln \Xi = -\frac{g_G V}{\pi^2 (hc)^3} \int_0^\infty \epsilon^2 \ln(1 - e^{-\beta\epsilon}) d\epsilon.
\end{equation}

Realizando una integración por partes y usando identidades estándar de integrales, se encuentra que:

\begin{equation}
\ln \Xi = \frac{g_G V}{3\pi^2 (hc)^3 \beta} \int_0^\infty \frac{\epsilon^3 d\epsilon}{e^{\beta\epsilon} - 1}.
\end{equation}

La presión se calcula como:

\begin{equation}
P_G = \frac{{k}_{\mathrm{B}} T}{V} \ln \Xi = \frac{g_G \pi^2 ({k}_{\mathrm{B}} T)^4}{90 (hc)^3},
\end{equation}

y en unidades naturales:

\begin{equation}
P_G = \frac{1}{3} \frac{E_G}{V} = g_G \frac{\pi^2}{90} T^4.
\end{equation}

Esto refleja la relación característica entre presión y energía de un gas ultrarrelativista.

% ------------------------------------------------------------------------------------------------ %

\subsection{Entropía}
\label{app:BE-Entropy}

La entropía se puede obtener directamente a partir de la energía utilizando la identidad:

\begin{equation}
S_G = \left( \frac{\partial E_G}{\partial T} \right)_{V} = \frac{4}{3} \frac{E_G}{T}.
\end{equation}

Sustituyendo la expresión de la energía:

\begin{equation}
S_G = g_G \frac{4\pi^2}{90} V T^3.
\end{equation}

% ------------------------------------------------------------------------------------------------ %

\subsection{Justificación de \(\mu = 0\) para gluones}

En sistemas con partículas sin número conservado, como los gluones, el potencial químico se anula en equilibrio termodinámico. Esto se debe a que los gluones pueden crearse y aniquilarse libremente, de modo que no existe una restricción asociada a su número total.

\begin{equation}
N_G(T,V,0) = \frac{g_G V}{\pi^2 (hc)^3} \int_0^\infty \frac{\epsilon^2 d\epsilon}{e^{\beta\epsilon} - 1} \propto T^3,
\end{equation}

lo que confirma que la ocupación de estados sigue siendo finita sin necesidad de introducir un término de control para \( N \). Por tanto, se toma:

\begin{equation}
\mu_G = 0.
\end{equation}

% ------------------------------------------------------------------------------------------------ %

\subsubsection*{Resumen de resultados fundamentales}

\begin{itemize}
    \item Energía total: \( E_G = g_G \frac{\pi^2}{30} V T^4 \)
    \item Presión: \( P_G = \frac{1}{3} \frac{E_G}{V} = g_G \frac{\pi^2}{90} T^4 \)
    \item Entropía: \( S_G = g_G \frac{4\pi^2}{90} V T^3 \)
    \item Potencial químico: \( \mu_G = 0 \)
\end{itemize}

Estas expresiones son esenciales para extender el análisis al marco de la estadística no extensiva de Tsallis, donde las correlaciones entre grados de libertad se introducen mediante el parámetro \( q \).


%%%%%%%%%%%%%%%%%%%%%%%%%%%%%%%%%%%%%%%%%%%%%%%%%%%%%%%%%%%%%%%%%%%%%%%%%%%%%%%%%%%%%%%%%%%%%%%%%%%%%%%%%%%%%

\section{Derivaciones detalladas del gas de quarks}
\label{app:FD_derivation}

El gas de quarks se modela como un sistema ideal de partículas fermiónicas ultrarrelativistas sin masa. Dado que se consideran quarks y antiquarks en equilibrio, el sistema incluye dos gases acoplados con potenciales químicos opuestos. Esta derivación permite obtener expresiones para la energía, presión, número neto de partículas y entropía.

% ------------------------------------------------------------------------------------------------ %

\subsection{Descripción del sistema}

Se considera un sistema compuesto por quarks y antiquarks con las siguientes características:

\begin{itemize}
    \item[$\triangleright$] Relación energía-momento: \( \epsilon = pc = \hbar c |\vec{k}| \), válida en el límite \( m \rightarrow 0 \).
    \item[$\triangleright$] Simetría partícula-antipartícula: \( \mu_{\text{quark}} = -\mu_{\text{antiquark}} = \mu \).
    \item[$\triangleright$] Factor de degeneración: \( g_Q = 12 \) (2 espines × 3 colores × 2 sabores \( u,d \)).
    \item[$\triangleright$] Estadística Fermi-Dirac:
    \[
    f_{\mathrm{FD}}^{\pm}(\epsilon) = \frac{1}{e^{\beta(\epsilon \mp \mu)} + 1}.
    \]
\end{itemize}

% ------------------------------------------------------------------------------------------------ %

\subsection{Número neto de partículas}

El número promedio total de quarks y antiquarks se obtiene sumando sobre los estados energéticos ocupados:

\begin{align}
N_+ &= \int_0^\infty g(\epsilon) f_{\mathrm{FD}}^+(\epsilon) d\epsilon, \\
N_- &= \int_0^\infty g(\epsilon) f_{\mathrm{FD}}^-(\epsilon) d\epsilon,
\end{align}

donde \( f_{\mathrm{FD}}^\pm \) son las distribuciones de quarks y antiquarks. La densidad de estados para partículas sin masa es la misma que para bosones:

\[
g(\epsilon) d\epsilon = \frac{g_Q V \epsilon^2 d\epsilon}{\pi^2 (hc)^3}.
\]

El exceso neto de quarks se define como:

\begin{equation}\label{eq-FD-Excedente-app}
N = N_+ - N_- = \frac{g_Q V}{\pi^2 (hc)^3} \int_0^\infty \epsilon^2 \left[ f_{\mathrm{FD}}^+(\epsilon) - f_{\mathrm{FD}}^-(\epsilon) \right] d\epsilon.
\end{equation}

Esta integral puede resolverse analíticamente en unidades naturales \( \hbar = c = \text{k}_{\mathrm{B}} = 1 \), resultando en:

\begin{equation}
N = \frac{g_Q V}{6} \left( \pi^2 \mu T^2 + \mu^3 \right).
\end{equation}

% ------------------------------------------------------------------------------------------------ %

\subsection{Energía total del sistema}

La energía total incluye la contribución de quarks y antiquarks:

\begin{equation}
E_Q = \int_0^\infty g(\epsilon) \, \epsilon \left[ f_{\mathrm{FD}}^+(\epsilon) + f_{\mathrm{FD}}^-(\epsilon) \right] d\epsilon.
\end{equation}

Evaluando la integral, se obtiene:

\begin{equation}
E_Q = g_Q V T^4 \left[ \frac{7\pi^2}{120} + \frac{1}{4} \left( \frac{\mu}{T} \right)^2 + \frac{1}{8\pi^2} \left( \frac{\mu}{T} \right)^4 \right].
\end{equation}

% ------------------------------------------------------------------------------------------------ %

\subsection{Presión}

En un gas ultrarrelativista sin masa, la relación fundamental entre energía y presión es:

\begin{equation}
P_Q = \frac{1}{3} \frac{E_Q}{V}.
\end{equation}

Sustituyendo la expresión para \( E_Q \):

\begin{equation}
P_Q = \frac{g_Q T^4}{3} \left[ \frac{7\pi^2}{120} + \frac{1}{4} \left( \frac{\mu}{T} \right)^2 + \frac{1}{8\pi^2} \left( \frac{\mu}{T} \right)^4 \right].
\end{equation}

% ------------------------------------------------------------------------------------------------ %

\subsection{Entropía}

La entropía se obtiene a partir de la relación:

\begin{equation}
S_Q = \frac{4}{3} \frac{E_Q}{T} - \frac{\mu N}{T}.
\end{equation}

Sustituyendo las expresiones de \( E_Q \) y \( N \), se llega a:

\begin{equation}
S_Q = g_Q V T^3 \left[ \frac{7\pi^2}{90} + \frac{1}{6} \left( \frac{\mu}{T} \right)^2 \right].
\end{equation}

% ------------------------------------------------------------------------------------------------ %

\subsubsection*{Resumen de resultados fundamentales}

\begin{itemize}
    \item \textbf{Número neto de quarks:} \( N = \dfrac{g_Q V}{6} \left( \pi^2 \mu T^2 + \mu^3 \right) \)
    \item \textbf{Energía:} \( E_Q = g_Q V T^4 \left[ \dfrac{7\pi^2}{120} + \dfrac{1}{4} \left( \dfrac{\mu}{T} \right)^2 + \dfrac{1}{8\pi^2} \left( \dfrac{\mu}{T} \right)^4 \right] \)
    \item \textbf{Presión:} \( P_Q = \dfrac{1}{3} \dfrac{E_Q}{V} \)
    \item \textbf{Entropía:} \( S_Q = g_Q V T^3 \left[ \dfrac{7\pi^2}{90} + \dfrac{1}{6} \left( \dfrac{\mu}{T} \right)^2 \right] \)
\end{itemize}

Estas expresiones son esenciales para el cálculo de la entropía total del sistema quark-gluón bajo el enfoque de Tsallis en la sección \ref{sec-PresTsa}, donde la no extensividad se introduce mediante una correlación entre las contribuciones individuales de quarks y gluones.

%%%%%%%%%%%%%%%%%%%%%%%%%%%%%%%%%%%%%%%%%%%%%%%%%%%%%%%%%%%%%%%%%%%%%%%%%%%%%%%%%%%%%%%%%%%%%%%%%%%%%%%%%%%%%

\section{Derivación de la presión en el modelo de Tsallis}
\label{app:Tsallis-pressure}

Partimos de la relación de Maxwell para sistemas termodinámicos generalizados:

\begin{equation}\label{eq:Maxwell-Tsallis-appendix}
\left. \frac{\partial S_q}{\partial V} \right|_{T,\mu} = \left. \frac{\partial P_q}{\partial T} \right|_{V,\mu},
\end{equation}

donde \( S_q \) representa la entropía total del sistema (quarks + gluones) bajo el formalismo de Tsallis. Asumimos que el volumen \( V \) y la temperatura \( T \) son cantidades extensivas, mientras que la entropía y la presión pueden presentar no extensividad a través del parámetro \( q \).

La expresión de entropía total está dada por (véase ecuación \ref{eq-Tsallis-Entropy-final}):

\begin{equation}
S_q = \left[ \frac{74\pi^2}{45} + 2\left( \frac{\mu}{T} \right)^2 \right] V T^3 + \frac{128\pi^2}{15}(1-q)\left[ \frac{7\pi^2}{90} + \frac{1}{6} \left( \frac{\mu}{T} \right)^2 \right] V^2 T^6.
\end{equation}

Al derivar con respecto a \( V \), obtenemos:

\begin{equation}\label{eq:dSdV}
\left. \frac{\partial S_q}{\partial V} \right|_{T,\mu} = \left[ \frac{74\pi^2}{45} + 2\left( \frac{\mu}{T} \right)^2 \right] T^3 + 2 \cdot \frac{128\pi^2}{15}(1-q)\left[ \frac{7\pi^2}{90} + \frac{1}{6} \left( \frac{\mu}{T} \right)^2 \right] V T^6.
\end{equation}

Aplicando la relación de Maxwell \eqref{eq:Maxwell-Tsallis-appendix}, esta derivada corresponde a la derivada de la presión con respecto a la temperatura:

\begin{equation}
\left. \frac{\partial P_q}{\partial T} \right|_{V,\mu} = \left. \frac{\partial S_q}{\partial V} \right|_{T,\mu}.
\end{equation}

Para obtener \( P_q \), integramos la expresión \eqref{eq:dSdV} con respecto a \( T \):

\begin{equation}
\begin{split}
P_q = & \left[ \frac{74\pi^2}{45} + 2\left( \frac{\mu}{T} \right)^2 \right] \int T^3 \, dT \\
& + 2 \cdot \frac{128\pi^2}{15}(1-q) \left[ \frac{7\pi^2}{90} + \frac{1}{6} \left( \frac{\mu}{T} \right)^2 \right] V \int T^6 \, dT + C(V, \mu, q).
\end{split}
\end{equation}

Al integrar término a término e imponer la condición de consistencia con el caso \( q = 1 \), se determina la constante de integración \( C(V, \mu, q) \) de modo que:

\[
P_{q=1} = P_Q + P_G.
\]

Usando las expresiones conocidas para gases de quarks (Fermi-Dirac) y gluones (Bose-Einstein) en el límite de Boltzmann-Gibbs:

\begin{equation}
P_{q=1} = \left[\frac{37\pi^2}{90} + \left(\frac{\mu}{T} \right)^2 + \frac{1}{2\pi^2} \left(\frac{\mu}{T} \right)^4 \right] T^4,
\end{equation}

lo que permite determinar explícitamente \( C(V, \mu, q) \). Finalmente, sustituimos y simplificamos para obtener la expresión completa:

\begin{equation}\label{eq:FinalTsallisPressureAppendix}
P_q = \left[ \frac{37\pi^2}{90} + \left( \frac{\mu}{T} \right)^2 + \frac{1}{2\pi^2} \left( \frac{\mu}{T} \right)^4 \right] T^4 + \frac{256\pi^2}{15}(1-q) \left[ \frac{\pi^2}{90} + \frac{1}{30} \left( \frac{\mu}{T} \right)^2 \right] V T^7.
\end{equation}

Esta expresión generaliza la presión del sistema a partir de la entropía de Tsallis e incluye explícitamente un término no extensivo que escala con \( V T^7 \). Este término se anula en el límite \( q \to 1 \), recuperando así la expresión clásica de Boltzmann-Gibbs.

\subsection*{Resumen esquemático: derivación de la energía en Tsallis} \label{app:flow}

\begin{figure}[H]
\centering
\begin{tikzpicture}[
    node distance=1.5cm,
    rect/.style={rectangle, draw, rounded corners=5pt, minimum width=3.5cm, minimum height=1cm, align=center, fill=blue!5},
    arrow/.style={->, >=stealth, thick}
]
% Nodos
\node (S) [rect] {Entropía total \\ $S_q = S_Q + S_G + (1-q)S_Q S_G$};
\node (P) [rect, below of=S] {Relación de Maxwell \\ $\displaystyle\frac{\partial S_q}{\partial V} = \frac{\partial P_q}{\partial T}$};
\node (E) [rect, below of=P] {Densidad de energía \\ $\epsilon_q = -P_q + T s_q + \mu \Delta n$};

% Flechas
\draw [arrow] (S) -- (P);
\draw [arrow] (P) -- (E);

% Anotación
\node [right=0.5cm of E, text width=4.5cm, anchor=west, font=\small] {
\begin{itemize}
\item[$\triangleright$] En este modelo, \( \Delta n = 0 \)
\item[$\Rightarrow$] \( \mu \Delta n = 0 \), ya que el número total de partículas no se ve afectado por las correlaciones.
\end{itemize}
};
\end{tikzpicture}
\caption{Flujo de derivación de la densidad de energía en el modelo de Tsallis.}
\label{fig:derivation-flow}
\end{figure}
\section{Modo fundamental y presión de bolsa en el modelo de bolsa}
\label{app:bag-pressure}

En el modelo de bolsa del MIT, las soluciones permitidas para las funciones de onda de los quarks están determinadas por una condición de contorno sobre la superficie de la bolsa esférica. Esta condición conduce a una ecuación de autovalores, cuyas soluciones discretas determinan los posibles modos normales.

\subsection{Condición de cuantización}

La ecuación de cuantización más baja para el modo esférico se escribe como:

\begin{equation}
\omega_{n\kappa} = p_{n\kappa} R,
\end{equation}

donde \( \omega_{n\kappa} \) es la energía adimensionalizada, \( p_{n\kappa} \) es el momento cuántico, y \( R \) es el radio de la bolsa. El modo fundamental (\( n = 1, \kappa = -1 \)) tiene el valor:

\begin{equation}
\omega_0 = \omega_{1, -1} \approx 2.04.
\end{equation}

Este valor define un momento máximo accesible para los quarks dentro de la bolsa:

\begin{equation}
p_{\text{max}} = \frac{\omega_0}{R}.
\end{equation}

\subsection{Energía de quarks confinados}

Bajo este límite, la energía cinética de los quarks (en ausencia de potencial químico y masa) se expresa como:

\begin{equation}
E_Q = \frac{(g_Q + g_{\bar{Q}}) V}{2\pi^2 \hbar^3} \int_0^{p_{\text{max}}} \frac{p^3 \, dp}{1 + e^{p/T(r)}},
\end{equation}

donde se ha asumido simetría quark-antiquark y \( g_Q = 12 \), como antes.

\subsection{Presión de bolsa}

La energía total del sistema está compuesta por la contribución de quarks y gluones. La presión de bolsa se interpreta como la presión neta que impide que los quarks escapen del volumen de confinamiento, y se define como:

\begin{equation}
B(r) = \frac{E_{\text{total}} - E_Q}{V},
\end{equation}

siendo \( E_{\text{total}} \) la energía de la cavidad completa a temperatura \( T(r) \).

Esta expresión es la base para determinar la función \( B(r) \) a partir de simulaciones numéricas, como se presenta en la Fig.~\ref{fig:Bpressure}.


\section{Reconstrucci\'on de la Presi\'on de Quarks \( P_Q(r) \)}

La distribuci\'on de presi\'on de quarks se obtiene a partir del t\'ermino \( D \) de los factores de forma gravitacionales (GFFs). Partimos de la parametrizaci\'on fenomenol\'ogica:

\begin{equation}
d_1(t) = d_1(0) \left( 1 - \frac{t}{M^2} \right)^{-\alpha}
\end{equation}

El t\'ermino \( d_1(t) \) se conecta a la distribuci\'on radial de presi\'on mediante una transformada de Bessel:

\begin{equation}
d_1(t) \propto \int \frac{j_0(r \sqrt{-t})}{2t} p(r) r^2 dr
\end{equation}

donde \( j_0 \) es la funci\'on de Bessel esf\'erica de primer tipo.

Al invertir esta relaci\'on, se obtiene:

\begin{equation}
p(r) = -\frac{1}{k_p \pi^2} \int_0^\infty x^4 j_0(r x) d_1(-x^2) dx
\end{equation}

Evaluando explícitamente la integral para el ansatz de \( d_1(t) \), se llega a una forma cerrada:

\begin{equation}
P_Q(r) = \frac{M^6 d_1(0)}{16\pi k_p r} e^{-M r} (M r - 3)
\end{equation}

donde los parámetros son:
\begin{itemize}
\item \( d_1(0) = -2.04 \),
\item \( M = 5\,\mathrm{fm}^{-1} \),
\item \( \alpha = 3 \),
\item \( k_p = 55 \).
\end{itemize}

Esta soluci\'on reproduce correctamente el comportamiento observado experimentalmente en~\cite{Burkert_2018}.

\section{Modelo de Presi\'on Total \( P_q(r) \) en el Tsallis-MIT Bag Model}

En el marco del modelo Tsallis-MIT, la presi\'on total se compone de dos contribuciones:

\begin{enumerate}
    \item Presi\'on de quarks: proporcional a \( T(r)^4 \),
    \item Correcci\'on glu\'onica Tsallis: proporcional a \( (1-q) T(r)^7 \).
\end{enumerate}

La expresi\'on completa es:

\begin{equation}
P_q(r) = \frac{37\pi^2}{90} T(r)^4 + \frac{256\pi^2}{15}(1-q) V T(r)^7
\end{equation}

donde:
\begin{itemize}
    \item \( T(r) = T_0 e^{-r^2/R_T^2} \) es el perfil de temperatura radial,
    \item \( V = \frac{4}{3} \pi R_{bag}^3 \) es el volumen fijo de la bolsa,
    \item \( q \) es el par\'ametro de no extensividad.
\end{itemize}

Este modelo permite interpolar entre el comportamiento extensivo de Boltzmann-Gibbs (\( q \to 1 \)) y correcciones no extensivas importantes para sistemas densos o con fuertes correlaciones.

\section{Unidades y Factores de Conversi\'on}

En todo el trabajo se han empleado \textbf{unidades naturales} donde \( \hbar = c = 1 \). En estas unidades:

\begin{equation}
1\,\mathrm{fm} = \frac{1}{197.3269804}\,\mathrm{MeV}^{-1}
\end{equation}

De esta manera:
\begin{itemize}
    \item \( \mathrm{MeV}^4 \) puede convertirse en \( \mathrm{MeV/fm}^3 \) mediante un factor de \( (1/\hbar c)^3 \).
    \item \( \mathrm{MeV}^7 \) necesita conversi\'on distinta debido al volumen adicional (el t\'ermino \( T^7 \) incluye ya \( V \)).
\end{itemize}

Los factores utilizados fueron:

\begin{align}
\mathrm{CONVERSION}_{T^4} &= \left( \frac{1}{\hbar c} \right)^3 \\
\mathrm{CONVERSION}_{T^7} &= \left( \frac{1}{\hbar c} \right)^6
\end{align}

Este tratamiento asegura que todas las presiones estén expresadas en unidades consistentes de \( \mathrm{MeV/fm^3} \).