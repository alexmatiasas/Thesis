\chapter{Presión de quarks y gluones en el marco de la estadística de Tsallis}\label{ch-Tsallis}

% \pagestyle{fancy}
% \fancyhead{} % clear all header fields
% \fancyhead[RO,LE]{\textbf{\chaptername\,\thechapter}  }
% \fancyfoot{} % clear all footer fields
% \fancyfoot[LE,RO]{\thepage}
% % \fancyfoot[LO,CE]{Introducción}
% \fancyfoot[CO,CE]{Estadística de Tsallis}

% =====================Estilo de página================================
\pagestyle{fancy}
\fancyhf{} % Limpia todos los campos de encabezado y pie de página
\fancyhead[LE]{\nouppercase{\textit{\rightmark}}} % Sección en páginas pares (izquierda)
\fancyhead[RO]{\nouppercase{\textit{\rightmark}}} % Sección en páginas impares (derecha)
\fancyhead[RE]{\nouppercase{\hfill \textbf{Capítulo 2. Presión de quarks y gluones en\\el marco de la estadística de Tsallis}}} % Capítulo en páginas impares (izquierda)
\fancyhead[LO]{\nouppercase{\textbf{Capítulo 2. Presión de quarks y gluones en\\el marco de la estadística de Tsallis} \hfill}} % Capítulo en páginas pares (derecha)
\fancyfoot[LE]{\nouppercase{\thepage \hfill {Pressure Distribution Inside Nucleons in a Tsallis-MIT Bag Model}}} % Pie de página en páginas pares
\fancyfoot[RO]{\nouppercase{{Pressure Distribution Inside Nucleons in a Tsallis-MIT Bag Model} \hfill \thepage}} % Pie de página en páginas impares
% =====================================================================

\section{Introducción}

La mecánica estadística estándar está basada en la entropía de \Gls{boltzman-gibbs} (\acrshort{bg}), que se define como

\begin{equation}
{S}_{\mathrm{BG}} = - {k}_{\mathrm{B}} \sum_{i=1}^{W} {p}_{i} \ln {p}_{i}, \quad \sum_{i=1}^{W} {p}_{i} = 1,
\end{equation}

donde \( W \) es el número de configuraciones microscópicas del sistema y \( {p}_{i} \) es la probabilidad de acceder a la \( i \)-ésima configuración. Sin embargo, sistemas no lineales dinámicos de muchos cuerpos, como los que se encuentran en física de partículas o en plasmas, a menudo no cumplen con las condiciones de \emph{ergodicidad} (que su valor esperado sea igual a su promedio a largo plazo) o \emph{extensividad} (que las propiedades del sistema sean proporcionales al número de partículas). Para estos sistemas, la estadística de \acrshort{bg} no es adecuada, lo que ha llevado al desarrollo de extensiones como la estadística de \Gls{tsallis}.

La estadística de Tsallis generaliza la entropía de \acrshort{bg} mediante la introducción de un parámetro no extensivo \( q \), definido por:

\begin{equation}
{S}_{q} = {k}_{\mathrm{B}} \frac{1-\sum_{i} {p}_{i}^{q}}{q-1} \qquad (q\in \mathbb{R};\; {S}_{q=1} = {S}_{\mathrm{BG}}),
\end{equation}

El parámetro \( q \) mide el grado de no extensividad del sistema. Para \( q = 1 \), se recupera la estadística de \acrshort{bg}, mientras que valores de \( q \neq 1 \) indican desviaciones de la extensividad, lo que es característico de sistemas con correlaciones de largo alcance, inhomogeneidades o memoria de largo plazo.% \cite{TsallisStatistics}.

Bajo esta generalización, en un sistema constituido por dos subsistemas probabilisticamente independientes $A$ y $B$ (i.e., si ${p}_{ij}^{A+B} = {p}_{i}^{A}{p}_{j}^{B}$), entonces

\begin{equation}\label{eq-TsallisTwoSystems}
\frac{{S}_{{q}}(A+B)}{{k}_{\mathrm{B}}} = \frac{{S}_{q}(A)}{{k}_{\mathrm{B}}} + \frac{{S}_{q}(B)}{{k}_{\mathrm{B}}} + (1 - q) \frac{{S}_{q}(A)}{{k}_{\mathrm{B}}}\frac{{S}_{q}(B)}{{k}_{\mathrm{B}}}\footnote{Más adelante se usarán unidades naturales por lo que no aparecerá la constante de Boltzmann, ${k}_{\mathrm{B}}$},
\end{equation}

de donde se observa claramente que cuando $q=1$ se regresa a la estadística de \acrshort{bg} %[ref Tsallis statistics].

La estadística de Tsallis ha demostrado ser particularmente útil en el estudio de sistemas complejos y fuera del equilibrio, como los que se encuentran en física de altas energías, plasmas o sistemas biológicos. %\cite{TsallisApplications}. 
En este trabajo, exploraremos cómo esta generalización de la entropía puede aplicarse al estudio de hadrones y al confinamiento de quarks, proporcionando una descripción más precisa de sistemas que no pueden ser tratados adecuadamente con la estadística de \acrshort{bg}.

En el capítulo anterior, se presentó el modelo de bolsa como un marco teórico para entender el confinamiento de quarks dentro de los hadrones. Sin embargo, este modelo tradicional tiene limitaciones en la descripción de sistemas fuera del equilibrio o en condiciones extremas. La estadística de Tsallis, con su parámetro no extensivo \( q \), ofrece una herramienta poderosa para abordar estas limitaciones y proporcionar una descripción más general de sistemas complejos. %\cite{BagModelLimitations}.

% Este capítulo se divide en varias secciones. En la primera, se presenta la entropía de Tsallis y su relación con la entropía de \acrshort{bg}. Posteriormente, se discuten las propiedades clave de la estadística de Tsallis, incluyendo su aplicación a sistemas no extensivos. Finalmente, se explorará cómo esta estadística puede aplicarse al estudio de hadrones y al confinamiento de quarks, proporcionando una descripción más general de sistemas complejos.

% NOTA: En la misma referencia se agrega la obtención de la energía de Tsallis para las masas

% % Bibliografía (opcional)
% \begin{thebibliography}{9}
%     \bibitem{TsallisStatistics} 
%     C. Tsallis, "Possible generalization of Boltzmann-Gibbs statistics", Journal of Statistical Physics, 1988.
    
%     \bibitem{TsallisApplications} 
%     C. Tsallis, "Applications of statistical mechanics to complex systems", Physics Reports, 2009.
    
%     \bibitem{BagModelLimitations} 
%     A. Chodos et al., "Limitations of the bag model in describing hadron properties", Physical Review D, 1974.
%     \end{thebibliography}

\section{Presión dentro del hadrón}\label{sec-PresTsa}

A partir de la ecuación \eqref{eq-TsallisTwoSystems}, podemos obtener la entropía dentro de un hadrón que consiste en una mezcla de gases de quarks y gluones. Para ello, comenzamos considerando el cálculo de cada contribución: los quarks, vistos como un gas ideal de \acrfull{fd} ultrarrelativista, y los gluones, vistos como un gas ideal de \acrfull{be} ultrarrelativista. Ambos se consideran sin masa y no interactuantes, ya que la interacción se introduce a través del parámetro \( q \) de Tsallis.

\subsection{Presión de gluones, un gas ideal de Bose - Einstein ultrarelativista}

Los niveles de energía de un bosón en un gas ideal de \acrshort{be} ultrarrelativista están dados por:

\begin{equation}
{\epsilon}_{k} = cp = \hbar c k,
\end{equation}

donde $p$ es la magnitud del momento de las partículas del gas y $k$ la magnitud del vector de onda. La función de partición para este sistema es:

\begin{equation}\label{eq-partfunc}
{\Xi}^{\mathrm{BE}}\left(T,V,\mu\right) = \prod_{k=1}^{\infty}\frac{1}{1-\xi {e}^{-\beta {\epsilon}_{k}}},
\end{equation}

con $\beta = (k_B T)^{-1}$ y $\xi = e^{\beta\mu}$ siendo la fugacidad del gas. El número promedio de partículas en cada estado ${\epsilon}_k$ es:

\begin{equation}
    \left\langle {n}_{k} \right\rangle = \frac{1}{{\xi}^{-1}{e}^{\beta{\epsilon}_{k}}-1} 
\end{equation}

Las cantidades termodinámicas fundamentales se expresan como:

\begin{equation}\label{eq-BE-Ntotal}
N(T,V,\mu) = \sum_k \langle n_k \rangle = \sum_k \frac{1}{{\xi}^{-1}{e}^{\beta{\epsilon}_{k}}-1}
\end{equation}

\begin{equation}\label{eq-BE-Etotal}
E(T,V,\mu) = \sum_k \langle n_k \rangle \epsilon_k = \sum_k \frac{\epsilon_k}{{\xi}^{-1}{e}^{\beta{\epsilon}_{k}}-1}
\end{equation}

Para el límite termodinámico, reemplazamos las sumas por integrales usando la densidad de estados:

\begin{equation}\label{eq-totalestados}
\Sigma = \frac{4\pi V}{(hc)^3} \int_0^\infty \epsilon^2 d\epsilon
\end{equation}

Esto nos lleva a:

\begin{equation}\label{eq-BE-Ntotalint}
N(T,V,\mu) = \frac{4\pi V}{(hc)^3} \int_0^\infty \frac{\epsilon^2 d\epsilon}{{\xi}^{-1}e^{\beta\epsilon}-1}
\end{equation}

\begin{equation}\label{eq-BE-Etotalint}
E(T,V,\mu) = \frac{4\pi V}{(hc)^3} \int_0^\infty \frac{\epsilon^3 d\epsilon}{{\xi}^{-1}e^{\beta\epsilon}-1}
\end{equation}

Para gluones (partículas sin masa con $\mu=0$), estas expresiones se simplifican a:

\begin{equation}\label{eq-BE-Ntotalintnofug}
N(T,V) = \frac{4\pi V}{(hc)^3} \int_0^\infty \frac{\epsilon^2 d\epsilon}{e^{\beta\epsilon}-1}
\end{equation}

\begin{equation}\label{eq-BE-Etotalintnofug}
E(T,V) = \frac{4\pi V}{(hc)^3} \int_0^\infty \frac{\epsilon^3 d\epsilon}{e^{\beta\epsilon}-1}
\end{equation}

Mediante el cambio de variable $x=\beta\epsilon$, obtenemos:

\begin{equation}\label{eq-BE-Ntotalintnofug-x}
N(T,V) = \frac{4\pi V}{(hc)^3 \beta^3} \int_0^\infty \frac{x^2 dx}{e^x-1}
\end{equation}

\begin{equation}\label{eq-BE-Etotalintnofug-x}
E(T,V) = \frac{4\pi V}{(hc)^3 \beta^4} \int_0^\infty \frac{x^3 dx}{e^x-1}
\end{equation}

Estas integrales se resuelven en términos de funciones especiales\footnote{Con $\Gamma(n)$ como la función gamma ($\Gamma(n) = (n-1)!$) 
y $\zeta(n)$ como la función zeta de Riemann 
($\zeta(n) = \sum_{k=1}^\infty k^{-n}$), donde $n \in \mathbb{N}$.}:

\begin{equation}\label{eq-sol-int-N}
    \int_0^\infty \frac{x^2 dx}{e^x-1} = \Gamma(3)\zeta(3) = 2\zeta(3)
\end{equation}

\begin{equation}\label{eq-sol-int-E}
    \int_0^\infty \frac{x^3 dx}{e^x-1} = \Gamma(4)\zeta(4) = \frac{\pi^4}{15}
\end{equation}

Considerando la degeneración de los gluones ($g_G = 16$ para 8 tipos de gluones con 2 proyecciones de espín), obtenemos\footnote{Usando unidades naturales $\hbar={k}_{\mathrm{B}}=c=1$ y $h=2\pi \hbar = 2\pi$}:

\begin{equation}\label{eq-BE-Etotalgluons}
E_G(T,V) = g_G \frac{\pi^2}{30} V T^4
\end{equation}

La presión se obtiene a partir del potencial gran canónico:

\begin{equation}\label{eq-BE-P1}
P = \frac{k_B T}{V} \ln \Xi^{BE}
\end{equation}

Desarrollando esta expresión y realizando integración por partes, encontramos la relación fundamental:

\begin{equation}\label{eq-BE-P2}
P = \frac{1}{3} \frac{E}{V}
\end{equation}

Lo que nos da la presión de gluones:

\begin{equation}\label{eq-BE-Pgluons}
P_G = g_G \frac{\pi^2}{90} T^4
\end{equation}

Finalmente, la entropía se calcula como:

\begin{equation}\label{eq-BE-Sgluons}
S_G = \frac{4}{3} \frac{E_G}{T} = g_G \frac{4\pi^2}{90} V T^3
\end{equation}

\subsubsection*{Resultados fundamentales}
\begin{enumerate}[i.]
    \item \textbf{Energía}: $E_G(T,V) = \dfrac{g_G \pi^2}{30} V T^4$ \eqref{eq-BE-Etotalgluons}
    \item \textbf{Presión}: $P_G(T) = \dfrac{g_G \pi^2}{90} T^4$ \eqref{eq-BE-Pgluons}
    \item \textbf{Entropía}: $S_G(T,V) = \dfrac{2g_G \pi^2}{45} V T^3$ \eqref{eq-BE-Sgluons}
\end{enumerate}

\subsection{Presión de quarks, un gas ideal de Fermi - Dirac ultrarrelativista}\label{sec-Pquarks}

\subsubsection{Descripción del sistema}
Para partículas ultrarrelativistas ($\epsilon = \|\vec{p}\| c$)\footnote{De la relación energía-momento $\epsilon = \sqrt{p^2c^2 + m^2c^4}$ con $m \to 0$}, consideramos:
\begin{itemize}
\item[$\bullet$] Sistema mixto de quarks/antiquarks como gases de Fermi-Dirac acoplados
\item[$\bullet$] Relación entre potenciales químicos: $\mu_+ = -\mu_- = \mu$ (simetría partícula-antipartícula)
\item[$\bullet$] Factor de degeneración: $g_Q = 12$ (2 spines $\times$ 3 colores $\times$ 2 sabores $u,d$)
\end{itemize}

\subsubsection{Propiedades termodinámicas}
La función de partición del sistema combinado es:
\begin{equation}
\Xi(T,V,\mu) = \prod_{\epsilon_+} \left(1 + e^{-\beta(\epsilon_+-\mu)}\right) \times \prod_{\epsilon_-} \left(1 + e^{-\beta(\epsilon_-+\mu)}\right)
\end{equation}

Con las siguientes cantidades fundamentales:
\begin{itemize}
\item[$\bullet$] Número de partículas:
\begin{equation}
N_\pm = \sum_{\epsilon_\pm} \frac{1}{e^{\beta(\epsilon_\pm\mp\mu)} + 1}
\end{equation}

\item[$\bullet$] Exceso neto de quarks:
\begin{equation}\label{eq-FD-Excedente}
N = N_+ - N_- = \frac{g_Q V}{6\pi^2} \left(\pi^2 \mu T^2 + \mu^3\right)
\end{equation}
\end{itemize}

\subsubsection{Energía y presión}
La energía total del sistema:
\begin{equation}\label{eq-FD-Energy}
E_Q = g_Q V T^4 \left[\frac{7\pi^2}{120} + \frac{1}{4}\left(\frac{\mu}{T}\right)^2 + \frac{1}{8\pi^2}\left(\frac{\mu}{T}\right)^4\right]
\end{equation}

La presión del gas de quarks:
\begin{equation}
P_Q = \frac{1}{3}\frac{E_Q}{V} = \frac{g_Q T^4}{3} \left[\frac{7\pi^2}{120} + \frac{\mu^2}{4T^2} + \frac{\mu^4}{8\pi^2T^4}\right]
\end{equation}

\subsubsection{Entropía}
La entropía del sistema:
\begin{equation}\label{eq-FD-Entropy}
S_Q = \frac{4}{3}\frac{E_Q}{T} - \mu\frac{N}{T} = g_Q V T^3 \left[\frac{7\pi^2}{90} + \frac{1}{6}\left(\frac{\mu}{T}\right)^2\right]
\end{equation}

\subsubsection*{Resultados fundamentales}
\begin{enumerate}[i.]
    \item \textbf{Energía}: Ecuación \eqref{eq-FD-Energy} (escala con $T^4$)
    \item \textbf{Presión}: $P_Q = E_Q/3V$ (relación ultrarrelativista)
    \item \textbf{Exceso de quarks}: Ecuación \eqref{eq-FD-Excedente}
    \item \textbf{Entropía}: $\propto V T^3$ (comportamiento característico)
\end{enumerate}

\paragraph{Nota:} Las derivaciones completas se encuentran en el Apéndice \ref{app:math_derivations}.

\section{El protón en el modelo de Tsallis}

Este apartado desarrolla el modelo no extensivo del protón como sistema quark-gluón, generalizando las propiedades termodinámicas mediante el parámetro $q$ de Tsallis.

\subsection{La entropía en el modelo de Tsallis}

Consideramos el protón como un sistema compuesto por:

\begin{enumerate}[i.]
    \item Gas de quarks/antiquarks (Q): Estadística Fermi-Dirac
    \item Gas de gluones (G): Estadística Bose-Einstein
\end{enumerate}

La entropía de Tsallis incorpora correlaciones mediante:

\begin{equation}\label{eq-Entropy-Tsallis}
{S}_{q} = \underbrace{{S}_{1}(Q) + {S}_{1}(G)}_{\text{Estadística de \acrshort{bg}}} + \underbrace{(1-q){S}_{1}(Q){S}_{1}(G)}_{\text{Correlaciones}}
\end{equation}

donde los términos individuales, ${S}_{1}(Q)$ y ${S}_{1}(G)$\footnote{\textbf{Nota sobre la elección de ${S}_{1}(Q)$ o ${S}_{1}(G)$}:
La utilización de entropías BG para los subsistemas individuales refleja que la no extensividad surge exclusivamente de sus correlaciones mutuas, no de sus propiedades internas. Esto garantiza coherencia con el límite BG cuando $q=1$.}, son las entropías BG de quarks (ecuación \ref{eq-FD-Entropy}) y gluones (ecuación \ref{eq-BE-Sgluons}). La derivación detallada se presenta en el Apéndice \ref{app:math_derivations}.

\paragraph{Interpretación física}  
\begin{itemize}
    \item[$\bullet$] $\bm{{S}_1\left(Q\right)}$, $\bm{{S}_1\left(G\right)}$: Entropías de BG independientes
    \item[$\bullet$] $\bm{\left(1-q\right)}$: \\ 
    \begin{tabular}{@{\quad}ll}
        $\triangleright$ $q<1$: & Correlaciones fuertes (superextensividad) \\
        $\triangleright$ $q>1$: & Efectos de exclusión (subextensividad) \\
        $\triangleright$ $q=1$: & Recupera estadística de BG estándar
    \end{tabular}
\end{itemize}

\subsubsection*{Caso límite Boltzmann-Gibbs ($q=1$)}
\begin{equation}\label{eq-BG-limit}
\lim_{q \to 1} {S}_q = {S}_{Q+G} = \underbrace{{g}_Q \left[\frac{7\pi^2}{90} + \frac{1}{6}\left(\frac{\mu}{T}\right)^2\right]V T^3}_{\text{Quarks}} + \underbrace{\frac{2g_G\pi^2}{45}V T^3}_{\text{Gluones}}
\end{equation}

\subsubsection*{Forma explícita con degeneración}
Sustituyendo los factores de degeneración ($g_Q=12$, $g_G=16$):

\begin{equation}\label{eq-Tsallis-Entropy-final}
{S}_q = \underbrace{\left[\frac{74\pi^2}{45} + 2\left(\frac{\mu}{T}\right)^2\right]V T^3}_{\text{Término extensivo}} + \underbrace{\frac{128\pi^2}{15}(1-q)\left[\frac{7\pi^2}{90} + \frac{1}{6}\left(\frac{\mu}{T}\right)^2\right]V^2 T^6}_{\text{Corrección no extensiva}}
\end{equation}

\begin{equation}
\begin{split}
{S}_{q} = & {g}_{Q} \left[\frac{7{\pi}^{2}}{90} + \frac{1}{6} \left(\frac{\mu}{T} \right)^{2} \right] V{T}^{3} + 4{g}_{G} \frac{{\pi}^{2}}{90} V {T}^{3} \\
& + \left(1-q \right) {g}_{Q}{g}_{G} \frac{4{\pi}^{2}}{90} \left[\frac{7{\pi}^{2}}{90} + \frac{1}{6} \left(\frac{\mu}{T} \right)^{2}\right]{V}^{2}{T}^{6}
\end{split}
\end{equation}

Luego de reordenar y sustituir los factores de degeneración se llega a que

\begin{equation}\label{eq-Tsallis-Entropy}
{S}_{q} = \left[\frac{74{\pi}^{2}}{45} + 2 \left(\frac{\mu}{T} \right)^{2} \right]V{T}^{3} +  \frac{128{\pi}^{2}}{15} (1 - q) \left[\frac{7{\pi}^{2}}{90} + \frac{1}{6} \left(\frac{\mu}{T} \right]^{2} \right]{V}^{2}{T}^{6}
\end{equation}

De donde fácilmente se puede comprobar que cuando $q=1$, devolvemos a la expresión \eqref{eq-BG-limit} 

\begin{figure}[!h]
    \centering
    \begin{tikzpicture}[
        node distance=1.5cm,
        concept/.style={rectangle, draw, rounded corners=3pt, fill=blue!10, minimum width=4cm, text width=3.8cm, align=center},
        interaction/.style={ellipse, draw, fill=green!20, minimum width=3cm, align=center},
        result/.style={rectangle, draw, fill=red!10, minimum width=4cm, text width=3.8cm, align=center},
        arrow/.style={->, >=stealth, thick},
        dashedarrow/.style={->, >=stealth, thick, dashed}
    ]
    
    % Nodos principales
    \node[concept] (q) {Parámetro de Tsallis \\ $q$};
    \node[interaction, below left=2cm and 0.5cm of q] (interpretation) {
        \textbf{Interpretación:} \\
        $q=1$: BG estándar \\
        $q>1$: Correlaciones negativas \\
        $q<1$: Correlaciones positivas
    };
    \node[concept, below right=2cm and 0.5cm of q] (entropy) {
        \textbf{Entropía no extensiva} \\
        $S_q = S_1(Q) + S_1(G)$ \\
        $+ (1-q)S_1(Q)S_1(G)$
    };
    \node[result, below=2cm of entropy] (pressure) {
        \textbf{Presión generalizada} \\
        $P_q = P_{\text{BG}}$ \\
        $+ \frac{256\pi^2}{15}(1-q)V T^7$
    };
    
    % Conexiones principales
    \draw[arrow] (q) -- node[left, near start] {Controla} (entropy);
    \draw[arrow] (entropy) -- node[right] {Determina} (pressure);
    
    % Conexiones de interpretación (ajustadas)
    \draw[dashedarrow] (q) -- node[above, sloped, pos=0.6] {Describe} (interpretation.north);
    \draw[dashedarrow] (interpretation.east) -- ++(0.5,0) |- node[pos=0.55, above] {Afecta} (entropy.west);
    
    % Leyenda explicativa reposicionada
    \node[below=0.8cm of interpretation, text width=5cm, align=left, font=\small] {
        \textbf{Relaciones físicas:} \\
        \begin{enumerate}
            \item[$\triangleright$] El parámetro $q$ modifica la entropía
            \item[$\triangleright$] La entropía generalizada afecta la presión
            \item[$\triangleright$] Todo converge a BG cuando $q=1$
        \end{enumerate}
    };
    \end{tikzpicture}
    \caption[Diagrama de relaciones Tsallis]{Jerarquía de relaciones en el modelo: (1) El parámetro $q$ controla la entropía no extensiva $S_q$; (2) Las interpretaciones físicas de $q$ (caja verde) justifican su uso; (3) La entropía modificada determina la presión generalizada $P_q$. Las flechas punteadas indican relaciones secundarias.}
    \label{fig:Tsallis-flow-complete}
\end{figure}

Como muestra la Figura \ref{fig:Tsallis-flow-complete}, el parámetro $q$ no solo controla la entropía del sistema (Ec. \ref{eq-Entropy-Tsallis}), sino que a través de su interpretación como medidor de correlaciones (recuadro verde), afecta directamente las propiedades termodinámicas como la presión (Ec. \ref{eq-Pq-final}).

\subsection{Presión generalizada}\label{subsec-Tsallis-pressure}

Partiendo de la relación de Maxwell:

\begin{equation}\label{eq-Maxwell-Tsallis}
    \left.\frac{\partial{S}_q}{\partial V}\right|_{T,\mu} = \left.\frac{\partial{P}_q}{\partial T}\right|_{V,\mu}
\end{equation}

La integración sobre $T$ (ver la derivación completa de $P_q$ se encuentra en la Sección \ref{app:Tsallis-pressure}) da:

\begin{equation}\label{eq-Pq-final}
    \begin{split}
    {P}_q &= \underbrace{\left[\frac{37\pi^2}{90} + \left(\frac{\mu}{T}\right)^2 + \frac{1}{2\pi^2}\left(\frac{\mu}{T}\right)^4\right]T^4}_{\text{Término extensivo (Presión de \acrshort{bg}, $q=1$)}} \\
    &\quad + \underbrace{\frac{256\pi^2}{15}(1-q)V T^7 \left[\frac{{\pi}^{2}}{90} + \frac{1}{30} \left(\frac{\mu}{T} \right)^{2} \right]}_{\text{Término no extensivo}}
    \end{split}
\end{equation}

\paragraph{Discusión física}  
El término $\propto V T^7$ es característico de Tsallis y:
\begin{itemize}
    \item[$\bullet$] Domina a altas $T$, revelando desviaciones no extensivas
    \item[$\bullet$] Escala con el volumen del sistema
    \item[$\bullet$] Se anula cuando $q=1$ (línea azul en Fig. \ref{fig:comparison})
\end{itemize}

\paragraph{Discusión física del parámetro $q$:}
El parámetro $q$ cuantifica:
\begin{itemize}
    \item[$\bullet$] La fuerza del acoplamiento quark-gluón
    \item[$\bullet$] Desviaciones del equilibrio termodinámico
    \item[$\bullet$] Efectos de memoria a largo alcance
    
\end{itemize}
Valores experimentales típicos en QCD: $q \approx 1.1 - 1.2$ para sistemas de alta energía.% \cite{TsallisQCD}.

\subsection*{Verificación de consistencia}
\begin{itemize}
    \item[$\bullet$] Para $q=1$ se recupera exactamente el caso BG
    \item[$\bullet$] El término no extensivo escala con $V T^7$, dominando a altas temperaturas
    \item[$\bullet$] La simetría $\mu \leftrightarrow -\mu$ se preserva
\end{itemize}

\begin{table}[h]
    \centering
    \caption{Comparación entre estadística BG y Tsallis}
    \begin{tabular}{lcc}
    \toprule
    \textbf{Propiedad} & \textbf{BG ($q=1$)} & \textbf{Tsallis ($q\neq1$)} \\
    \midrule
    Entropía & Aditiva & No aditiva \\
    Presión & $\sim T^4$ & $\sim T^4 + (1-q)V T^7$ \\
    \bottomrule
    \end{tabular}
    \label{tab:BG-vs-Tsallis}
\end{table}
    