\chapter{Presión de quarks y gluones en el marco de la estadística de Tsallis}\label{ch-Tsallis}

% \pagestyle{fancy}
% \fancyhead{} % clear all header fields
% \fancyhead[RO,LE]{\textbf{\chaptername\,\thechapter}  }
% \fancyfoot{} % clear all footer fields
% \fancyfoot[LE,RO]{\thepage}
% % \fancyfoot[LO,CE]{Introducción}
% \fancyfoot[CO,CE]{Estadística de Tsallis}

% =====================Estilo de página================================
\pagestyle{fancy}
\fancyhf{} % Limpia todos los campos de encabezado y pie de página
\fancyhead[LE]{\nouppercase{\textit{\rightmark}}} % Sección en páginas pares (izquierda)
\fancyhead[RO]{\nouppercase{\textit{\rightmark}}} % Sección en páginas impares (derecha)
\fancyhead[RE]{\nouppercase{\hfill \textbf{Capítulo 2. Presión de quarks y gluones en\\el marco de la estadística de Tsallis}}} % Capítulo en páginas impares (izquierda)
\fancyhead[LO]{\nouppercase{\textbf{Capítulo 2. Presión de quarks y gluones en\\el marco de la estadística de Tsallis} \hfill}} % Capítulo en páginas pares (derecha)
\fancyfoot[LE]{\nouppercase{\thepage \hfill {Pressure Distribution Inside Nucleons in a Tsallis-MIT Bag Model}}} % Pie de página en páginas pares
\fancyfoot[RO]{\nouppercase{{Pressure Distribution Inside Nucleons in a Tsallis-MIT Bag Model} \hfill \thepage}} % Pie de página en páginas impares
% =====================================================================

\begin{chaptersummary}[Resumen del capítulo \thechapter: Estadística de Tsallis]
    En este capítulo se introduce el marco teórico de la estadística no extensiva de Tsallis, una generalización de la estadística de Boltzmann-Gibbs. Se presentan las definiciones fundamentales de entropía, energía interna y temperatura en el formalismo \( q \)-generalizado, así como su relación con las funciones de partición y distribuciones de probabilidad. 
\end{chaptersummary}

%%%%%%%%%%%%%%%%%%%%%%%%%%%%%%%%%%%%%%%%%%%%%%%%%%%%%%%%%%%%%%%%%%%%%%%%%%%%%%%%%%%%%%%%%%%%%%%%%%%%%%%%%%%%%
\section{Fundamentos de la estadística de Tsallis}

La estadística no extensiva de Tsallis fue propuesta en 1988 como una generalización del marco tradicional de Boltzmann-Gibbs, con el objetivo de describir sistemas complejos que exhiben correlaciones de largo alcance, inestabilidad caótica o estructuras no ergódicas \cite{Tsallis1988, Tsallis2009, Tsallis_2009}.

El enfoque de Tsallis introduce un parámetro de no extensividad \( q \in \mathbb{R} \), a partir del cual se define una entropía generalizada:

\begin{equation}
    S_q = k_{\mathrm{B}} \frac{1 - \sum_i p_i^q}{q - 1},
\end{equation}

donde \( p_i \) representa la probabilidad del microestado \( i \), y \( k_{\mathrm{B}} \) es la constante de Boltzmann. En el límite \( q \rightarrow 1 \), se recupera la entropía de Boltzmann-Gibbs:

\begin{equation}
    \lim_{q \rightarrow 1} S_q = -k_{\mathrm{B}} \sum_i p_i \ln p_i.
\end{equation}

Este formalismo ha demostrado ser exitoso en la descripción de sistemas como plasmas astrofísicos, turbulencia, colisiones hadrónicas y materia nuclear, donde las distribuciones de energía no presentan un decaimiento exponencial, sino colas de potencia que reflejan una dinámica fuera del equilibrio termodinámico \cite{Tsallis_2014, Barboza_Mendoza_2019, Bhattacharyya_2018}.

\break

Bajo esta generalización, en un sistema constituido por dos subsistemas probabilisticamente independientes $A$ y $B$ (i.e., si ${p}_{ij}^{A+B} = {p}_{i}^{A}{p}_{j}^{B}$), entonces 

\begin{equation}\label{eq-TsallisTwoSystems}
\frac{{S}_{{q}}(A+B)}{{k}_{\mathrm{B}}} = \frac{{S}_{q}(A)}{{k}_{\mathrm{B}}} + \frac{{S}_{q}(B)}{{k}_{\mathrm{B}}} + (1 - q) \frac{{S}_{q}(A)}{{k}_{\mathrm{B}}}\frac{{S}_{q}(B)}{{k}_{\mathrm{B}}}\footnote{Más adelante se usarán unidades naturales por lo que no aparecerá la constante de Boltzmann, ${k}_{\mathrm{B}}$},
\end{equation}

de donde se observa claramente que cuando $q=1$ se regresa a la estadística de \gls{bg} 

En este capítulo, emplearemos esta generalización para explorar la termodinámica de un sistema de quarks y gluones confinado, extendiendo el modelo de bolsa del MIT mediante distribuciones \( q \)-generalizadas, lo cual permitirá caracterizar presiones internas y densidades de energía con mayor generalidad.

%%%%%%%%%%%%%%%%%%%%%%%%%%%%%%%%%%%%%%%%%%%%%%%%%%%%%%%%%%%%%%%%%%%%%%%%%%%%%%%%%%%%%%%%%%%%%%%%%%%%%%%%%%%%%
\section{Presión dentro del hadrón}
\label{sec-PresTsa}

A partir de la ecuación \eqref{eq-TsallisTwoSystems}, es posible obtener la entropía y presión dentro de un hadrón considerado como una mezcla de gases de quarks y gluones\ \cite{Greiner2001}. Para ello, analizamos por separado las contribuciones de cada componente, asumiendo que:

\begin{enumerate}[i.]
    \item Los \emph{quarks} se comportan como un gas ideal ultrarrelativista de Fermi-Dirac sin masa.
    \item Los \emph{gluones} se modelan como un gas ideal ultrarrelativista de Bose-Einstein, también sin masa.
\end{enumerate}

Ambos gases se consideran libres y no interactuantes de forma explícita, ya que la interacción efectiva se introduce mediante el parámetro de no extensividad \( q \) de Tsallis.

% ------------------------------------------------------------------------------------------------ %
\subsection{Presión de gluones: gas ideal ultrarrelativista de Bose-Einstein}

Los niveles de energía de los gluones, como bosones sin masa, se describen por:

\begin{equation}
\epsilon_k = c p = \hbar c k,
\end{equation}

donde \( p \) es el momento y \( k \) el número de onda del modo. La función de partición canónica para este sistema es:

\begin{equation}\label{eq-partfunc}
\Xi^{\mathrm{BE}}(T,V,\mu) = \prod_j \frac{1}{1 - \xi e^{-\beta \epsilon_j}},
\end{equation}

con \( \beta = ({k}_{\mathrm{B}} T)^{-1} \) y fugacidad \( \xi = e^{\beta \mu} \). El número promedio de partículas por estado es:

\begin{equation}
\langle n_j \rangle = \frac{1}{\xi^{-1} e^{\beta \epsilon_j} - 1}.
\end{equation}

Las cantidades termodinámicas fundamentales se obtienen sumando sobre todos los estados:

\begin{align}
N(T,V,\mu) &= \sum_j \langle n_j \rangle = \sum_j \frac{1}{\xi^{-1} e^{\beta \epsilon_j} - 1}, \label{eq-BE-Ntotal} \\
E(T,V,\mu) &= \sum_j \epsilon_j \langle n_j \rangle = \sum_j \frac{\epsilon_j}{\xi^{-1} e^{\beta \epsilon_j} - 1}. \label{eq-BE-Etotal}
\end{align}

\break

En el límite termodinámico, estas sumas se reemplazan por integrales utilizando la densidad de estados:

\begin{equation}
\Sigma(\epsilon) \, d\epsilon = \frac{4\pi V}{(hc)^3} \epsilon^2 d\epsilon,
\end{equation}

lo que permite reescribir:

\begin{align}
{N}_{G}(T,V,\mu) &= \frac{4\pi V}{(hc)^3} \int_0^\infty \frac{\epsilon^2 \, d\epsilon}{\xi^{-1} e^{\beta \epsilon} - 1}, \label{eq-BE-Ntotalint} \\
{E}_{G}(T,V,\mu) &= \frac{4\pi V}{(hc)^3} \int_0^\infty \frac{\epsilon^3 \, d\epsilon}{\xi^{-1} e^{\beta \epsilon} - 1}. \label{eq-BE-Etotalint}
\end{align}

En el caso particular de los gluones, se tiene \( \mu = 0 \), ya que su número no está conservado, y la fugacidad se vuelve \( \xi = 1 \). Las expresiones anteriores se simplifican a:

\begin{align}
{N}_{G}(T,V) &= \frac{4\pi V}{(hc)^3} \int_0^\infty \frac{\epsilon^2 \, d\epsilon}{e^{\beta \epsilon} - 1}, \label{eq-BE-Ntotalintnofug} \\
{E}_{G}(T,V) &= \frac{4\pi V}{(hc)^3} \int_0^\infty \frac{\epsilon^3 \, d\epsilon}{e^{\beta \epsilon} - 1}. \label{eq-BE-Etotalintnofug}
\end{align}

Mediante el cambio de variable \( x = \beta \epsilon \), estas integrales toman la forma:

\begin{align}
{N}_{G}(T,V) &= \frac{4\pi V}{(hc)^3 \beta^3} \int_0^\infty \frac{x^2 dx}{e^x - 1}, \label{eq-BE-Ntotalintnofug-x} \\
{E}_{G}(T,V) &= \frac{4\pi V}{(hc)^3 \beta^4} \int_0^\infty \frac{x^3 dx}{e^x - 1}. \label{eq-BE-Etotalintnofug-x}
\end{align}

Estas integrales se evalúan con funciones especiales:

\begin{align}
\int_0^\infty \frac{x^2 dx}{e^x - 1} &= \Gamma(3)\zeta(3) = 2\zeta(3), \label{eq-sol-int-N} \\
\int_0^\infty \frac{x^3 dx}{e^x - 1} &= \Gamma(4)\zeta(4) = \frac{\pi^4}{15}. \label{eq-sol-int-E}
\end{align}

Finalmente, considerando la degeneración de gluones \( g_G = 16 \) (8 tipos de gluones con 2 polarizaciones), y trabajando en unidades naturales \( \hbar = c = {k}_{\mathrm{B}} = 1 \), obtenemos:

\begin{equation}\label{eq-BE-Etotalgluons}
E_G = g_G \frac{\pi^2}{30} V T^4,
\end{equation}

\begin{equation}\label{eq-BE-Pgluons}
P_G = \frac{1}{3} \frac{E_G}{V} = g_G \frac{\pi^2}{90} T^4.
\end{equation}

La entropía asociada al sistema de gluones se expresa como:

\begin{equation}\label{eq-BE-Sgluons}
S_G = \frac{4}{3} \frac{E_G}{T} = g_G \frac{4\pi^2}{90} V T^3.
\end{equation}

% ------------------------------------------------------------------------------------------------ %
\subsection{Presión de quarks: gas ideal ultrarrelativista de Fermi-Dirac}
\label{sec-Pquarks}

Consideramos ahora la contribución de los quarks y antiquarks, modelados como un sistema mixto de fermiones sin masa en el régimen ultrarrelativista. Bajo esta aproximación, la energía de una partícula se reduce a \( \epsilon = |\vec{p}| c \)\footnote{Resultado obtenido al tomar el límite \( m \to 0 \) en la relación relativista \( \epsilon = \sqrt{p^2 c^2 + m^2 c^4} \).}.

\paragraph{Características del sistema:}
\begin{itemize}
    \item[$\triangleright$] Se trata de un gas compuesto por quarks y antiquarks en equilibrio termodinámico.
    \item[$\triangleright$] Se impone simetría entre partículas y antipartículas: \( \mu_+ = -\mu_- = \mu \).
    \item[$\triangleright$] Factor de degeneración: \( g_Q = 12 \) (2 espines × 3 colores × 2 sabores \( u,d \)).
\end{itemize}

% .-.-.-.-.-.-.-.-.-.-.-.-.-.-.-.-.-.-.-.-.-.-..-.-.-.-.-.-.-.-.-.-.-.-.-.-.-.-.-.-.-.-.-.-.-.-.-.-.%

\subsubsection*{Función de partición}

La función de partición total para el sistema combinado es el producto de las funciones de quarks y antiquarks:

\begin{equation}
\Xi(T, V, \mu) = \prod_j \left(1 + e^{-\beta(\epsilon_j - \mu)}\right)
\prod_j \left(1 + e^{-\beta(\epsilon_j + \mu)}\right),
\end{equation}

donde el índice \( j \) recorre todos los estados energéticos posibles.

% .-.-.-.-.-.-.-.-.-.-.-.-.-.-.-.-.-.-.-.-.-.-..-.-.-.-.-.-.-.-.-.-.-.-.-.-.-.-.-.-.-.-.-.-.-.-.-.-.%

\subsubsection*{Número de partículas y exceso neto}

La población media de partículas en cada estado se expresa como:

\begin{equation}
N_\pm = \sum_j \frac{1}{e^{\beta(\epsilon_j \mp \mu)} + 1},
\end{equation}

y el número neto de quarks (exceso sobre antiquarks) es:

\begin{equation}\label{eq-FD-Excedente}
{N}_{Q} = N_+ - N_- = \frac{g_Q V}{6 \pi^2} \left( \pi^2 \mu T^2 + \mu^3 \right),
\end{equation}

resultado obtenido tras pasar al límite termodinámico e integrar sobre la densidad de estados.

% .-.-.-.-.-.-.-.-.-.-.-.-.-.-.-.-.-.-.-.-.-.-..-.-.-.-.-.-.-.-.-.-.-.-.-.-.-.-.-.-.-.-.-.-.-.-.-.-.%

\subsubsection*{Energía total y presión}

La energía total se calcula a partir de la suma de contribuciones energéticas de quarks y antiquarks:

\begin{equation}\label{eq-FD-Energy}
E_Q = g_Q V T^4 \left[
\frac{7 \pi^2}{120} + \frac{1}{4} \left( \frac{\mu}{T} \right)^2 + \frac{1}{8 \pi^2} \left( \frac{\mu}{T} \right)^4
\right].
\end{equation}

Dado que se trata de un gas ultrarrelativista, la presión asociada satisface \( P_Q = E_Q / (3V) \), por lo que:

\begin{equation}
P_Q = \frac{g_Q T^4}{3} \left[
\frac{7 \pi^2}{120} + \frac{\mu^2}{4 T^2} + \frac{\mu^4}{8 \pi^2 T^4}
\right].
\end{equation}

% .-.-.-.-.-.-.-.-.-.-.-.-.-.-.-.-.-.-.-.-.-.-..-.-.-.-.-.-.-.-.-.-.-.-.-.-.-.-.-.-.-.-.-.-.-.-.-.-.%

\subsubsection*{Entropía}

Aplicando la relación termodinámica:

\[
S = \frac{{E}_{Q} + P V - \mu {N}_{Q}}{T},
\]

se obtiene la entropía total del sistema:

\begin{equation}\label{eq-FD-Entropy}
S_Q = g_Q V T^3 \left[
\frac{7 \pi^2}{90} + \frac{1}{6} \left( \frac{\mu}{T} \right)^2
\right].
\end{equation}

\paragraph{Nota:} Las derivaciones completas de estas expresiones pueden consultarse en el Apéndice~\ref{app:math_derivations}.


\section{El protón en el modelo de Tsallis}
\label{sec-proton-tsallis}

Una vez obtenidas las propiedades termodinámicas de los gases de gluones y quarks en el marco de la estadística de Boltzmann-Gibbs (BG), es posible extender este análisis al contexto de la estadística de Tsallis, considerando al protón como un sistema compuesto por ambos componentes correlacionados.

\paragraph{Configuración del sistema}

El modelo no extensivo considera al protón como una mezcla compuesta por:

\begin{enumerate}[i.]
    \item Un gas de quarks y antiquarks sin masa, regido por la estadística de Fermi-Dirac.
    \item Un gas de gluones, bosones sin masa, obedeciendo estadística de Bose-Einstein.
\end{enumerate}

Ambos sistemas se consideran independientes a nivel de estado, pero correlacionados globalmente mediante una formulación no extensiva.

\subsection{Entropía generalizada de Tsallis}

La entropía total del sistema se describe mediante la expresión de Tsallis para dos subsistemas independientes \( A \) y \( B \):

\begin{equation}
S_q(A+B) = S_q(A) + S_q(B) + (1 - q) S_q(A) S_q(B),
\end{equation}

donde \( q \in \mathbb{R} \) es el parámetro de no extensividad. Aplicado al sistema hadrónico:

\begin{equation}\label{eq-Entropy-Tsallis}
S_q = S_1(Q) + S_1(G) + (1 - q) S_1(Q) S_1(G),
\end{equation}

donde \( S_1(Q) \) y \( S_1(G) \) son las entropías de quarks y gluones calculadas bajo la estadística estándar de Boltzmann-Gibbs.

\paragraph{Interpretación física del parámetro \( q \):}
\begin{itemize}
    \item[$\triangleright$] \( q = 1 \): Recupera la estadística extensiva estándar.
    \item[$\triangleright$] \( q > 1 \): Describe correlaciones negativas (efectos de exclusión o repulsión).
    \item[$\triangleright$] \( q < 1 \): Describe correlaciones positivas (agrupamiento o efectos de atracción).
\end{itemize}

Este parámetro ha sido empleado exitosamente para describir sistemas como plasmas cuánticos, colisiones hadrónicas y materia nuclear a altas energías \cite{Tsallis_2014, Bhattacharyya_2018, Barboza_Mendoza_2019}.

\subsubsection*{Límite extensivo \( q \to 1 \)}

En el caso límite de \( q = 1 \), la expresión \eqref{eq-Entropy-Tsallis} se reduce a:

\begin{equation}\label{eq-BG-limit}
S_{q \to 1} = S_{Q+G} = S_1(Q) + S_1(G),
\end{equation}

cuyas formas explícitas, con degeneraciones \( g_Q = 12 \), \( g_G = 16 \), son:

\begin{align}
S_1(Q) &= g_Q V T^3 \left[ \frac{7 \pi^2}{90} + \frac{1}{6} \left( \frac{\mu}{T} \right)^2 \right], \\
S_1(G) &= g_G V T^3 \frac{4 \pi^2}{90}.
\end{align}

\subsubsection*{Forma explícita general para \( S_q \)}

Sustituyendo en \eqref{eq-Entropy-Tsallis}:

\begin{equation}\label{eq-Tsallis-Entropy-final}
\begin{split}
S_q &= \left[ \frac{74 \pi^2}{45} + 2 \left( \frac{\mu}{T} \right)^2 \right] V T^3 \\
&\quad + \frac{128 \pi^2}{15} (1 - q) \left[ \frac{7 \pi^2}{90} + \frac{1}{6} \left( \frac{\mu}{T} \right)^2 \right] V^2 T^6,
\end{split}
\end{equation}

donde el primer término corresponde a la suma extensiva y el segundo a la corrección cuadrática en volumen derivada de las correlaciones no extensivas.

\subsection{Presión generalizada en el marco de Tsallis}
\label{subsec-Tsallis-pressure}

Aplicando la relación termodinámica de Maxwell para sistemas con entropía generalizada:

\begin{equation}\label{eq-Maxwell-Tsallis}
\left( \frac{\partial S_q}{\partial V} \right)_{T, \mu} = \left( \frac{\partial P_q}{\partial T} \right)_{V, \mu},
\end{equation}

y considerando la forma explícita de \( S_q \) obtenida en \eqref{eq-Tsallis-Entropy-final}, se obtiene (ver derivación completa en el Apéndice \ref{app:Tsallis-pressure}):

\begin{equation}\label{eq-Pq-final}
\begin{split}
P_q &= \underbrace{T^4 \left[ \frac{37 \pi^2}{90} + \left( \frac{\mu}{T} \right)^2 + \frac{1}{2 \pi^2} \left( \frac{\mu}{T} \right)^4 \right]}_{\text{Componente extensiva}} \\
&\quad + \underbrace{\frac{256 \pi^2}{15} (1 - q) V T^7 \left[ \frac{\pi^2}{90} + \frac{1}{30} \left( \frac{\mu}{T} \right)^2 \right]}_{\text{Corrección no extensiva}}.
\end{split}
\end{equation}

\paragraph{Análisis de los términos:}
\begin{itemize}
    \item El término \( \sim T^4 \) es la presión total bajo estadística BG.
    \item La corrección \( \sim V T^7 (1-q) \) revela el impacto de las correlaciones no extensivas, particularmente dominante a altas temperaturas.
    \item Se preserva la simetría \( \mu \leftrightarrow -\mu \).
\end{itemize}

\subsubsection*{Comparación entre estadística BG y Tsallis}

\begin{table}[h]
    \centering
    \caption{Comparación entre resultados termodinámicos en BG y Tsallis}
    \begin{tabular}{lcc}
        \toprule
        \textbf{Propiedad} & \textbf{BG (\( q = 1 \))} & \textbf{Tsallis (\( q \neq 1 \))} \\
        \midrule
        Entropía & Aditiva & No aditiva (correlaciones) \\
        Presión & \( \sim T^4 \) & \( \sim T^4 + (1 - q) V T^7 \) \\
        \bottomrule
    \end{tabular}
    \label{tab:BG-vs-Tsallis}
\end{table}


\begin{figure}[!h]
    \centering
    \begin{tikzpicture}[
        node distance=1.5cm,
        concept/.style={rectangle, draw, rounded corners=3pt, fill=blue!10, minimum width=4cm, text width=3.8cm, align=center},
        interaction/.style={ellipse, draw, fill=green!20, minimum width=3cm, align=center},
        result/.style={rectangle, draw, fill=red!10, minimum width=4cm, text width=3.8cm, align=center},
        arrow/.style={->, >=stealth, thick},
        dashedarrow/.style={->, >=stealth, thick, dashed}
    ]
    
    % Nodos principales
    \node[concept] (q) {Parámetro de Tsallis \\ $q$};
    \node[interaction, below left=2cm and 0.5cm of q] (interpretation) {
        \textbf{Interpretación:} \\
        $q=1$: BG estándar \\
        $q>1$: Correlaciones negativas \\
        $q<1$: Correlaciones positivas
    };
    \node[concept, below right=2cm and 0.5cm of q] (entropy) {
        \textbf{Entropía no extensiva} \\
        $S_q = S_1(Q) + S_1(G)$ \\
        $+ (1-q)S_1(Q)S_1(G)$
    };
    \node[result, below=2cm of entropy] (pressure) {
        \textbf{Presión generalizada} \\
        $P_q = P_{\text{BG}}$ \\
        $+ \frac{256\pi^2}{15}(1-q)V T^7$
    };
    
    % Conexiones principales
    \draw[arrow] (q) -- node[left, near start] {Controla} (entropy);
    \draw[arrow] (entropy) -- node[right] {Determina} (pressure);
    
    % Conexiones de interpretación (ajustadas)
    \draw[dashedarrow] (q) -- node[above, sloped, pos=0.6] {Describe} (interpretation.north);
    \draw[dashedarrow] (interpretation.east) -- ++(0.5,0) |- node[pos=0.55, above] {Afecta} (entropy.west);
    
    % Leyenda explicativa reposicionada
    \node[below=0.8cm of interpretation, text width=5cm, align=left, font=\small] {
        \textbf{Relaciones físicas:} \\
        \begin{enumerate}
            \item[$\triangleright$] El parámetro $q$ modifica la entropía
            \item[$\triangleright$] La entropía generalizada afecta la presión
            \item[$\triangleright$] Todo converge a BG cuando $q=1$
        \end{enumerate}
    };
    \end{tikzpicture}
    \caption[Diagrama de relaciones Tsallis]{Jerarquía de relaciones en el modelo: (1) El parámetro $q$ controla la entropía no extensiva $S_q$; (2) Las interpretaciones físicas de $q$ (caja verde) justifican su uso; (3) La entropía modificada determina la presión generalizada $P_q$. Las flechas punteadas indican relaciones secundarias.}
    \label{fig:Tsallis-flow-complete}
\end{figure}

Como muestra la Figura \ref{fig:Tsallis-flow-complete}, el parámetro $q$ no solo controla la entropía del sistema (Ec. \ref{eq-Entropy-Tsallis}), sino que a través de su interpretación como medidor de correlaciones (recuadro verde), afecta directamente las propiedades termodinámicas como la presión (Ec. \ref{eq-Pq-final}).

\begin{remark}[Relevancia física del parámetro \( q \)]
El parámetro \( q \) funciona como cuantificador de correlaciones a largo alcance, acoplamientos residuales o desviaciones del equilibrio térmico. Su valor en sistemas de QCD típicamente se encuentra en el rango \( q \approx 1.1 - 1.2 \), según se ha observado en colisiones hadrónicas y sistemas de materia nuclear \cite{Bhattacharyya_2018, Tsallis_2014}.
\end{remark}


\section*{Conclusiones del capítulo}
Este capítulo ha establecido una base termodinámica sólida para describir la presión y la entropía en sistemas de quarks y gluones mediante el formalismo no extensivo de Tsallis. La generalización obtenida permite capturar efectos de correlación entre los constituyentes, que serán explorados más adelante en contextos fenomenológicos o comparaciones con datos experimentales.