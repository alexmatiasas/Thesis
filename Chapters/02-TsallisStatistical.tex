\chapter{Presión de quarks y gluones en el marco de la estadística de Tsallis}\label{ch-Tsallis}

% \pagestyle{fancy}
% \fancyhead{} % clear all header fields
% \fancyhead[RO,LE]{\textbf{\chaptername\,\thechapter}  }
% \fancyfoot{} % clear all footer fields
% \fancyfoot[LE,RO]{\thepage}
% % \fancyfoot[LO,CE]{Introducción}
% \fancyfoot[CO,CE]{Estadística de Tsallis}

% =====================Estilo de página================================
\pagestyle{fancy}
\fancyhf{} % Limpia todos los campos de encabezado y pie de página
\fancyhead[LE]{\nouppercase{\textit{\rightmark}}} % Sección en páginas pares (izquierda)
\fancyhead[RO]{\nouppercase{\textit{\rightmark}}} % Sección en páginas impares (derecha)
\fancyhead[RE]{\nouppercase{\hfill \textbf{Capítulo 2. Presión de quarks y gluones en\\el marco de la estadística de Tsallis}}} % Capítulo en páginas impares (izquierda)
\fancyhead[LO]{\nouppercase{\textbf{Capítulo 2. Presión de quarks y gluones en\\el marco de la estadística de Tsallis} \hfill}} % Capítulo en páginas pares (derecha)
\fancyfoot[LE]{\nouppercase{\thepage \hfill {Pressure Distribution Inside Nucleons in a Tsallis-MIT Bag Model}}} % Pie de página en páginas pares
\fancyfoot[RO]{\nouppercase{{Pressure Distribution Inside Nucleons in a Tsallis-MIT Bag Model} \hfill \thepage}} % Pie de página en páginas impares
% =====================================================================

\begin{chaptersummary}[Resumen del capítulo \thechapter: Estadística de Tsallis]
    En este capítulo se introduce el marco teórico de la estadística no extensiva de Tsallis, una generalización de la estadística de Boltzmann-Gibbs. Se presentan las definiciones fundamentales de entropía, energía interna y temperatura en el formalismo \( q \)-generalizado, así como su relación con las funciones de partición y distribuciones de probabilidad. 
\end{chaptersummary}

\section{Fundamentos de la estadística de Tsallis}

La estadística no extensiva de Tsallis fue propuesta en 1988 como una generalización del marco tradicional de Boltzmann-Gibbs, con el objetivo de describir sistemas complejos que exhiben correlaciones de largo alcance, inestabilidad caótica o estructuras no ergódicas \cite{Tsallis1988, Tsallis2009, Tsallis_2009}.

El enfoque de Tsallis introduce un parámetro de no extensividad \( q \in \mathbb{R} \), a partir del cual se define una entropía generalizada:

\begin{equation}
    S_q = k_{\mathrm{B}} \frac{1 - \sum_i p_i^q}{q - 1},
\end{equation}

donde \( p_i \) representa la probabilidad del microestado \( i \), y \( k_{\mathrm{B}} \) es la constante de Boltzmann. En el límite \( q \rightarrow 1 \), se recupera la entropía de Boltzmann-Gibbs:

\begin{equation}
    \lim_{q \rightarrow 1} S_q = -k_{\mathrm{B}} \sum_i p_i \ln p_i.
\end{equation}

Este formalismo ha demostrado ser exitoso en la descripción de sistemas como plasmas astrofísicos, turbulencia, colisiones hadrónicas y materia nuclear, donde las distribuciones de energía no presentan un decaimiento exponencial, sino colas de potencia que reflejan una dinámica fuera del equilibrio termodinámico \cite{Tsallis_2014, Barboza_Mendoza_2019, Bhattacharyya_2018}.

\break

Bajo esta generalización, en un sistema constituido por dos subsistemas probabilisticamente independientes $A$ y $B$ (i.e., si ${p}_{ij}^{A+B} = {p}_{i}^{A}{p}_{j}^{B}$), entonces 

\begin{equation}\label{eq-TsallisTwoSystems}
\frac{{S}_{{q}}(A+B)}{{k}_{\mathrm{B}}} = \frac{{S}_{q}(A)}{{k}_{\mathrm{B}}} + \frac{{S}_{q}(B)}{{k}_{\mathrm{B}}} + (1 - q) \frac{{S}_{q}(A)}{{k}_{\mathrm{B}}}\frac{{S}_{q}(B)}{{k}_{\mathrm{B}}}\footnote{Más adelante se usarán unidades naturales por lo que no aparecerá la constante de Boltzmann, ${k}_{\mathrm{B}}$},
\end{equation}

de donde se observa claramente que cuando $q=1$ se regresa a la estadística de \gls{bg} 

En este capítulo, emplearemos esta generalización para explorar la termodinámica de un sistema de quarks y gluones confinado, extendiendo el modelo de bolsa del MIT mediante distribuciones \( q \)-generalizadas, lo cual permitirá caracterizar presiones internas y densidades de energía con mayor generalidad.

\section{Presión dentro del hadrón}
\label{sec-PresTsa}

A partir de la ecuación \eqref{eq-TsallisTwoSystems}, es posible obtener la entropía y presión dentro de un hadrón considerado como una mezcla de gases de quarks y gluones \cite{Greiner2001}. Para ello, analizamos por separado las contribuciones de cada componente, asumiendo que:

\begin{enumerate}[i.]
    \item Los \emph{quarks} se comportan como un gas ideal ultrarrelativista de Fermi-Dirac sin masa.
    \item Los \emph{gluones} se modelan como un gas ideal ultrarrelativista de Bose-Einstein, también sin masa.
\end{enumerate}

Ambos gases se consideran libres y no interactuantes de forma explícita, ya que la interacción efectiva se introduce mediante el parámetro de no extensividad \( q \) de Tsallis.

\subsection{Presión de gluones: gas ideal ultrarrelativista de Bose-Einstein}

Los niveles de energía de los gluones, como bosones sin masa, se describen por:

\begin{equation}
\epsilon_k = c p = \hbar c k,
\end{equation}

donde \( p \) es el momento y \( k \) el número de onda del modo. La función de partición canónica para este sistema es:

\begin{equation}\label{eq-partfunc}
\Xi^{\mathrm{BE}}(T,V,\mu) = \prod_k \frac{1}{1 - \xi e^{-\beta \epsilon_k}},
\end{equation}

con \( \beta = (k_{\mathrm{B}} T)^{-1} \) y fugacidad \( \xi = e^{\beta \mu} \). El número promedio de partículas por estado es:

\begin{equation}
\langle n_k \rangle = \frac{1}{\xi^{-1} e^{\beta \epsilon_k} - 1}.
\end{equation}

Las cantidades termodinámicas fundamentales se obtienen sumando sobre todos los estados:

\begin{align}
N(T,V,\mu) &= \sum_k \langle n_k \rangle = \sum_k \frac{1}{\xi^{-1} e^{\beta \epsilon_k} - 1}, \label{eq-BE-Ntotal} \\
E(T,V,\mu) &= \sum_k \epsilon_k \langle n_k \rangle = \sum_k \frac{\epsilon_k}{\xi^{-1} e^{\beta \epsilon_k} - 1}. \label{eq-BE-Etotal}
\end{align}

\break

En el límite termodinámico, estas sumas se reemplazan por integrales utilizando la densidad de estados:

\begin{equation}
\Sigma(\epsilon) \, d\epsilon = \frac{4\pi V}{(hc)^3} \epsilon^2 d\epsilon,
\end{equation}

lo que permite reescribir:

\begin{align}
N(T,V,\mu) &= \frac{4\pi V}{(hc)^3} \int_0^\infty \frac{\epsilon^2 \, d\epsilon}{\xi^{-1} e^{\beta \epsilon} - 1}, \label{eq-BE-Ntotalint} \\
E(T,V,\mu) &= \frac{4\pi V}{(hc)^3} \int_0^\infty \frac{\epsilon^3 \, d\epsilon}{\xi^{-1} e^{\beta \epsilon} - 1}. \label{eq-BE-Etotalint}
\end{align}

En el caso particular de los gluones, se tiene \( \mu = 0 \), ya que su número no está conservado, y la fugacidad se vuelve \( \xi = 1 \). Las expresiones anteriores se simplifican a:

\begin{align}
N(T,V) &= \frac{4\pi V}{(hc)^3} \int_0^\infty \frac{\epsilon^2 \, d\epsilon}{e^{\beta \epsilon} - 1}, \label{eq-BE-Ntotalintnofug} \\
E(T,V) &= \frac{4\pi V}{(hc)^3} \int_0^\infty \frac{\epsilon^3 \, d\epsilon}{e^{\beta \epsilon} - 1}. \label{eq-BE-Etotalintnofug}
\end{align}

Mediante el cambio de variable \( x = \beta \epsilon \), estas integrales toman la forma:

\begin{align}
N(T,V) &= \frac{4\pi V}{(hc)^3 \beta^3} \int_0^\infty \frac{x^2 dx}{e^x - 1}, \label{eq-BE-Ntotalintnofug-x} \\
E(T,V) &= \frac{4\pi V}{(hc)^3 \beta^4} \int_0^\infty \frac{x^3 dx}{e^x - 1}. \label{eq-BE-Etotalintnofug-x}
\end{align}

Estas integrales se evalúan con funciones especiales:

\begin{align}
\int_0^\infty \frac{x^2 dx}{e^x - 1} &= \Gamma(3)\zeta(3) = 2\zeta(3), \label{eq-sol-int-N} \\
\int_0^\infty \frac{x^3 dx}{e^x - 1} &= \Gamma(4)\zeta(4) = \frac{\pi^4}{15}. \label{eq-sol-int-E}
\end{align}

Finalmente, considerando la degeneración de gluones \( g_G = 16 \) (8 tipos de gluones con 2 polarizaciones), y trabajando en unidades naturales \( \hbar = c = k_{\mathrm{B}} = 1 \), obtenemos:

\begin{equation}\label{eq-BE-Etotalgluons}
E_G = g_G \frac{\pi^2}{30} V T^4,
\end{equation}

\begin{equation}\label{eq-BE-Pgluons}
P_G = \frac{1}{3} \frac{E_G}{V} = g_G \frac{\pi^2}{90} T^4.
\end{equation}

La entropía asociada al sistema de gluones se expresa como:

\begin{equation}\label{eq-BE-Sgluons}
S_G = \frac{4}{3} \frac{E_G}{T} = g_G \frac{4\pi^2}{90} V T^3.
\end{equation}