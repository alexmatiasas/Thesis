\chapter{Resultados y conclusiones}
%\addcontentsline{toc}{chapter}{Introduction}

\pagestyle{fancy}
\fancyhf{}
\fancyhead[LE]{\nouppercase{\textbf{\leftmark}\hfill\textit{\rightmark}}}
\fancyhead[RO]{\nouppercase{\textit{\rightmark}\hfill\textbf{\leftmark}}}
\fancyfoot[LE]{\nouppercase{\thepage\hfill Pressure Distribution Inside Nucleons in a Tsallis-MIT Bag Model}}
\fancyfoot[RO]{\nouppercase{Pressure Distribution Inside Nucleons in a Tsallis-MIT Bag Model \hfill \thepage}}

\section{Comparación con resultados de Lattice QCD}

La figura~\ref{fig:Results_LQCD} compara las distribuciones de presión radial \( r^2 P(r) \) obtenidas con el modelo Tsallis-MIT para distintos valores del potencial químico \( \mu \), con los resultados recientes de Lattice QCD reportados por Shanahan y Detmold~\cite{shanahanPressureDistributionShear2019}. Las curvas continuas representan nuestras predicciones, mientras que los puntos corresponden a distribuciones extraídas numéricamente mediante simulaciones de QCD en el retículo.

\begin{wrapfigure}{o}{0.58\textwidth}
    \centering
    \includegraphics[width=0.58\textwidth]{./Images/MIT-BagModel.png}
    \caption[Comparación de presión radial con Lattice QCD]{\emph{Distribuciones radiales de presión \( r^2 P(r) \) obtenidas con el modelo Tsallis-MIT para distintos potenciales químicos \( \mu \) (líneas negras), comparadas con resultados de Lattice QCD de la referencia~\cite{shanahanPressureDistributionShear2019}. Los círculos representan la presión total \( P_q \), y los triángulos invertidos la componente gluónica \( P_G \).}}
    \label{fig:Results_LQCD}
\end{wrapfigure}

Se observa un notable acuerdo cualitativo entre ambas aproximaciones en el rango \( r \lesssim 1.2\,\mathrm{fm} \), donde la presión repulsiva alcanza su máximo alrededor de \( r \approx 0.5\,\mathrm{fm} \). Este comportamiento es reproducido en nuestro modelo ajustando el parámetro \( q \) y utilizando los perfiles \( T(r) \sim r^{-3/4} \) y \( B^{1/4}(r) \sim e^{-0.2936r} \), desarrollados en el Capítulo~\ref{ch-ProtonBagParameters}.

\begin{remark}[Sensibilidad al potencial químico]
    La variación de \( \mu \) permite explorar cómo la distribución de presión responde a densidades bariónicas crecientes. A medida que \( \mu \) aumenta, la presión repulsiva en la región central crece ligeramente, mientras que la zona de presión negativa se intensifica, indicando mayor confinamiento.
\end{remark}

Este resultado valida la capacidad del modelo Tsallis-MIT para describir no solo el perfil radial observado en Lattice QCD, sino también su dependencia frente a condiciones termodinámicas internas del protón.

\section{Influencia del potencial químico en la presión total}

La figura~\ref{fig:TotalPressureTsallis} muestra la evolución de la presión radial ponderada \( r^2 P(r) \) para distintos valores de \( \mu \), manteniendo fijo el parámetro de Tsallis en \( q = 1.05 \). Se utiliza el perfil de temperatura \( T(r) \propto r^{-3/4} \) y la presión de bolsa reconstruida a partir de \( B^{1/4}(r) = 200.9\,e^{-0.2936r}\,\mathrm{MeV} \).

\begin{figure}
    \centering
    \includegraphics[width=0.58\textwidth]{./Images/PressureDistributionsTot-Q-G.png}
    \caption[Descomposición de presión radial]{\emph{Presión total \( P_q(r) \) (línea continua), presión gluónica \( P_G(r) \) (línea punteada larga) y presión de quarks \( P_Q(r) \) (línea punteada corta) para \( \mu = 100\,\mathrm{MeV} \) y \( q = 1.05 \). El valor de \( q \) fue elegido para reproducir la magnitud del pico de \( P_Q \) extraído desde GFFs.}}
    \label{fig:PressureDecompResult}
\end{figure}

Al incrementar el potencial químico:
\begin{itemize}
    \item La presión repulsiva cerca del centro se incrementa ligeramente.
    \item La transición hacia la región de presión negativa se vuelve más abrupta y profunda.
\end{itemize}

Este comportamiento refleja cómo el confinamiento se refuerza a mayores densidades bariónicas, en concordancia con expectativas de transiciones de fase a alta densidad.

El modelo Tsallis-MIT logra capturar esta dinámica sin necesidad de ajustes adicionales, destacando su versatilidad para describir la estructura interna del protón bajo diferentes condiciones.

\section{Extracción de la presión de gluones y validación de q}

La figura~\ref{fig:PressureDecompResult} muestra nuevamente las tres contribuciones a la presión radial ponderada \( r^2 P(r) \): la presión total \( P_q(r) \), la presión de quarks \( P_Q(r) \) extraída desde GFFs, y la presión de gluones \( P_G(r) = P_q(r) - P_Q(r) \). Este gráfico ya había sido presentado en el Capítulo~\ref{ch:TotalPandGluons} como parte del método de extracción, pero aquí lo utilizamos para validar el valor adoptado para el parámetro no extensivo \( q = 1.05 \).

Se observa que el valor \( q = 1.05 \) permite ajustar la presión total de forma que su pico coincida en magnitud con el de \( P_Q \), aunque desfasado en radio. Esto respalda la idea de que \( q \) encapsula los efectos efectivos del confinamiento, como se discutió en el Capítulo~\ref{ch-PhysicalMeaningQ}.

\begin{figure}
    \centering
    \includegraphics[width=0.58\textwidth]{./Images/TotalPressureTsallis.png}
    \caption[Efecto de \( \mu \) en la presión total radial]{\emph{Distribución radial ponderada \( r^2 P(r) \) obtenida con el modelo Tsallis-MIT para \( q = 1.05 \) y potenciales químicos \( \mu = 0, 20, 40, 60, 80\,\mathrm{MeV} \). La presión de bolsa se reconstruye a partir del ajuste \( B^{1/4}(r) = 200.9\,e^{-0.2936r}\,\mathrm{MeV} \).}}
    \label{fig:TotalPressureTsallis}
\end{figure}

\section{Reconstrucción de presiones equivalentes con distintos \( q \)}

Como se analizó en el Capítulo~\ref{ch-PhysicalMeaningQ}, es posible reconstruir perfiles de presión efectivos sin una presión de bolsa explícita, mediante la variación funcional del parámetro \( q \). En la figura~\ref{fig:B_reconstructed_combined}, se comparan dos distribuciones \( r^2 P(r) \) calculadas con valores distintos de \( q \) y sus correspondientes formas funcionales para \( B(r) \), mostrando que ambos enfoques reproducen perfiles similares.

\begin{figure}[H]
    \centering
    \includegraphics[width=0.75\textwidth]{./Images/Comparacion_B_old_new_combined.png}
    \caption[Comparación de perfiles con distintos \( q \)]{\emph{Comparación entre distribuciones \( r^2 P(r) \) obtenidas con \( q = 1.002 \) y una forma tradicional de \( B(r) \), y con \( q = 0.9 \) usando una forma funcional reconstruida. Ambos perfiles son compatibles en forma general, lo que respalda la hipótesis de que \( q \) puede absorber el efecto de confinamiento.}}
    \label{fig:B_reconstructed_combined}
\end{figure}

Este resultado fortalece la idea de que \( q \) puede interpretarse como un parámetro dinámico que encapsula la física del confinamiento, y no solo como un modificador estadístico.

\section{Conclusiones generales}

\begin{itemize}
    \item Se construyó un modelo efectivo basado en la estadística de Tsallis acoplado al modelo de bolsa MIT, capaz de reproducir distribuciones de presión hadrónica consistentes con QCD en el retículo.
    \item La dependencia radial del parámetro \( q(r) \) permite reinterpretar el confinamiento como un fenómeno emergente asociado a correlaciones no extensivas, prescindiendo potencialmente de la presión de bolsa fija.
    \item La validez del modelo se verificó frente a resultados de Lattice QCD y mediante reconstrucción inversa de perfiles de presión.
    \item La sensibilidad del modelo al potencial químico sugiere posibles aplicaciones en materia densa o escenarios de transición de fase (neutron stars, heavy-ion collisions, etc.).
\end{itemize}

\begin{remark}[Perspectivas]
    Futuras investigaciones podrían integrar una evolución dinámica de \( q(r,t) \), explorar efectos de anisotropías, o conectar con observables experimentales más allá de GFFs.
\end{remark}

\section*{Palabras finales}

Este trabajo presentó una formulación extendida del modelo de bolsa para nucleones mediante el uso de estadística no extensiva de Tsallis, integrando de manera natural los efectos del confinamiento y la correlación de largo alcance. Más allá de los resultados numéricos y validaciones presentadas, este estudio busca abrir nuevas líneas de exploración en la descripción efectiva de la materia hadrónica. 

Queda como perspectiva futura el desarrollo de modelos dinámicos basados en \( q(r,t) \), así como su contraste directo con datos experimentales provenientes de dispersión profunda y colisiones de alta energía.

\vspace{1em}
\noindent
\textit{“En la frontera entre estadística y confinamiento, emergen nuevas formas de comprender la estructura de la materia.”}