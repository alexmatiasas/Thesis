\chapter{Appendix On Collisions}
\label{ch: appendix-collisions}

\section{On collisions}

\begin{equation}
4
\end{equation}

\section{Python code}



% Renombrando las figuras
\renewcommand{\lstlistingname}{Fragmento de código}

\begin{lstlisting}[language=Python, caption=Python example, label=code:python_example]
    # Scripts/pressure.py
    import numpy as np
    from sympy import Symbol, Function, symbols, Pow, Integral, sin, pi, oo, lambdify

    class PressureDistribution:
        def __init__(self, r_min=0., r_max=2.4, n_samples=200, units="GeV/fm^3"):
            self.r_min = r_min
            self.r_max = r_max
            self.r = np.linspace(r_min, r_max, n_samples)
            self.n_samples = n_samples
            self.units = units

        def p_distr_tex(self):
            P = self.p_distr_exp()
            return latex(P)

    class QuarksPressureDistribution(PressureDistribution):
        def __init__(self, *args, **kwargs):
            super().__init__(*args, **kwargs)
            self.units = '\\times 10^{-2} GeV/fm^3'
    
        def p_distr_exp(self):
            d_1 = Function('d_1')
            t, M, alpha, k_p = symbols('t M alpha k_p', real=True)
            x = Symbol('x', real=True)
            r = Symbol('r', real=True)
            d_1 = Lambda(t, d_1(0) * (1 - (t / Pow(M, 2))) ** -alpha)
            P = Integral(-d_1(-x**2) * x**3 * sin(r * x) / (pi**2 * k_p * r), (x, 0, oo)) \
                .subs({alpha: 3, k_p: 0.55, M: 5, d_1(0): -2.04}) \
                .doit().subs({arg(r): 0}).simplify().factor()
            return P
    
        def p_distr(self):
            r = Symbol('r', real=True)
            P = lambdify(r, self.p_distr_exp(), 'numpy')
            return P(self.r) 
\end{lstlisting}