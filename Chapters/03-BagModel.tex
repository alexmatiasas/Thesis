\chapter{Modelo de bolsa}
\label{ch-BagModel}

\pagestyle{fancy}
\fancyhf{}
\fancyhead[LE]{\nouppercase{\textbf{\leftmark}\hfill\textit{\rightmark}}}
\fancyhead[RO]{\nouppercase{\textit{\rightmark}\hfill\textbf{\leftmark}}}
\fancyfoot[LE]{\nouppercase{\thepage\hfill Pressure Distribution Inside Nucleons in a Tsallis-MIT Bag Model}}
\fancyfoot[RO]{\nouppercase{Pressure Distribution Inside Nucleons in a Tsallis-MIT Bag Model \hfill \thepage}}

\begin{chaptersummary}[Resumen del capítulo \thechapter: Modelo de bolsa]
    Este capítulo introduce el modelo de bolsa del \gls{mit} como marco fenomenológico para el confinamiento de quarks. Se parte de los fundamentos físicos provistos por la \gls{qcd} y se justifica el modelo desde la libertad asintótica, la inclusión de gluones y la estructura relativista-invariante. Se desarrolla formalmente la versión esférica del modelo, derivando sus ecuaciones de movimiento, condiciones de frontera y soluciones modales, que constituyen la base sobre la cual se construye la extensión no extensiva que proponemos.
\end{chaptersummary}

Comprender cómo los quarks se confinan dentro de los hadrones ha sido uno de los desafíos más importantes de la física de partículas. Aunque la \gls{qcd} provee el marco teórico fundamental para describir las interacciones fuertes, su naturaleza no perturbativa a bajas energías hace difícil una solución exacta para sistemas ligados como los bariones. Bajo este contexto, los modelos fenomenológicos como el modelo de bolsa del \gls{mit} ofrecen una herramienta útil para capturar aspectos esenciales de la dinámica hadrónica.

El modelo de bolsa parte de tres principios motivados por la \gls{qcd}:

\begin{enumerate}[a)]
\item \textbf{Confinamiento y libertad asintótica:} Los quarks están confinados a una región finita de espacio, pero se mueven libremente dentro de ella debido a la debilidad de las interacciones a corta distancia.
\item \textbf{Inclusión de gluones:} Aunque no siempre se introducen explícitamente, las correcciones debidas al intercambio de gluones son esenciales para describir propiedades del espectro hadrónico.
\item \textbf{Condición de singlete de color:} Solo estados globalmente neutros (singletes de color) son permitidos, lo cual es consistente con la ley de Gauss aplicada a campos de color confinados.
\end{enumerate}

A partir de estas ideas, se construye un modelo donde los quarks están confinados dentro de una “bolsa” esférica, cuyo tamaño y energía total resultan del equilibrio entre presión interna y presión de vacío externo, modelada por una constante \( B \) conocida como presión de bolsa.

\section{Formulación del modelo en cavidad esférica}

El enfoque formal del modelo parte de la acción para un campo de quarks sin masa en una cavidad esférica, dada por:

\begin{equation}
W = \int dt \left[ \int_{V} d^3x \left( \frac{i}{2} \bar{\psi} \overleftrightarrow{\partial}_{\mu} \gamma^{\mu} \psi - B \right) - \frac{1}{2} \int_{S} d^2x \bar{\psi} \psi \right]
\end{equation}

Este modelo asume:

\begin{enumerate}[i.]
\item Los quarks se comportan como partículas libres dentro de la cavidad.
\item Fuera de la cavidad, el vacío tiene menor energía, impidiendo la propagación de los campos.
\item Las condiciones de frontera imponen confinamiento y cuantizan las soluciones posibles.
\end{enumerate}

A partir de esta acción se derivan las ecuaciones de movimiento (la ecuación de Dirac) y condiciones de frontera que aseguran el confinamiento de quarks y el equilibrio de presiones. Las soluciones modales resultantes definen una base completa para construir estados hadrónicos dentro del modelo.

\section{Modelo de bolsa: descripción preliminar}

El modelo de bolsa del \gls{mit} ofrece una descripción fenomenológica del confinamiento de quarks, en la cual estos se encuentran libres en el interior de una región finita del espacio —la bolsa— y no pueden escapar debido a una presión negativa que actúa como barrera. En este esquema, los quarks se tratan como partículas relativistas sin masa dentro de la bolsa, mientras que el vacío exterior se modela como un medio con energía más baja.

Esta diferencia energética entre el interior y el exterior se modela mediante la llamada \emph{presión de bolsa} \( B \), que actúa como un parámetro fundamental para estabilizar el sistema. El tamaño de la bolsa queda entonces determinado por el equilibrio entre:

\begin{itemize}
    \item[$\bullet$] La \emph{energía cinética} asociada al confinamiento cuántico de los quarks,
    \item[$\bullet$] y la \emph{energía volumétrica} proporcional a \( B \), que penaliza el crecimiento del volumen.
\end{itemize}

Este modelo fue originalmente formulado por el grupo del MIT \cite{Chodos_1974} y ha sido extensamente utilizado para estimar propiedades de los hadrones ligeros, tales como sus masas, radios y momentos magnéticos.

Desde el punto de vista de la \gls{qcd}, el modelo incorpora dos elementos fundamentales:

\begin{enumerate}
    \item \textbf{Singlete de color:} Las condiciones de frontera del modelo requieren que el estado confinado sea un singlete de color, en consistencia con el principio de confinamiento.
    \item \textbf{Intercambio de gluones:} Aunque el modelo en su forma más simple no incluye campos gluónicos explícitamente, se pueden incorporar como correcciones que contribuyen al espectro de energía de los estados ligados.
\end{enumerate}

A pesar de su simplicidad, el modelo ha mostrado ser efectivo al capturar ciertas características de los hadrones. En esta tesis, extendemos este marco incorporando conceptos de la estadística no extensiva de \gls{tsallis}, con el objetivo de explorar nuevas formas de describir la energía, temperatura y presión internas del protón.