\chapter{Significado F\'isico del Par\'ametro \( q \) de Tsallis}\label{ch-PhysicalMeaningQ}

\fancyhf{} % clear all header fields
\fancyhead[LE]{\nouppercase{\textbf{Cap\'itulo 5. Significado f\'isico del \\par\'ametro \( q \) \hfill\textit{\rightmark}}}}
\fancyhead[RO]{\nouppercase{\textit{\rightmark}\hfill\textbf{Cap\'itulo 5. Significado f\'isico del \\par\'ametro \( q \)}}}
\fancyfoot[LE]{\nouppercase{\thepage\hfill {Pressure Distribution Inside Nucleons in a Tsallis-MIT Bag Model}}}
\fancyfoot[RO]{\nouppercase{{Pressure Distribution Inside Nucleons in a Tsallis-MIT Bag Model} \hfill \thepage}}

\begin{chaptersummary}
Este cap\'itulo analiza el significado f\'isico del par\'ametro de no extensividad \( q \) en el contexto del modelo de bolsa. Se argumenta que \( q \) encapsula de manera natural la f\'isica del confinamiento de quarks y gluones, eliminando la necesidad expl\'icita de introducir una presi\'on de bolsa fija \( B \). As\'i, se establece una relaci\'on entre \( q \) y \( B \), sugiriendo una reinterpretaci\'on del confinamiento en t\'erminos de correlaciones estad\'isticas no extensivas.
\end{chaptersummary}

\section{Confinamiento y Presi\'on de Bolsa}
En el modelo de bolsa tradicional, los quarks est\'an confinados dentro de una regi\'on espacial debido a una presi\'on de vac\'io \( B \) que evita su escape. La densidad Lagrangiana correspondiente se expresa como:

\begin{equation}
{L}_{\mathrm{bag}} = \left( {L}_{\mathrm{QCD}} - B \right) \theta_V
\end{equation}

donde \( \theta_V \) es la funci\'on escal\'on de Heaviside que define el interior de la bolsa.

\section{Interpretaci\'on Alternativa mediante \( q \)}
La presi\'on de bolsa \( B \) es un par\'ametro fenomenol\'ogico que modela la energ\'ia de las fluctuaciones del vac\'io QCD. Sin embargo, al considerar correlaciones estad\'isticas no extensivas, el par\'ametro \( q \) puede reinterpretarse como una medida de dichas correlaciones, evitando la necesidad de introducir un \( B \) expl\'icito.

\section{Relaci\'on entre \( q \) y la Presi\'on de Bolsa}
Bajo esta perspectiva, la presi\'on total en el modelo de bolsa puede reescribirse como:

\begin{equation}
P_q(T,\mu) = P_{q_0}(T,\mu) - B(r)
\end{equation}

o alternativamente, aislando \( q \):

\begin{equation}
q(r) = q_0 + \frac{B(r)}{\dfrac{256 \pi^2}{15} \left( \dfrac{\pi^2}{90} + \dfrac{1}{30} \left( \dfrac{\mu}{T} \right)^2 \right) V T^7}
\label{eq:q-bag-relation}
\end{equation}

donde \( V \) es el volumen efectivo de la bolsa y \( (T, \mu) \) representan la temperatura y potencial qu\'imico locales.

\section{Implicaciones F\'isicas}
Esta relaci\'on implica que el confinamiento puede interpretarse como una consecuencia natural de las correlaciones entre part\'iculas en un sistema fuertemente acoplado, caracterizado por \( q > 1 \). En consecuencia, el modelo estad\'istico de Tsallis proporciona una descripci\'on m\'as fundamental del confinamiento en t\'erminos de la termodin\'amica de sistemas no extensivos.

\begin{remark}[Trabajo Futuro]
La posibilidad de desarrollar modelos de bolsa que prescindan expl\'icitamente de \( B \) y dependan \'unicamente de \( q \) ser\'a investigada en futuras extensiones de este trabajo.
\end{remark}
