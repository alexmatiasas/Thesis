\chapter{NOTAS}

\section{Notas de DeGrand sobre masas y otros parámetros de hadrones ligeros}

Los efectos de la energía cinética de quarks, energía de bolsa, masa de quarks extraño, intercambio de gluones colorados en más bajo orden, y energía asociada con ciertas fluctuaciones son incluidas. Estos son parametrizados por cuatro constantes que tienen significado fundamental y no cambian a partir de multipletes a multipletes. El ajuste al espectro es bueno. El orden de todos los estados es dado correctamente

En la teoría de quarks de estructura de hadrones tenemos las siguientes ideas fundamentales:

\begin{enumerate}
\item Los hadrones están compuestos de quarks
\item Los quarks vienen en varios ``sabores'', los tres de Gell-Mann y Zweig, aumentado quizá por nuevos quarks para nuevos grados hadrónicos de libertad como encanto, y en tres colores.
\item Los quarks interactúan entre ellos relativamente debilmente por el intercambio de un octeto de gluones acoplados sin masa, con color en la manera de Yang-Mills para sus índices de colores. 
\item La interacción debe ser débil a cortas distancias para explicar la escala en experimentos de dispersión de leptones; debe ser débil cerca de la transferencia de momento cero para contar para la falta de grandes renormalizaciones de estimaciones ingenuas del modelo de quarks de transmisiones entre bariones ligeros. 
\item La simetría SU(3) generada por la permutación de índices de color es inquebrantada. 
\item Quarks de diferentes sabores podrían tener masas diferentes para tomar en cuenta para el desglose del observador del SU(3) de Gell-Mann y para las altas masas de estados compuestas de quarks encantados.
\item Finalmente, y esencialmente, los quarks colorados y gluones colorados no son ellos mismos parte del espectro físico. Para cumplir esto, asumimos que los campos fenomenólogicos que describen la dinámica de quarks y gluones no permean todo el espació, sino prefieren estar confinados en el interior de hadrones.
\end{enumerate}

La única manera que conocemos para proporcionar la ``baja excitación'' de materia hadrónica consistente con invariancia de Lorentz es introduciendo un nuevo término, $-{g}_{\mu \nu} {\theta}_{s} B$, en el tensor de energía momento de la teoría. ${\theta}_{s}$ es una función que es unidad donde los campos de quark y gluones están definidos, cero donde no lo están. ${B}$ es una constante universal con las dimensiones de presión. Es entonces una consecuencia exacta de la simetría de color inquebrantada SU(3) que todos los estados tienen números cuánticos convencionales. Es tentador especular sobre un origen para este término poco convencional de algún lugar es más convencional parte de la teoría.



\section{Notas sobre \emph{Baryon structure in the bag theory}}

Un modelo de hadrón es considerado en que una partícula interactuando fuertemente consiste de campos confinados a una región finita de espacio que llamamos ``bolsa''. El confinamiento es logrado en una forma invariante de Lorentz suponiendo que la bolsa posee una energía positiva constante por unidad de volumen, $B$.

Para empezar, el efecto de la densidad de energía $B$ es agregar un término al tensor de energía usual:

\begin{equation}
{T}^{\mu \nu} = {T}_{\mathrm{campos}}^{\mu \nu} - {g}^{\mu \nu} B
\end{equation}

dentro de la bolsa. Fuera de la bolsa ${T}_{\mu \nu}$ se desvanece. Requerir conservación de energía momento lleva a condiciones de frontera sobre los campos en la superficie de la bolsa. Aquí especificamos los campos confinados sean sin masa, campos de espín $\frac{1}{2}$ llevando números cuánticos de quarks con color e interactuando con gluones vectoriales con color sin masa. Una consecuencia exacta de las condiciones de frontera de bolsa para tal interacción es que sólo estados singletes de color (que tienen trialidad cero) pueden existir. La constante de acoplamiento no necesita ser grande para lograr esto. Incluso cuando los campos de quarks son libres dentro de la bolsa, las ecuaciones de campo más las condiciones de frontera no son resolubles exactamente en tres dimensiones espaciales. En vez de eso las resolvemos en lo que parece ser una aproximación razonable a orden cero que es análoga a la ``teoría de Bohr'' para el átomo de hidrógeno: Las ecuaciones clásicas de movimiento admiten una clase de soluciones en que la superficie de la bolsa (en su marco de referencia) es una esfera de radio fijo. Las condiciones de frontera requieren que cada quark ocupe un modo con momento angular total $\frac{1}{2}$. Tratamos estos modos en una cavidad esférica fija como análoga a las órbitas circulares con radio fijo en la vieja teoría cuántica. El radio es entonces cuantizado por la condición que el operador número quark toma valores enteros. Para estos estados, la energía depende en qué modos están ocupados pero no en la forma del momento angular o isoespines de los quarks individuales son agregados para obtener el momento angular total e isoespín del hadrón. Así, por ejemplo, el estado de más baja energía $N(\frac{1}{2} +)$ y $\Delta(\frac{3}{2} +)$ son degenerados. Ya que $B$ es el único parámetro libre somos capaces de hacer predicciones para cantidades dimensionales tales como el momento magnético del protón y radio de carga y los splittings de masa de orden cero en el espectro bariónico. 

\subsection{Cálculos}

Las ecuaciones de movimiento y condiciones de frontera para un campo confinado de espín $\frac{1}{2}$ y sin masa

\begin{equation}
\slashed{\partial} {\psi}_{\alpha}(x) = 0
\end{equation}
 
dentro de la bolsa y

\begin{eqnarray}
i \slashed{n} {\psi}_{\alpha} (x) = {\psi}_{\alpha} (x), \\
\sum_{\alpha} n \cdot {\partial} \bar{\psi}_{\alpha} (x) {\psi}_{\alpha} (x) = 2B
\end{eqnarray}

sobre la superficie de la bolsa. ${n}_{\mu}$ es la 4-normal interior covariante a la superficie de la bolsa. $\alpha$ es un índice de simetría interna que escogemos para designar isoespín y color. Buscamos soluciones para que la frontera sea una esfera estática de radio ${R}_{0}$ en cuyo caso ${n}_{\mu} = (0, - \hat{r})$ y ecuaciones (2) se vuelven

\begin{eqnarray}
-i \hat{r} \cdot \va{\gamma}{\psi}_{\alpha} (x) = {\psi}_{\alpha}(x), \\
-\sum_{\alpha} \frac{\partial}{\partial r} \bar{\psi}_{\alpha}(x) {\psi}_{\alpha} (x) = 2B
\end{eqnarray}

en $r= {R}_{0}$.

La solución general a las ecuaciones (1) y (3) es una superposición (con coeficientes ${a}_{\alpha}$) de soluciones a la ecuación de Dirac libre:

\begin{equation}
{\psi}_{\alpha}(x,t) = \sum_{n \kappa j m} N ({\omega}_{n \kappa j}) {a}_{\alpha} (n \kappa j m) {\psi}_{n \kappa k m} (x, t).
\end{equation}

$j$ y $m$ etiquetan el modo del momento angular y su zomponente $z$. $\kappa$ es el número cuántico de Dirac\footnote{•}, $\kappa = \pm (j + \frac{1}{2})$, que diferencia los dos estados de paridad opuesta para cada valor de $j$.  El índice $n$ etiqueta frecuencias que están a ser determinadas por las condiciones de frontera lineales. La  condición de fontera cuadrática (3b) restringe los modos que pueden ser excitados. Entre otras cosas, 3b permite sólo soluciones para la ecuación de Dirac.
Para $j = \frac{1}{2}$, ya sea $\kappa = - 1$,

\begin{equation}
{\psi}_{n \, -1 \, \frac{1}{2} \, m} (x,t) = \frac{1}{\sqrt{4 \pi}} 
\left( 
\begin{array}{c}
i {j}_{0} ({\omega}_{n, \, -1} r / {R}_{0}) {U}_{m} \\
- {j}_{1} ({\omega}_{n, \, -1} r / {R}_{0}) \sigma \cdot \hat{r}{U}_{m} 
\end{array}
\right) \times {e}^{- i {\omega}_{n, \, -1} t / {R}_{0}}
\end{equation}

o ${\kappa} = 1$

\begin{equation}
{\psi}_{n \, 1 \, \frac{1}{2} \, m} (x,t) = \frac{1}{\sqrt{4 \pi}} 
\left( 
\begin{array}{c}
i{j}_{1} ({\omega}_{n, \, 1} r / {R}_{0}) \sigma \cdot \hat{r} {U}_{m} \\
{j}_{0} ({\omega}_{n, \, 1} r / {R}_{0}) {U}_{m} 
\end{array}
\right) \times {e}^{- i {\omega}_{n, \, 1} t / {R}_{0}}
\end{equation}

${U}_{m}$ es un espinos de Pauli bidimensional y ${j}_{\ell}(z)$ son las funciones de Bessel esféricas. Hemos omitido los índices $j$ sobre ${\omega}_{n \kappa}$ ya que solo $j = \frac{1}{2}$ es de interés en el presente. $N({\omega}_{n \kappa})$ es una constante de normalización escogida para conveniencia futura:

\begin{equation}
N({\omega}_{n \kappa}) \equiv \left( \frac{{\omega}_{n \kappa}^{\phantom{n \kappa} 3}}{2 {R}_{0}^{\phantom{0} 3} ({\omega}_{n \kappa} + \kappa) \sin^{2} {\omega}_{n \kappa}} \right)^{1/2}
\end{equation}

La condición de frontera lineal (3a) genera una condición eigenvalor para los modos de frecuencias ${\omega}_{n \kappa}$

$$
{j}_{0}({\omega}_{n \kappa}) = - \kappa {j}_{1} ({\omega}_{n \kappa}),
$$

o 

\begin{equation}\label{condeigenval}
\tan {\omega}_{n \kappa} = \frac{{\omega}_{n \kappa}}{{\omega}_{n \kappa} + \kappa}
\end{equation}

[Por convención escogemos $n$ positiva (negativa) secuencialmente para etiquetar las raíces positivas (negativas) de la eq 7] Las primeras soluciones a \eqref{condeigenval} son 

\begin{equation}
\begin{array}{ccc}
\kappa = - 1: & {\omega}_{1 \, -1} = 2.04; & {\omega}_{2 \, -1} = 5.40 \\
\kappa = + 1: & {\omega}_{1 \, 1} = 3.81; & {\omega}_{2 \, 1} = 7.00.
\end{array}
\end{equation}

La condición de frontera cuadrática requiere que $\sum_{\alpha} (\partial / \partial r) \bar{\psi}_{\alpha} (x) {\psi}_{\alpha}(x)$ sea independiente de tiempo y dirección para $r={R}_{0}$. La independencia angular requiere que $j = \frac{1}{2}$. Para obtener independencia temporal, ajustamos

\begin{equation}
\sum_{\alpha} {a}_{\alpha}^{*} (n \, \kappa \, j= \frac{1}{2} \, m) {a}_{\alpha} (n' \, \kappa' \, j= \frac{1}{2} \, m') = 0,
\end{equation}

a menos que $n = n'$, $\kappa = \kappa'$ o $n = -n'$, $\kappa = -\kappa'$ en cuyos casos no hay restricción ya que los términos dependientes del tiempo se cancela. La ecuación anterior es una restricción severa sobre los modos que deben ser ocupados. Deberíamos implementar la ecuación anterior requiriendo que para cada grado de libertad interno $\alpha$ sólo un modo normal, ${a}_{\alpha}(n \, \kappa \, j = \frac{1}{2} \, m)$ es excitado. Esto automáticamente será el caso para bariones de tres quarks si son requeridos a ser singletes de color.

Una vez que (9) es satisfecho, los términos independientes del tiempo en (3b) pueden ser coleccionados,

\begin{equation}
\sum_{\alpha \, n \, \kappa \, m} {\omega}_{n \kappa} {a}_{\alpha}^{*}(n \, \kappa \, \frac{1}{2} \, m) {a}_{\alpha}(n \, \kappa \, \frac{1}{2} \, m) = 4 \pi B {R}_{0}^{4},
\end{equation}


\subsubsection*{Conclusiones}

\begin{enumerate}

\item El campo en la bolsa se comporta sobre el promedio como un gas relativista perfecto; que es, la traza del tensor energía momento asociado con el campo, cuando es promediado sobre tiempo y espacio, es cero:
\begin{equation}
\left\langle \int_{R} {\mathrm{d}}^{3} x ({\Theta}_{\mu}^{\mu})_{\mathrm{campo}} \right\rangle = 0
\end{equation}
\item El volumen promediado en el tiempo de una bolsa es proporcional a su energía:
\begin{equation}
E = 4B \langle V \rangle
\end{equation}
\item El estado base y estados excitados más bajos de la bolsa contienen pocos partones de momento promedio de orden ${B}^{1/4}$ encerrados en un volumen de orden ${B}^{-3/4}$. [$B$ tiene la dimensión $(\mathrm{longitud})^{-4}$ con $\hbar=c=1$]
\item En el límite termodinámico la bolsa tiene una temperatura fija, ${T}_{0}$, independiente de su energía. ${T}_{0}$ es de orden ${B}^{-1/4}$. Esto es equivalente a las siguientes declaraciones
\begin{itemize}
\item La energía cinética promedio de los partones es de orden ${T}_{0}$ independiente de la energía de bolsa $E$ proporcionado el último es más grande que ${T}_{0}$: ${E} \gg {T}_{0}$.
\item La densidad de nivel asintótico ${\zeta(E)}$ del sistema es una función exponencial de $E$:
\[
\zeta \sim {e}^{E/{T}_{0}}
\]
\item El número, $N$, de partones más antipartones presente en el hadrón es proporcional a su energía:
\[
N \propto E/{T}_{0}
\]
\end{itemize}
\item Si la dinámica clásica es tal que hay un máximo momento angular del hadrón en una energía total dada $E$, ese máximo debe ser 
\[
{J}_{\mathrm{m\acute{a}x}} = \kappa {B}^{-1/3} {E}^{4 / 3},
\]
donde ${\kappa}$ es una constante adimensional determinada por la dinámica detallada. Si el límite clásico $({\hbar} \rightarrow 0)$ existe, las correcciones cuánticas a esta fórmula se reducirían por potencias de $E$. Si no hay trayectoria clásica a seguir, un argumento plausible sugiere que la trayectoría guía podría ser (para un gran $E$)
\[
{J}_{\mathrm{m\acute{a}x}} = {\kappa}' {B}^{-1/2} {E}^{2} \quad ({\hbar = 1}).
\]
\item El momento angular más probable para una $E$ grande está dada por 
\[
\bar{J} \propto ({B}^{-1/4} E)^{5/6}
\]
\end{enumerate}




































