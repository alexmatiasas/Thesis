% \gls{ }
% To print the term, lowercase. For example, \gls{maths} prints mathematics when used.
% \Gls{ }
% The same as \gls but the first letter will be printed in uppercase. Example: \Gls{maths} prints Mathematics
% \glspl{ }
% The same as \gls but the term is put in its plural form. For instance, \glspl{formula} will write formulas in your final document.
% \Glspl{ }
% The same as \Gls but the term is put in its plural form. For example, \Glspl{formula} renders as Formulas.

\newglossaryentry{boltzman-gibbs}
{
    name=Boltzmann-Gibbs,
    description={La estadística de Boltzmann-Gibbs es una generalización de la mecánica estadística clásica que describe sistemas termodinámicos en equilibrio. La entropía de Boltzmann-Gibbs se define como:
    \[
        S = -{k}_{\mathrm{B}} \sum_{i=1}^W p_i \ln p_i,
    \]
    donde \( {k}_{\mathrm{B}} \) es la constante de Boltzmann, \( p_i \) es la probabilidad del estado \( i \), y \( W \) es el número total de estados posibles. La estadística de Boltzmann-Gibbs es válida para sistemas con interacciones aditivas, como los gases ideales y los sólidos cristalinos}
}

\newglossaryentry{tsallis}
{
    name=Tsallis,
    description={
        La estadística de Tsallis, propuesta por Constantino Tsallis en 1988, es una generalización de la entropía de Boltzmann-Gibbs que se utiliza para describir sistemas con propiedades no extensivas, es decir, sistemas donde las interacciones entre partículas no son aditivas. La entropía de Tsallis se define como:
        \[
            S_q = \frac{1}{q-1} \left( 1 - \sum_{i=1}^W p_i^q \right),
        \]
        donde \( q \) es el parámetro de no extensividad, \( p_i \) es la probabilidad del estado \( i \), y \( W \) es el número total de estados posibles. Para \( q \to 1 \), la entropía de Tsallis se reduce a la entropía de Boltzmann-Gibbs:
        \[
            S_1 = -\sum_{i=1}^W p_i \ln p_i.
        \]
        La estadística de Tsallis ha encontrado aplicaciones en una amplia variedad de campos, incluyendo física de altas energías, sistemas complejos, y física de plasmas, entre otros}
}

\newglossaryentry{gamma_matrices}
{
    name=Matrices Gamma,
    description={Las matrices gamma (\( \gamma^\mu \)) son un conjunto de matrices que cumplen con las relaciones de anticonmutación del álgebra de Clifford:
    \[
        \{ \gamma^\mu, \gamma^\nu \} = 2 g^{\mu \nu} I,
    \]
    donde \( \{ \gamma^\mu, \gamma^\nu \} = \gamma^\mu \gamma^\nu + \gamma^\nu \gamma^\mu \) es el anticonmutador, \( g^{\mu \nu} \) es el tensor métrico de Minkowski, e \( I \) es la matriz identidad. En la representación de Dirac, las matrices gamma tienen la forma:
    \[
        \gamma^0 = \begin{pmatrix}
            I & 0 \\
            0 & -I
        \end{pmatrix}, \quad
        \gamma^i = \begin{pmatrix}
            0 & \sigma^i \\
            -\sigma^i & 0
        \end{pmatrix},
    \]
    donde \( \sigma^i \) son las matrices de Pauli. Estas matrices son fundamentales en la formulación de la ecuación de Dirac, que describe partículas fermiónicas como los quarks y los electrones}
}